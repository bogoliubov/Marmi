\section{Dinamica Topologica}

Consideriamo $X$ spazio metrico compatto, con $\ndom X=C(X,X)$ le funzioni continue e $\aut X=\homeo X$ gli omeomorfismi. 
Denoteremo il sistema dinamico con $(X,d,f)$, dove $d$ è la metrica dello spazio. Iniziamo con alcune definizioni generali.

\begin{defi} $x\in X$ è un \emph{punto errante} se ne esiste un intorno $U$ tale che 
\[\bigcup_n f^n(U)\cap U=\varnothing.\]
L'insieme di tali punti è detto l'\emph{insieme errante} del sistema. Un insieme $A$ è \emph{invariante} se $f(A)\subseteq A$, 
e per omeomorfismi si dice \emph{totalmente invariante} se lo è anche negativamente, cioè per $f^{-1}$, ovvero si ha $f^{-1}(A)=A=f(A)$. 
Gli insiemi di $\alpha$ e $\omega$-limite ($\alpha$ solo per omeo) sono
\[\alpha(x)=\bigcap_{n<0}\obar{\bigcup_{m\leq n}f^m(x)}\qquad \omega(x)=\bigcap_{n>0}\obar{\bigcup_{m\geq n}f^m(x)} \]
ovvero un $\omega$-limite è un punto $y$ limite di una sottosuccessione di $f^n(x)$. 
Se $x\in\omega(x)$ (risp. $x\in\alpha(x)$) si dice positivamente (risp. negativamente) ricorrente, e nel caso accadano entrambe le cose si dice che $x$ è \emph{ricorrente}, 
il che equivale a dire $x$ non errante. I punti ricorrenti sono quindi l'insieme non errante.
\end{defi}

\begin{esercizio}
 Gli insiemi $\omega(x)$, $\alpha(x)$ e l'insieme non errante sono chiusi e invarianti.
\end{esercizio}

%%%%%%%%%%%%%%%%%%%%%%%%%%%%%%%%%%%%%%%%%%%%%%%%%%%%%%%%%%%%%%%%%%%%%%%%%%%%%%%%%%%%%%%%%%%%%%%%%%%%%%%%
\subsection{Transitività}

\begin{defi} $(X,d,f)$ si dice \emph{topologicamente transitivo} se un punto $x\in X$ ha orbita densa, si dice \emph{minimale} se tutte le orbite sono dense.
\end{defi}

\begin{prop}Se $f\in\homeo (X)$ sono equivalenti:
\begin{lista}
	\item $f$ topologicamente transitivo;
	\item se $U$ è un aperto $f$-invariante (ovvero $U=f(U)=f^{-1}(U)$), allora $U=\emptyset$ oppure $\obar{U}=X$;
	\item se $U,V$ sono aperti non vuoti, esiste $N\in\Z$ tale che $f^N(U)\cap V\neq\emptyset$;
	\item $\set{x:\obar{O_f(x)}=X}$ è un $G_\delta$ denso, ovvero è \emph{generico}.
\end{lista}
\end{prop}

\begin{proof}$(1)\implica (2)$: sia $x$ un punto con l'orbita densa e $U$ un aperto $f$-invariante, che supponiamo non vuoto.
Da $O_f(x)\cap U\neq\emptyset$ segue che per qualche $j\in\Z$ $f^j(x)\in U$, quindi (invarianza di $U$) $O_f(x)\subseteq U$,
perciò $X=\obar{O_f(x)}\subseteq\obar{U}$. \\
$(2)\implica (3)$: poniamo $U':=\cup_{n\in\Z}f^n(U)$; $U'$ è non vuoto e invariante $\implica\obar{U'}=X$
$\implica \obar{U'}\cap V\neq\emptyset \implica U'\cap V\neq\emptyset \implica \exists N\ f^N(U)\cap V\neq\emptyset$. \\
$(3)\implica(4)$: sia $\set{U_n}$ una base numerabile di aperti. Ora $\obar{O_f(x)}=X\sse \forall n\ \exists j\ f^j(x)\in U_n
\sse \forall n\ \exists m\ x\in f^m(U_n)\sse x\in\cap_n\cup_{m\in\Z} f^m(U_n)$.
Per ipotesi, per ogni $n$ l'insieme $\cup_{m\in\Z} f^m(U_n)$ è un aperto denso.
Quindi $\set{x:\obar{O_f(x)}=X}$ è un $G_\delta$ e per il teorema di Baire è anche denso. \\
$(4)\implica (1)$: ovvio perché l'insieme $\set{x:\obar{O_f(x)}=X}$, essendo denso, è anche non vuoto.
\end{proof}

\begin{oss}\label{osstransitivita} Se $f:X\to X$ è solo continua (cioè è un endomorfismo), è facile dimostrare che le condizioni $(1)$ e $(3)$ sono
  ancora equivalenti se assumiamo che $X$ non abbia punti isolati.\newline
  Osserviamo inoltre che nel caso $f\in\homeo (X)$ la terza condizione del teorema è equivalente alla \emph{decomponibilità}: 
  $f$ non transitiva se e solo se esistono due aperti disgiunti (totalmente) invarianti.
\end{oss}

\begin{prop}Se $f:X\to X$ è topologicamente transitivo, gli unici \emph{integrali primi} continui (cioè
funzioni $F:X\to\R$ continue tali che $F\circ f=F$) sono le costanti. Il viceversa, in generale, è falso (cfr. gli Esempi).
\end{prop}

\begin{proof}Una tale $f$ deve essere costante sull'orbita di $x$ (il solito punto con orbita densa), quindi per continuità anche
sulla chiusura $\obar{O_f(x)}=X$.
\end{proof}

\begin{defi} $(X,d,f)$ è \emph{topologicamente mescolante} se dati aperti $U,V$ esiste $N\in\Z$ tale che per ogni $n\geq N$ vale $f^n(U)\cap V\neq \emptyset$.\end{defi}

\begin{oss} Per la caratterizzazione vista, mescolante implica transitivo. Non vale il viceversa, e un esempio sono le rotazioni di $S^1$. 
\end{oss}



%%%%%%%%%%%%%%%%%%%%%%%%%%%%%%%%%%%%%%%%%%%%%%%%%%%%%%%%%%%%%%%%%%%%%%%%%%%%%%%%%%%%
\subsection{Esempi}

I seguenti esempi di sistemi dinamici sono qui presentati come dinamiche topologiche, ma torneranno anche nello studio di quelle misurabili.

\begin{esempio}[Rotazioni Irrazionali] Consideriamo le rotazioni su $\T^1=\R/\Z$: $f(x)=x+\alpha$. 
Non è difficile vedere che se $\alpha\in\Q$ tutti i punti sono periodici, mentre se $\alpha\in\irr$ ogni orbita è densa, ovvero il sistema è minimale 
(cfr. la versione generalizzata al toro $d$-dimensionale nella parte sui Flussi). 
Questo non è un caso, come mostra il risultato seguente. \end{esempio}
\begin{lemma}[facoltativo] Sia $G$ un gruppo topologico: se $g_0\in G$ è tale che l'applicazione $g\mapsto g_0\cdot g$ (oppure $g\mapsto g\cdot g_0$) è topologicamente transitiva, 
allora è anche minimale. Se inoltre $G$ è metrizzabile, è un gruppo abeliano. 
\end{lemma}
\begin{proof}
 Se $\{g_0^kg_1\}_k$ è densa, poichè $g_0^kg=(g_0^kg_1)(g_1^{-1}g)$ si ha che l'orbita di $g$ si ottiene con una traslazione, che per ipotesi è un omeomorfismo, di un'orbita densa. 
 Per il secondo punto, abbiamo che il sottogruppo ciclico $G'=\{g_0^kg_1\}_k$ è denso, quindi poichè lo spazio è metrico si usa il criterio di continuità per successioni: 
 se $g_n\rightarrow g$ e $h_n\rightarrow h$ con $g_n, h_n \in G'$, si ha $[g,h]\leftarrow [g_n,h_n]=e$, e quindi per continuità $[g,h]=e$.
\end{proof}




\begin{esempio}[Raddoppiamento e Dinamica Simbolica] Le dinamiche date da successioni di numeri sono importanti perchè sono un oggetto a cui 
spesso ci si riconduce nello studio dei sistemi dinamici. Consideriamo
\[ \xymatrix{ \{0,1\}^\N \ar[r]^\sigma \ar[d]_\pi & \{0,1\}^\N \ar[d]^\pi \\ [0,1] \ar[r]_S & [0,1] } \]
 dove $S(x)=2x(\mbox{mod } 1)$ è il \emph{raddoppiamento} o \emph{dente di sega}, $\sigma(a_1,a_2,\dots)=(a_2,a_3,\dots)$ è lo \emph{shift} e 
 \[\pi(a_1,a_2,\dots)=\sum_1^\infty a_k 2^{-k}.\]
 La mappa $\pi$ è iniettiva e dunque $S$ risulta essere un sottosistema di $\sigma$. In questo modo si vede subito che il raddoppiamento è transitivo non minimale: 
 lavorando in $\{0,1\}^\N$, la successione ottenuta concatenando tutte le possibili parole finite di soli $0$ e $1$ (che sono numerabili) corrisponde a un punto con orbita densa; 
 invece le successioni periodiche corrispondono a punti periodici (se ne trovano di ogni ordine). 
 \Eacc importante osservare che i due sistemi non possono essere coniugati perchè $[0,1]$ è connesso e $\{0,1\}^\N$ è omeomorfo all'insieme di Cantor, 
 totalmente sconnesso. \newline
 Osserviamo inoltre che il raddoppiamento è un esempio di sistema topologicamente mescolante.
\end{esempio}



\begin{esempio}[Tenda e Logistic Map] Un altro esempio di dinamiche coniugate: 
su $[0,1]$ sono coniugate tramite $h(x)=\sin^2\left(\frac{\pi x}{2}\right)$ la mappa \emph{tenda} 
$f(x)=\begin{cases}2x \quad x\leq 1/2 \\ 2-2x \quad x\geq 1/2\end{cases}$ e la \emph{mappa logistica} $Q_4(x)=4x(1-x)$, $Q_4\circ h=h\circ f$.
La mappa tenda è transitiva, infatti per gli intervalli della forma $\bra{\frac{i}{2^k},\frac{i+1}{2^k}}$ l'immagine
tramite $T^n$ è definitivamente tutto $[0,1]$ (si dice che $T$ è \emph{espansiva}). Per cui anche la mappa logistica è transitiva (vedi anche il prossimo esempio).
\end{esempio}



\begin{esempio}[Transitività e fattori] Se $g$ è un fattore di $f$ e $f$ è topologicamente transitiva, allora lo è anche $g$.
Infatti per qualche $h:X\to Y$ surgettiva è $gh=hf$, quindi $g^k h=h f^k$, perciò
se $x\in X$ ha orbita densa abbiamo $O_g(h(x))=h(O_f(x))$, che è un insieme denso essendo $h$ surgettiva.\newline
Un esempio (un po' improprio perchè $S$ è discontinua) è il seguente: se
\[S(x):=\begin{cases}2x &\text{se }0\le x<\mz \\ 2x-1 &\text{se }\mz\le x\le 1\end{cases}\]
allora commuta il diagramma
\[ \xymatrix{X \ar[r]^S \ar[d]_T & X \ar[d]^T \\ X \ar[r]_T & X} \]
per cui $T$ è un fattore di $S$, e dunque abbiamo un'altra dimostrazione che $T$ è transitivo. 
\end{esempio}



\begin{esempio}[facoltativo] Avere solo integrali primi costanti non implica transitività. Sia $T:=\R^2/\Z^2$ il $2$-toro e $A\in SL(2,\Z)$ una trasformazione ergodica
su $T$, ad esempio il \emph{gatto di Arnold},
\[A=\begin{pmatrix} 2 & 1 \\ 1 & 1 \end{pmatrix}\]
Costruiamo $X$ partendo da due copie del toro, $T$ e $T'$ e identificando (quoziente topologico) i loro elementi neutri. 
La mappa su $T\sqcup T'$ che si ottiene accostando $A:T\to T$ e $A:T'\to T'$ induce una mappa $f:X\to X$.
$f$ non è topologicamente transitiva essendo i sottospazi $T$ e $T'$ invarianti (stiamo identificando $T$ con la sua proiezione a quoziente,
come pure $T'$).
Un integrale primo $F:X\to \R$ però è costante: infatti che $\restr{f}{T}:T\to T$ è topologicamente transitivo 
(essendo ergodico, vedi il relativo capitolo),
perciò $F$ è costante su $T$ e lo stesso vale per $T'$. Siccome (in $X$) $T$ e $T'$ si intersecano, deduciamo che $F$ è costante su tutto $X$. 
\end{esempio}



%%%%%%%%%%%%%%%%%%%%%%%%%%%%%%%%%%%%%%%%%%%%%%%%%%%%%%%%%%%%%%%%%%%%%%%%%%%%%%%%%%%%%%%%%%%%%%%%%%%%%%%%%%%%%
\subsection{Sistemi Caotici}


\begin{defi} $(X,d,f)$ è \emph{caotico secondo Devaney} se è topologicamente transitiva e l'insieme dei punti periodici $\per(f)$ è denso.
\end{defi}

% DEFINIZIONE RIMPIAZZATA DA QUELLA SEGUENTE
% \begin{defi}$f$ ha \emph{dipendenza sensibile} dalle condizioni iniziali se c'è un $\delta>0$ tale che
% per ogni $x\in X$ e per ogni suo intorno $U$ si possano sempre trovare un $y\in U$ e un $n\ge 0$ per cui
% $d\pa{f^n(x),f^n(y)}>\delta$. In altre parole, pensando l'intorno $U$ come un errore di approssimazione di $x$,
% per quanto sia buona l'approssimazione c'è sempre un $y\approx x$ la cui orbita (a qualche istante $n$) si separa da quella di $x$
% più della tolleranza $\delta$ fissata.
% \end{defi}

\begin{defi}$(X,d,f)$ ha \emph{dipendenza sensibile dai dati in} $x_0\in X$ se 
 \[\exists \epsilon>0:\quad \forall U \mbox{ intorno di } x_0 \quad \exists y_0\in U:\quad\exists n\geq 0: \quad d(f^nx_0,f^ny_0)>\epsilon\]
 e se questo vale $\forall x\in X$ diremo che la dinamica ha \emph{dipendenza sensibile}.
 Viceversa $(X,d,f)$ è \emph{stabile secondo Lyapunov} in $x_0\in X$ se
 \[\forall \epsilon>0 \quad \exists U \mbox{ intorno di } x_0: \quad \forall y_0\in U\quad\forall n\geq 0 \quad d(f^nx_0,f^ny_0)<\epsilon\]
 (contronominale della precedente) ovvero se le iterate $\{f^n\}$ sono equicontinue.
\end{defi}

\begin{teo}Se $f$ è caotica e $X$ non ha punti isolati, $f$ ha dipendenza sensibile dalle condizioni iniziali.
\end{teo}

\begin{lem}Se ci sono due orbite periodiche distinte, esiste una costante $c>0$ tale che per ogni $x\in X$
c'è un punto periodico $p$ la cui orbita dista tutta più di $c$ da $x$. In altre parole
$d(x,f^k(p))>c$ per ogni $k\ge 0$.
\end{lem}

\begin{proof}Siano $y$ e $z$ periodici con orbite distinte (e quindi disgiunte).
Sia $c>0$ tale che $d\pa{O_f(y),O_f(z)}>2c$. Per ogni $x\in X$ abbiamo
\[ d\pa{x,O_f(y)}+d\pa{x,O_f(z)}\ge d\pa{O_f(y),O_f(z)}, \]
quindi almeno uno tra $d\pa{x,O_f(y)}$ e $d\pa{x,O_f(z)}$ è $>c$.
\end{proof}

\begin{proof}[Dimostrazione del teorema]Se esiste una sola orbita periodica,
da $\obar{\per(f)}=X$ segue che $X$ è finito e quindi tutti i punti sarebbero isolati, assurdo.\newline
Perciò siamo nelle ipotesi del lemma: poniamo $\delta:=\frac{c}{4}$, con $c$ la costante che il lemma ci fornisce.
Sia $x\in X$ e $U$ un suo intorno, che possiamo supporre contenuto in $B(x,\delta)$. Per densità di $\per(f)$ esiste $q\in U$ periodico,
sia $n$ il periodo. Per il lemma esiste un punto $p$ periodico con $d\pa{x,O_f(p)}>c=4\delta$. Inoltre per transitività
esiste un $z\in U$ che nella sua vita resta vicino almeno $\delta$ all'orbita di $p$ per almeno $n$ passi consecutivi:
infatti posto $W_i:=B\pa{f^i(p),\delta}$ (per $i=1,\dots,n$) e $V:=\cap_{i=1}^n f^{-i}(W_i)$ (non vuoto perché $p\in V$)
sappiamo che esistono $z\in U$ e $k\ge 0$ tali che $f^k(z)\in V$ (cfr. Osservazione \ref{osstransitivita}). $z$ soddisfa
$f^{k+i}(z)\in W_i$ per $i=1,\dots,n$. Per uno di questi indici $i$, $k+i$ è multiplo di $n$,
perciò $f^{k+i}(q)=q$, da cui otteniamo
\[ \begin{split} &d\pa{f^{k+i}(q),f^{k+i}(z)}\ge d\pa{q,f^i(p)}-d\pa{f^{k+i}(z),f^i(p)}\ge \\
&\ge d\pa{x,f^i(p)}-d\pa{x,q}-d\pa{f^{k+i}(z),f^i(p)}>4\delta-\delta-\delta=2\delta \end{split} \]
usando tante volte la disuguaglianza triangolare. Ma allora almeno uno tra
$d\pa{f^{k+i}(x),f^{k+i}(q)}$ e $d\pa{f^{k+i}(x),f^{k+i}(z)}$ è maggiore di $\delta$ (sempre per la triangolare), che è la tesi
essendo $q,z\in U$.
\end{proof}



