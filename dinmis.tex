\section{Dinamica Misurabile}

Sia $(X,\A,\mu)$ uno spazio di probabilità. Un sistema dinamico misurabile è dato da una mappa $f:X\rightarrow X$ che preserva la misura, 
ovvero tale che $f^{-1}\A\subseteq \A$ (misurabile) e per ogni $A\in\A$ vale $\mu(f^{-1}A)=\mu(A)$. (Scegliamo questa definizione con $f^{-1}$ coerentemente con la nozione di misura immagine.)
I sistemi dinamici topologici considerati finora sono anche sistemi misurabili, come prova il seguente

\begin{teo}[Krylov-Bogoliubov] Dati $X$ spazio metrico compatto e $f:X\to X$ continua, esiste sempre una probabilità $f$-invariante.
\end{teo}

\begin{proof}Sia $C(X)$ lo spazio (di Banach) delle funzioni continue a valori reali su $X$. Sappiamo che il suo duale è lo spazio $C(X)'=\mathcal{M}(X)$ delle misure finite con segno.
Sia $\mu$ una qualsiasi misura di probabilità su $X$ (ad esempio una delta). ``Spalmiamo'' questa misura tramite $f$, ottenendo misure sempre più vicine all'essere invarianti:
poniamo $\mu_n:=\frac{1}{n}\sum_{k=0}^{n-1}(f^k)_*\mu$ (cioè $\mu_n$ è la media dei pushforward di $\mu$ tramite le prime $n$ iterate di $f$). \\
Le misure $\mu_n$ sono tutte positive e di probabilità, quindi per ogni $n$ $\norm{\mu_n}=1$. Ma la palla unitaria (chiusa) di $\mathcal{M}(X)$, dotata
della topologia debole*, è compatta (per Banach-Alaoglu) ed è anche metrizzabile (essendo $C(X)$ separabile). Quindi esiste una sottosuccessione convergente
debole*: $\mu_{n_k}\overset{*}{\rightharpoonup}\nu$. Il limite $\nu$ è una misura positiva (essendo per ogni $\phi\in C(X)$ nonnegativa $\int_X \phi\,d\nu=
\lim_{n\to\infty}\int_X\phi\,d\mu_n\ge 0$) e di probabilità (perché $\nu(X)=\int_X 1\,d\nu=\lim_{n\to\infty}\int_X 1\,d\mu_n=1$). \\
Verifichiamo che $\nu$ è $f$-invariante: per ogni $\phi\in C(X)$ vale
\[ \begin{split}&\int_X \phi\,d\nu-\int_X \phi\,d(f_*\nu)=\int_X \pa{\phi-\phi\circ f}\,d\nu=\lim_{n\to\infty}\int_X \pa{\phi-\phi\circ f}\,d\mu_n \\
&=\lim_{n\to\infty}\pa{\int_X \frac{\phi+\phi\circ f+\dots+\phi\circ f^{n-1}}{n}\,d\mu-\int_X \frac{(\phi\circ f)+\dots+(\phi\circ f)\circ f^{n-1}}{n}\,d\mu} \\
&=\lim_{n\to\infty}\int_X\frac{\phi-\phi\circ f^n}{n}\,d\mu_n=0 \end{split} \]
essendo $\abs{\int_X\frac{\phi-\phi\circ f^n}{n}\,d\mu_n}\le\frac{2}{n}\norm{\phi}_\infty\to 0$. Questo dice che $\nu=f_*\nu$ come elementi del duale di $C(X)$,
quindi (siccome $C(X)'=\mathcal{M}(X)$) le misure $\nu$ e $f_*\nu$ coincidono.
\end{proof}

Una bigezione che coniuga un sistema misurabile con uno qualsiasi induce una misura invariante: è questo il caso dei seguenti esempi.

\begin{esempio} La mappa tenda $T$ preserva la misura di Lebesgue. \Eacc coniugata alla mappa logistica $Q_4(x)=4x(1-x)$ tramite $\sin^2\left(\frac{\pi x}{2}\right)$,
 e da ciò si ricava la misura invariante $\frac{dx}{\pi\sqrt{x(1-x)}}$. 
\end{esempio}

\begin{esempio} La mappa logistica $Q_4$ è coniugata alla \emph{mappa di Ulam} 
 \[U:\R\rightarrow\R \qquad x\mapsto \ln \left(\frac{4 e^x}{(1-e^x)^2}\right)\]
tramite la \emph{logic map}
 \[k:[0,1]\rightarrow \R \qquad x\mapsto \ln\left(\frac{x}{1-x}\right). \]
Otteniamo in questo modo una misura inveriante per la mappa di Ulam, $\frac{dx}{\pi \pa{e^{\frac{x}{2}}+e^{-\frac{x}{2}}}}$.
\end{esempio}

\begin{esempio} Consideriamo la mappa $f:\R\ni x\mapsto \frac{1}{2}\pa{x-\frac{1}{x}}$, che è la mappa del metodo di Newton per il polinomio $1+x^2$. 
Si verifica facilmente che ha come misura invariante la distribuzione di Cauchy $\frac{dx}{\pi (1+x^2)}$, nota per non avere momenti di nessun ordine. 
Vogliamo ottenere questo risultato tramite un coniugio. 
In generale, dato un polinomio reale di grado $2$ con radici distinte, la mappa del metodo di Newton $f(x)=x-\frac{p(x)}{p'(x)}$ 
si può vedere come mappa sulla sfera di Riemann, $\widehat \C\rightarrow\widehat\C$, e come tale è coniugata a $Q(x)=x^2$ tramite la trasformazione di M\"obius che porta i due punti
fissi (le radici del polinomio) in $0$ e $\infty$. Nel nostro caso la trasformazione è $h(x)=\frac{x-i}{x+i}$, che mappa il semipiano superiore nel disco unitario, e l'asse reale
nella circonferenza unitaria. Restringendo il coniugio al bordo di questi due domìni, stiamo coniugando la nostra $f$ originale con il raddoppiamento su $S^1$, 
e con un semplice calcolo si ricava che il coniugio è $S^1\ni\theta\mapsto -\cot \frac{\theta}{2}\in\R$; trasportando la misura di Lebesgue su $S^1$ si ha quanto detto. 
\end{esempio}

Se ci mettiamo nel caso $f:[0,1]\rightarrow [0,1]$ a tratti monotona e $C^1$ (ammettendo partizioni in intervalli al più numerabili), grazie alla formula di cambio di variabili
non è difficile controllare se una misura assolutamente continua rispetto Lebesgue $\rho dx$ è invariante: con un facile conto si ottiene che $\rho$ deve soddisfare
\[\rho(x)=\LL\rho(x)=\sum_{y\in f^{-1}(x)} \frac{\rho(y)}{|f'(y)|}\]
in cui $\LL$ è detto \emph{operatore di trasferimento} o di Perron-Frobenius.

\begin{esercizio} Con il metodo appena introdotto, è facile verificare che la mappa di Gauss $G(x)=\left\{\frac{1}{x}\right\}$ in $(0,1)$ ammette come misura invariante
 $\frac{dx}{(1+x)\ln 2}$.\end{esercizio}



%%%%%%%%%%%%%%%%%%%%%%%%%%%%%%%%%%%%%%%%%%%%%%%%%%%%%%%%%%%%%%%%%%%%%%%%%%%%%%%
\subsection{Ergodicità}

\begin{defi} Se $A\in\mathcal{A}$, $T\in\N^+$, la \emph{frequenza media delle visite} ad $A$ dell'orbita di $x$ tra $0$ e $T$ è
\[\nu(x,T,A):=\frac{1}{T}\sum_{j=0}^{T-1}\chi_A\pa{f^j(x)},\]
scriveremo inoltre $\obar{\nu}(x,A):=\obar{\lim}_{T\to\infty}\nu(x,T,A)$ e $\ubar{\nu}(x,A):=\ubar{\lim}_{T\to\infty}\nu(x,T,A)$.
Più in generale, dato un osservabile $\phi$, ovvero una mappa $X\rightarrow \R$ misurabile, definiamo la \emph{somma di Birkhoff} $T$-esima
\[(S_T\phi)(x):=\sum_{j=0}^{T-1}(\phi\circ f^j)(x)\]
e la relativa media temporale $\frac{1}{T}S_T\phi(x)$. Un osservabile costante sulle orbite si dice \emph{integrale primo}.
\end{defi}

Dimostreremo più avanti il seguente risultato, 
che ci permette di dare la definizione di sistema ergodico come sistema in cui la statistica delle orbite coincide con la probabilità a priori.

\begin{teo}[Birkhoff]Per ogni $A\in\mathcal{A}$ esiste q.o. $\nu(x,A):=\lim_{T\to\infty}\nu(x,T,A)$. \\
Più in generale per ogni $\phi\in L^1(X)$ esiste q.o. il $\lim_{T\to\infty}\frac{(S_T\phi)(x)}{T}$.
\end{teo}

\begin{defi}$(X,\mathcal{A},\mu,f)$ è \emph{ergodico} se per ogni $A\in\mathcal{A}$ vale $\nu(x,A)=\mu(A)$ q.o.
\end{defi}

\begin{teo}Sono equivalenti:
\begin{lista}
	\item $X$ ergodico 
	\item per ogni $A\in\mathcal{A}$ con $f(A)\subseteq A$, uno tra $A$ e $A^c$ è trascurabile 
	\item ogni integrale primo $\phi$ (cioè $\phi\in L^1(X)$ tale che $\phi=\phi\circ f$ q.o.) è costante q.o. 
	\item per ogni $\phi\in L^1(X)$ vale $\frac{1}{T}(S_T\phi)(x)\to\int_X \phi\,d\mu$ per q.o. $x$ 
	\item per ogni $A,B\in\mathcal{A}$ vale $\lim_{n\to\infty}\frac{1}{n}\sum_{j=0}^{n-1}\mu\pa{f^{-j}(A)\cap B}=\mu(A)\mu(B)$. 
\end{lista}
\end{teo}

\begin{proof} Seguendo l'ordine:\begin{itemize}
    \item[$(1)\implica(2)$] Sia $A\in\mathcal{A}$ tale che $f(A)\subseteq A$; possiamo assumere $\mu(A)>0$. Poichè $f^j(x)\in A$ per ogni $j\ge 0$, per ogni $x\in A$
	    è $\nu(x,A)=1$, ma $\nu(x,A)$ può essere diverso da $\mu(A)$ solo su un insieme trascurabile, quindi $\mu(A)=1$.
    \item[$(2)\implica(3)$] Per chiarire l'idea vediamo prima il caso in cui $\phi=\phi\circ f$ ovunque.
	    Poniamo $A_\gamma:=\set{x:\phi(x)\le\gamma}$ e osserviamo che
	    \[ f^{-1}(A_\gamma)=\set{x:\phi\circ f(x)\le\gamma}=A_\gamma. \]
	    Quindi $f\pa{f^{-1}(A_\gamma)}\subseteq A_\gamma=f^{-1}(A_\gamma)$, cioè $f^{-1}(A_\gamma)$ è invariante nel senso del punto $(2)$,
	    da cui $\mu(A_\gamma)=\mu\pa{f^{-1}(A_\gamma)}\in\set{0,1}$.
	    Ma la funzione $\gamma\mapsto\mu(A_\gamma)$ è crescente e ha limiti $0$ per $\gamma\to -\infty$ e $1$ per $\gamma\to +\infty$,
	    perciò $\obar{\gamma}:=\inf\set{\gamma:\mu(A_\gamma)=1}\in\R$. Essendo anche 
	    \[A_{\obar{\gamma}}=\bigcap_n A_{\obar{\gamma}+\frac{1}{n}} \qquad \set{x:\phi(x)<\obar{\gamma}}=\bigcup_n A_{\obar{\gamma}-\frac{1}{n}}\]
	    otteniamo $\mu(A_{\obar{\gamma}})=1$
	    e $\mu\pa{\set{x:\phi(x)<\obar{\gamma}}}=0$, cioè $\phi\equiv\obar{\gamma}$ quasi ovunque. 
	    Vediamo ora il caso in cui $\phi=\phi\circ f$ q.o.: sia $B_\gamma:=\set{x:\phi\circ f\le\gamma}$;
	    siccome $\mu\pa{A_\gamma\Delta B_\gamma}=0$ possiamo scrivere $A_\gamma=B_\gamma\Delta N$ con $N$ trascurabile.
	    \[B_\gamma=f^{-1}(A_\gamma)=f^{-1}(B_\gamma)\Delta f^{-1}(N),\]
	    da cui, posto $\tilde{N}:=f^{-1}(N)$, trascurabile,
	    \[f\pa{B_\gamma\Delta\tilde{N}}\subseteq B_\gamma=\pa{B_\gamma\Delta\tilde{N}}\Delta\tilde{N},\]
	    ovvero $B_\gamma\Delta\tilde{N}$ è ``quasi invariante''. Per l'Osservazione \ref{ossergodicita} e l'ipotesi $(2)$ segue
	    \[\mu(A_\gamma)=\mu(B_\gamma)=\mu(B_\gamma\Delta\tilde{N})\in\set{0,1}\]
	    e si conclude come nel caso precedente.
    \item[$(3)\implica (4)$] Possiamo assumere $\phi\ge 0$.
	    Essendo $\norm{\phi\circ f^j}_1=\norm{\phi}_1$ per ogni $j\ge 0$, abbiamo $\norm{\frac{1}{T}S_T\phi}_1\le \norm{\phi}_1$ per ogni $T>0$.
	    Per il teorema di Birkhoff è ben definito q.o. 
	    \[\tilde{\phi}:=\lim_{T\to\infty}\frac{1}{T}S_T\phi\]
	    e per il lemma di Fatou $\tilde{\phi}\in L^1(X)$. Ora $\tilde{\phi}$ è un integrale primo, quindi è costante q.o., diciamo $\tilde{\phi}=c$ q.o. 
	    Integrando otteniamo 
	    \[c=\int_X\tilde{\phi} \le \liminf_{n\to\infty}\int_X\frac{S_T\phi}{T}\,d\mu=\int_X\phi\,d\mu\]
	    (dato che $f$ conserva $\mu$). Quindi 
	    \[\lim_{T\to\infty}\frac{1}{T}S_T\phi\le \int_X\phi\,d\mu.\] 
	    Per i troncamenti $\phi\wedge N$, per convergenza dominata, vale di più: 
	    \[\lim_{T\to\infty}\frac{1}{T}S_T(\phi\wedge N)=\int_X \phi\wedge N\,d\mu\]
	    \[ \int_X\phi\wedge N=\lim_{T\to\infty}\frac{1}{T}S_T(\phi\wedge N)\le\lim_{T\to\infty}\frac{1}{T}S_T\phi\le \int_X\phi\,d\mu \]
	    e mandando $N\to\infty$ otteniamo la tesi per convergenza monotona.
    \item[$(4)\implica (1)$] Basta scegliere $\phi:=\chi_A$.
    \item[$(4)\implica(5)$] Ponendo $\phi:=\chi_A$ otteniamo $\phi\circ f^j=\chi_{f^{-j}(A)}$, da cui
	    \[ \frac{1}{n}\sum_{j=0}^{n-1}\mu\pa{f^{-j}(A)\cap B}=\int_B\frac{S_n\phi}{n}\,d\mu\to\int_B\pa{\int_X\phi\,d\mu}\,d\mu=\mu(B)\mu(A) \]
	    (abbiamo usato il teorema di convergenza dominata).
    \item[$(5)\implica (2)$] Dato $A$ invariante, scegliamo $B:=A^c$ e osserviamo che $A\subseteq f^{-j}(A)$. Perciò
	    \[ \mu(A)=\mu\pa{f^{-j}(A)}=\mu(A)+\mu\pa{f^{-j}(A)\cap B}, \]
	    da cui segue che $\mu\pa{f^{-j}(A)\cap B}=0$ per ogni $j\ge 0$. Ma allora da $(5)$ abbiamo $\mu(A)\mu(B)=0$.
      \end{itemize}
\end{proof}

\begin{oss}\label{ossergodicita} Considerando il teorema appena visto, notiamo che \begin{itemize}
    \item Se un insieme $A\in\mathcal{A}$ è ``quasi invariante'', nel senso che $f(A)\subseteq A\cup N$ per qualche
      $N$ trascurabile, $A$ è equivalente a un insieme $A'$ invariante (cioè $\mu(A\Delta A')=0$ e $f(A')\subseteq A'$):
      basta porre $N':=\bigcup_{j=0}^\infty$ e $A':=A\setminus N'$. Ora se $x\in A'$ abbiamo
      $f(x)\in A\cup N$ e $f(x)\nin N'$ (perché $f^{-1}(N')\subseteq N'$), quindi $f(A')\subseteq (A\cup N)\setminus N'=A\setminus N'=A'$.
    \item Se il sistema dinamico è \emph{strongly mixing}, ovvero per ogni $A,B\in\mathcal{A}$ vale
      \[\lim_{j\to\infty}\mu\pa{f^{-j}(A)\cap B}=\mu(A)\mu(B)\] 
      (\emph{decadimento delle correlazioni}), la condizione $(5)$ del teorema è banalmente verificata, e quindi il sistema è ergodico.
    \item Nei punti $(3)$ e $(4)$ possiamo sostituire $L^1$ con un qualsiasi $L^p$, infatti abbiamo usato solo il fatto che le funzioni caratteristiche sono $L^1$, 
      ma sono anche $L^p$ per ogni $p$.
\end{itemize}
\end{oss}


\begin{oss}[facoltativa]Sia $(X,d,f)$ uno spazio metrico compatto con una trasformazione $f:X\to X$ continua che preserva
una misura $\mu$: $X$ ha contemporaneamente le strutture di sistema dinamico topologico
e misurabile, ed è naturale chiedersi se ci sono relazioni tra la transitività topologica e l'ergodicità.
Assumendo che ogni aperto non vuoto abbia misura positiva, l'ergodicità implica la transitività:
dati $U,V$ aperti non vuoti è $\lim_{n\to\infty}\frac{1}{n}\sum_{j=0}^{n-1}\mu\pa{f^{-j}(U)\cap V}=\mu(U)\mu(V)>0$,
quindi per qualche $N\ge 0$ abbiamo $\mu\pa{f^{-N}(U)\cap V}>0$.
Dunque $f^{-N}(U)\cap V\neq\emptyset$ e la sua immagine tramite $f^N$ è inclusa in $U\cap f^N(V)$, perciò
quest'ultimo insieme è non vuoto e questa è la tesi.
\end{oss}

\begin{esempio}Sia $\alpha\in\irr$, la rotazione $R_\alpha(x)=x+\alpha$ su $\T^1$ è ergodica. Per quanto osservato, possiamo restringerci a osservabili in $L^2$.
Ora, dato un integrale primo $\phi\in L^2(X)$, i suoi coefficienti di Fourier soddisfano
\[ \widehat{\phi}(k)=\widehat{\phi\circ R_\alpha}(k)=e^{2\pi i k\alpha}\widehat{\phi}(k) \]
per ogni $k\in\Z$. Ma se $k\neq 0$ abbiamo $e^{2\pi i k\alpha}\neq 1$ (perché $\alpha$ è irrazionale), quindi
$\widehat{\phi}(k)=0$ per $k\neq 0$ e questo dice che $\phi$ è costante q.o.
\end{esempio}




%%%%%%%%%%%%%%%%%%%%%%%%%%%%%%%%%%%%%%%%%%%%%%%%%%%%%%%%%%%%%%%%%%%%%
\subsection{Teoremi Ergodici}

I teoremi di questa sezione garantiscono la convergenza delle somme di Birkhoff in $L^2$ (von Neumann) e quasi ovunque (Birkhoff).

\begin{teo}[ergodico di von Neumann] Sia $U$ un'isometria di $H$ spazio di Hilbert, allora
 \[\forall\phi\in H \quad \exists \lim_{n\rightarrow\infty}\frac{1}{n}\sum_{j=0}^{n-1}U^j\phi:=\hat\phi\]
 e inoltre $U\hat\phi=\phi$. Per cui in una dinamica misurabile $(X,\A,\mu,f)$, considerando nello spazio di Hilbert $L^2_\C(X,\A,\mu)$ l'\emph{operatore di Koopman} $U_f$ tale che $U_f \phi=\phi\circ f$, che è un'isometria essendo $\mu$ $f$-invariante, si ha che le somme di Birkhoff convergono in $L^2$. 
 \end{teo}

\begin{oss}
 In dimensione finita, cioè per $U$ unitario in un $\C$-spazio $k$-dimensionale, 
 il teorema spettrale permette di diagonalizzare $U$ riconducendosi al caso monodimensionale sugli autospazi: $Uz=e^{i\alpha}z$ per $z\in\C$. 
 Allora
 \[\frac{1}{n}\sum_{j=0}^{n-1}e^{ij\alpha}z=\begin{cases} z \qquad \alpha=0 \\ 
 \frac{1}{n}\frac{1-e^{in\alpha}}{1-e^{i\alpha}}z \overset{n\rightarrow\infty}{\longrightarrow}0 \qquad \alpha\neq 0\end{cases}. \]
 Per applicare questo ragionamento in dimensione infinita servirebbe il teorema spettrale relativo (Stone). La dimostrazione che daremo si deve a Riesz.
\end{oss}

\begin{proof}
 Sia $D=\{\phi\in H|\exists\psi\in H : \phi=U\psi-\psi\}$ (insieme dei \emph{cobordi}); anzitutto mostriamo che $\forall \theta \in D^\bot \quad U\theta=\theta$. 
 Infatti se $\theta \in D^\bot$, si ha
 \[(\theta, U\theta-\theta)=0\]
 \[(U\theta,U\theta-\theta)=(U\theta,U\theta)-(U\theta,\theta)=(\theta-U\theta,\theta)=0\]
 sommando quindi $(U\theta-\theta,U\theta-\theta)=0$ da cui quanto detto.
 Sia ora $I=\{\phi\in H | U\phi=\phi\}$, allora $(D\cup I)^\bot=\{0\}$, e quindi $D\cup I$ è denso in $H$. 
 Osserviamo che in $D\cup I$ vale la tesi, per cui concludiamo vedendo che la tesi vale anche in $\overline{D\cup I}$, ovvero che la tesi passa al limite. 
 Sia $\phi_k\rightarrow\phi$ di Cauchy in $D\cup I$, ovvero $h,k$-definitivamente $\|\phi_h-\phi_k\|<\epsilon$, allora anche $\|\hat\phi_h-\hat\phi_k\|<\epsilon$, 
 cioè anche $\hat\phi_k$ è di Cauchy, e quindi ha limite $\hat\phi$, e allora da
 \[\|\hat\phi-\frac{1}{n}\sum_{j=0}^{n-1}U^j\phi\|\leq \|\frac{1}{n}\sum_{j=0}^{n-1}U^j(\phi-\phi_k)\|+
  \|\frac{1}{n}\sum_{j=0}^{n-1}(U^j\phi_k-\hat\phi_k)\|+\|\hat\phi_k-\hat\phi\|\]
 si conclude che vale proprio $\|\hat\phi-\frac{1}{n}\sum_{j=0}^{n-1}U^j\phi\|\rightarrow 0$.
\end{proof}

\begin{defi}In un sistema dinamico misurabile $(X,\A,\mu,S)$, per $A\in\A$, $x\in X$ diciamo somma di Birkhoff ($n$-esima) \[T(x,A,n)=\sum_{j=0}^{n-1}\chi_A(S^jx)\]
inoltre saranno $\nu(x,A,n)=\frac{1}{n}T(x,A,n)$, $\bar\nu(x,A)=\limsup_{n\rightarrow\infty}\frac{1}{n}T(x,A,n)$ e rispettivamente $\underline\nu(x,A)$, e infine, se esiste, $\nu(x,A)=\lim_{n\rightarrow\infty}\frac{1}{n}T(x,A,n)$.
\end{defi}


\begin{teo}[di Birkhoff per frequenze di visita] Per $\mu$-quasi ogni $x\in X$ esiste $\nu(x,A)$.\end{teo}
\begin{proof}
 Fissiamo $\epsilon>0$ e definiamo
 \[\bar\tau(x,A,\epsilon)=\min\{n\in\N:\nu(x,A,n)\geq\bar\nu(x,A)-\epsilon\}\]
 e analogamente $\underline\tau$ per $\underline\nu$. Supponiamo dapprima che $\forall x\in X \quad \bar\tau(x,A,\epsilon)\leq M$ (con $M$ che dipenderà eventualmente da $\epsilon$). Fissiamo $n$ più grande di $M$ e definiamo ricorsivamente una sottosuccessione di $\{S^jx\}_0^{n-1}$
 \[\begin{cases}
 x_0=x \qquad \tau_0=\bar\tau(x_0,A,\epsilon)\\
 x_{k+1}=S^{\bar\tau(x_k,A,\epsilon)}x_k=S^{\tau_k}x_0 \qquad \tau_k=\sum^{k-1}\bar\tau(x_h,A,\epsilon)
 \end{cases} \]
 che quindi deve concludersi con $x_{K}$ tale che $\bar\tau_{K}<n$ e $\bar\tau_{K+1}\geq n$. Osserviamo che, poichè $\nu(x,A,\bar\tau(x,A,\epsilon))\geq\bar\nu(x,A)-\epsilon$ e $\bar\nu(x,A)=\bar\nu(Sx,A)$, si hanno le seguenti disuguaglianze
 \[\begin{split}
  T(x_k,A,\bar\tau(x_k,A,\epsilon))&=\bar\tau(x_k,A,\epsilon)\nu(x_k,A,\bar\tau(x_k,A,\epsilon)) \\
  &\geq \bar\tau(x_k,A,\epsilon) (\bar\nu(x_k,A)-\epsilon)=\bar\tau(x_k,A,\epsilon)(\bar\nu(x_0,A)-\epsilon)
 \end{split}\]
 
 \[\begin{split}
  T(x,A,n)&=\sum^{K-1}T(x_k,A,\bar\tau(x_k,A,\epsilon))+\underbrace{T(x_k,A,n-K)}_{\geq 0}   \\
	  &\geq \underbrace{\sum^{K-1} \bar\tau(x_k,A,\epsilon)}_{\bar\tau_K}(\bar\nu(x,A)-\epsilon)   \\
	  &\geq (n-M)(\bar\nu(x,A)-\epsilon)
 \end{split}\]
da cui, integrando, si ha
\[n \mu(A)=\int_X T(x,A,n) d\mu(x)\geq (n-M)\left(\int_X \bar\nu(x,A)d\mu(x)-\epsilon\right)\]
e dunque per arbitrarietà di $\epsilon$ abbiamo $\mu(A)\geq \int_X \bar\nu(x,A)d\mu(x)$. Analogamente supponendo $\underline\tau$ limitato, si ottiene l'altra disuguaglianza:
\[\int_X \underline\nu(x,A)d\mu\geq\mu(A)\geq \int_X \bar\nu(x,A)d\mu\]
per cui $\underline\nu(x,A)\geq \bar\nu(x,A)$ per $\mu$-quasi ogni $x\in X$, e quindi la tesi.

Rimuoviamo infine l'ipotesi $\bar\tau$ limitato: fissato $\epsilon>0$, scegliamo $C$ tale che
\[\mu(A_C)=\mu\{x\in X: \bar\tau(x,A,\epsilon)>C\}<\epsilon\]
poniamo quindi $\bar A=A\cup A_C$ e
\[\bar\tau'(x,A,\epsilon)=\begin{cases}\bar\tau(x,A,\epsilon) \qquad x\notin A_C\\
					1 \qquad x\in A_C
  \end{cases}\]
e definiamo analogamente a prima la successione $x_{k+1}'=S^{\bar\tau'(x_k,A,\epsilon)}x_k'$. Abbiamo allora che
\[T(x_k,\bar A,\bar\tau'(x_k,A,\epsilon))
    \begin{cases} =T(x_k',\bar A,1)=1\geq \bar\nu(x_0,A)-\epsilon \qquad x\in A_C \\
		  \begin{aligned}\geq T(x_k,A,\bar\tau'(x_k,A,\epsilon))&=T(x_k,A,\bar\tau(x_k,A,\epsilon))\\
			&\geq \bar\tau(x_k,A,\epsilon)(\bar\nu(x_0,A)-\epsilon) \qquad x\in A \end{aligned}
    \end{cases}\]
\[T(x_k,\bar A,\bar\tau'(x_k,A,\epsilon))\geq \bar\tau'(x_k,A,\epsilon)(\bar\nu(x_0,A)-\epsilon)\]
per cui concludiamo esattamente come sopra sommando sulla successione e integrando, ottenendo
\[\mu(A)+\epsilon\geq\mu(\bar A)\geq \int_X \bar\nu(x,A)-\epsilon.\]
\end{proof}

%%%%%%%%%%%%%%%%%%%%%%%%%%%%%%%%%%%%%%%%%%%%%%%%%%%%%%%%%%%%%%%%%%%%%%%%%%%%%%%%%%%%%%
\subsection{Unica Ergodicità}

\begin{teo}Sia $(X,\mathcal{A},\mu,S)$ ergodico e $\mu_1$ un'altra misura di probabilità $S$-invariante.
Sono equivalenti:
\begin{lista}
\item $\mu\neq\mu_1$
\item $\mu_1$ non è assolutamente continua rispetto a $\mu$
\item esiste un $A\in\mathcal{A}$ $S$-invariante (cioè $S(A)\subseteq A$) con $\mu(A)=0$ e $\mu_1(A)\neq 0$.
\end{lista}
\end{teo}

\begin{proof}Ovviamente $(3)\implica (2)$ e $(2)\implica (1)$, resta da vedere $(1)\implica (3)$.
Se $\mu\neq\mu_1$ esistono $A$ misurabili con $\mu_1(A)>\mu(A)$. Consideriamo 
\[M:=\sup\set{\mu_1(A)-\mu(A)\mid A\in\mathcal{A}}>0,\]
questo $\sup$ è in realtà un $\max$ e viene realizzato dal supporto $P$ di $(\mu_1-\mu)^+$ (la parte positiva della decomposizione di Hahn di $\mu_1-\mu$). 
Se $(\mu_1-\mu)(A)=M$ allora $A$ coincide con $P$ a meno di insiemi $\abs{\mu_1-\mu}$-trascurabili, scriveremo $A\sim P$.
L'invarianza di $\mu$ e $\mu_1$ dà $(\mu_1-\mu)\pa{S^{-1}(P)}=M$, quindi $S^{-1}(P)\sim P$; analogamente $S^{-n}(P)\sim P$,
quindi $P':=\bigcup_{n\ge 0}S^{-n}(P)\sim P$. L'insieme $P'$ è $S$-invariante e $1\ge \mu_1(P')>\mu(P')$, perciò per ergodicità di $\mu$, $\mu(P')=0$.
\end{proof}

\begin{oss}$(2)\implica (3)$ si può anche mostrare così: per ipotesi c'è un $N\in\mathcal{A}$ tale che $\mu(N)=0$ e $\mu_1(N)\neq 0$.
Poniamo $A:=\limsup_{k\to\infty}f^{-k}(N)=\bigcap_{m\ge 0}\bigcup_{n\ge m}f^{-n}(N)$ (l'insieme dei punti che
cascano frequentemente in $A$). $A$ è $S$-invariante e $\mu(A)=0$, $\mu_1(A)\neq 0$ grazie all'invarianza di $\mu$ e $\mu_1$.
\end{oss}

\begin{oss}Se $\mu_1\neq\mu$ è anch'essa ergodica, si ha $\mu_1(P')=1$, quindi $\mu\perp\mu_1$.
\end{oss}

\begin{defi}$(X,\mathcal{A},\mu,S)$ è \emph{unicamente ergodico} se $\mu$ è l'unica misura di probabilità su $\A$
$S$-invariante.
\end{defi}

\begin{teo}Sia $X$ uno spazio metrico compatto, $\A=\mathcal{B}(X)$ la $\sigma$-algebra dei boreliani
e $S:X\to X$ continua. Se $\mu$ è unicamente ergodica, $\mu$ è ergodica e per ogni $f\in C(X,\R)$
le somme di Birkhoff $\frac{1}{n}\sum_{j=0}^{n-1}f\circ S^j$ convergono uniformemente a $\widehat{f}:=\int_X f\,d\mu$.
\end{teo}

La seconda parte della dimostrazione assomiglia molto a quella di Krylov-Bogoliubov,
perciò evitiamo di ripetere tutti i dettagli nell'utilizzare la compattezza delle misure.

\begin{proof}Mostriamo che $\mu$ è ergodica: se $A$ è un boreliano invariante (cioè $S(A)\subseteq A$)
e $0<\mu(A)<1$, allora $\pa{\frac{\uno_A}{\mu(A)}}\mu$ è una misura invariante, di probabilità e diversa da $\mu$
(invarianza: $A\subseteq S^{-1}(A)$ e $\mu(A)=\mu\pa{S^{-1}(A)}$, quindi $\mu\pa{S^{-1}(A)\setminus A}=0$;
perciò, per ogni $B$, $(\uno_A\mu)\pa{S^{-1}(B)}=\mu\pa{S^{-1}(B)\cap A}=\mu\pa{S^{-1}(B)\cap S^{-1}(A)}=\mu(B\cap A)=(\uno_A\mu)(B)$)
e questo contraddice l'ipotesi. \\
Mostriamo per assurdo anche la seconda parte, cioè supponiamo che esistano $f:X\to\R$ continua e $\epsilon>0$ tali che
$\norm{\frac{1}{n_i}\sum_{j=0}^{n_i-1}f\circ S^j-\int_X f\,d\mu}_\infty\ge\epsilon$ per qualche sottosuccessione $n_i\uparrow\infty$.
Esiste allora una successione di punti $(x_i)$ tale che
\[ \abs{\frac{1}{n_i}\sum_{j=0}^{n_i-1}f\pa{S^j(x_i)}-\int_X f\,d\mu}\ge\epsilon. \]
Poniamo $\nu_i:=\frac{1}{n_i}\sum_{j=0}^{n_i-1}\delta_{S^j(x_i)}$: $\nu_i$ è una misura di probabilità che soddisfa (per costruzione)
$\abs{\int_X f\,d\nu_i-\int_X f\,d\mu}\ge\epsilon$. \\
Per la compattezza debole* delle misure (viste come il duale di $C(X)$), a meno di un'ulteriore sottosuccessione
abbiamo $\nu_i\overset{*}{\rightharpoonup}\nu$ per qualche misura di probabilità $\nu$. \\
Ora $\abs{\int_X f\,d\nu-\int_X f\,d\mu}=\lim_{i\to\infty}\abs{\int_X f\,d\nu_i-\int_X f\,d\mu}\ge\epsilon$, da cui $\mu\neq\nu$.
Per arrivare all'assurdo basta mostrare che $\nu$ è invariante, che equivale a $\int_X g\circ S\,d\nu=\int_X g\,d\nu$. Ma
\[ \int_X (g\circ S-g)\,d\nu=\lim_{i\to\infty}\frac{1}{n_i}\sum_{j=0}^{n_i-1}\pa{g\pa{S^{j+1}(x_i)}-g\pa{S^j(x_i)}}
=\lim_{i\to\infty}\frac{1}{n_i}\pa{g\pa{S^{n_i}(x_i)}-g(x_i)}=0 \]
e abbiamo finito.
\end{proof}

\begin{oss}[facoltativa] Vale anche il viceversa del teorema appena visto: se per ogni $f\in C(X,\R)$ le somme $\frac{1}{n}\sum_{j=0}^{n-1}f\circ S^j$ 
convergono puntualmente a $\int_X f\,d\mu$, allora $S$ è unicamente ergodica. Infatti se $\nu$ è un'altra misura invariante, per convergenza dominata si ha
\[\int_X f\,d\nu=\int_X \frac{1}{n}\sum_{j=0}^{n-1}f\circ S^j d\nu \xrightarrow{n\rightarrow\infty}\int_X \left(\int_X f d\mu\right)d\nu=\int_X f d\mu\]
per ogni $f\in C(X,\R)$, da cui $\mu$ e $\nu$ coincidono.
\end{oss}



\begin{esercizio}Consideriamo la cifra più significativa di $2^k$, per $k=1,2,\dots$.
Nella successione che si ottiene compare più spesso il $7$ o l'$8$?
\end{esercizio}

\begin{soluz}Sia $s$ tale che $10^s\le 2^k<10^{s+1}$. $c\in\set{1,\dots,9}$ è la cifra più significativa di $2^k$
quando $c\cdot 10^s\le 2^k<(c+1)10^s$, cioè quando $s+\log_{10}(c)\le k\log_{10}(2)<s+\log_{10}(c+1)$.
Inoltre $s=\floor{k\log_{10}(2)}$, quindi ci stiamo chiedendo se è
$\set{k\log_{10}(2)}\in\pa{\log_{10}(7),\log_{10}(8)}$ oppure $\set{k\log_{10}(2)}\in\pa{\log_{10}(8),\log_{10}(9)}$
(usiamo gli intervalli aperti perché per $k>3$ gli estremi degli intervalli non verranno mai realizzati). \\
Sia $\alpha:=\log_{10}(2)$ (irrazionale) e $R_\alpha(x):=x+\alpha$ la solita rotazione
su $\mathbb{T}^1=\R/\Z$; ci stiamo chiedendo chi è maggiore tra $\nu(0,A_1)$ e $\nu(0,A_2)$, dove $A_1$
e $A_2$ sono gli intervalli scritti sopra, proiettati su $\mathbb{T}^1$. \\
Birkhoff non garantisce neppure che queste frequenze asintotiche esistano (il singoletto $\set{0}$ è trascurabile!),
ma possiamo applicare invece l'ultimo teorema. \\
Intanto occorre verificare che la misura di Lebesgue è l'unica misura $R_\alpha$-invariante:
se $\mu$ è una misura di probabilità tale che $\pa{R_\alpha}_*\mu=\mu$ otteniamo
\[ \widehat{\mu}(k)=\widehat{\pa{R_\alpha}_*\mu}(k)=e^{-2\pi i k\alpha}\widehat{\mu}(k), \]
da cui $\widehat{\mu}(k)=0$ se $k\neq 0$, mentre $\widehat{\mu}(0)=\mu(\mathbb{T}^1)=1$
(ricordiamo che $\widehat{\mu}(k):=\int_0^1 e^{-2\pi i kx}\,d\mu(x)$
e che vale l'unicità: se due misure hanno gli stessi coefficienti di Fourier, allora coincidono). \\
Quindi i coefficienti di Fourier di $\mu$ sono univocamente determinati, ovvero c'è una sola misura invariante,
che deve essere quella di Lebesgue. \\
Siamo nelle ipotesi dell'ultimo teorema; le funzioni $\uno_{A_1}$ e $\uno_{A_2}$ non sono continue, ma possiamo cavarcela
approssimando dal basso. Siano per esempio $\phi_k\uparrow \uno_{A_1}$ con $0\le\phi_k\le 1$ e $\phi_k\in C(\mathbb{T}^1)$.
Il teorema ci dice che
\[ \int_{\mathbb{T}^1} \phi_k\,dx=\ubar{\lim}_{T\to\infty}\frac{(S_T\phi_k)(0)}{T}
\le\ubar{\lim}_{T\to\infty}\frac{(S_T\uno_{A_1})(0)}{T}=\ubar{\nu}(0,A_1) \]
e mandando $k\to\infty$ otteniamo $\abs{A_1}\le\ubar{\nu}(0,A_1)\le\obar{\nu}(0,A_1)$.
Chiamando $J_1$ la parte interna di $\mathbb{T}^1\setminus A_1$,
analogamente vale $\abs{J_1}\le\ubar{\nu}(0,J_1)$. Sommando le due disuguaglianze arriviamo a
\[ 1=\abs{A_1}+\abs{J_1}\le\obar{\nu}(0,A_1)+\ubar{\nu}(0,J_1)\le \obar{\lim}_{T\to\infty}\nu(0,T,A_1\cup J_1)\le 1, \]
per cui devono valere tutte le uguaglianze: in particolare $\ubar{\nu}(0,A_1)=\obar{\nu}(0,A_1)=\abs{A_1}$. \\
Allo stesso modo $\nu(0,A_2)=\abs{A_2}$. Ma $\abs{A_1}<\abs{A_2}$, dunque il $7$ compare più spesso dell'$8$.
\end{soluz}

%%%%%%%%%%%%%%%%%%%%%%%%%%%%%%%%%%%%%%%%%%%%%%%%%%%%%%%%%%%%%%%%%%%%%%%%%%%%%%%%%%%%%%%%
\subsection{Sistemi Mescolanti}

Ritorniamo alla nozione di sistema \emph{strongly mixing} definita più sopra, più forte dell'ergodicità.

\begin{defi}$(X,\mathcal{A},\mu,f)$ è \emph{mescolante} se
per ogni $A,B\in\mathcal{A}$ abbiamo \[\mu\pa{f^{-n}(A)\cap B}\xrightarrow{n\to\infty}\mu(A)\mu(B).\]
\end{defi}

\begin{esempio}La rotazione irrazionale $R_\alpha\in\homeo(\mathbb{T}^1)$ non è mescolante.
Infatti, preso un piccolo intervallo $I$ centrato in $x_0$, esistono infiniti $n>0$ tali che $f^{-n}(x_0)\approx x_0+\mz$,
quindi per questi $n$ abbiamo $f^{-n}(I)\cap I=\emptyset$, contraddicendo la definizione di mescolanza con $A,B:=I$.
\end{esempio}

\begin{prop}$(X,\mathcal{A},\mu,f)$ è mescolante se e solo se
\[\forall\phi,\psi\in L^2(X) \qquad  \ang{U_f^n\phi,\psi}\to\ang{\phi,1}\ang{1,\psi}\]
dove $U_f:L^2(X)\to L^2(X)$, $U_f(\phi)=\phi\circ f$ è l'operatore di Koopman.
\end{prop}

\begin{oss} Scritto in questo modo, l'enunciato funziona anche per funzioni a valori complessi. 
Inoltre, tornando a una vecchia osservazione sul decadimento delle correlazioni, il teorema afferma che va a $0$
\[C_n(\phi,\psi):=\ang{U_f^n\phi,\psi}-\ang{\phi,1}\ang{1,\psi},\]
l'$n$-\emph{esimo coefficiente di correlazione}.\end{oss}

\begin{proof}$(\Leftarrow)$: ovvio restringendosi alle funzioni caratteristiche e ricordando che $U_f(\uno_A)=\uno_{f^{-1}(A)}$. \\
$(\Rightarrow)$: la proprietà vale per funzioni caratteristiche e per bilinearità anche per funzioni semplici.
Per $\phi,\psi\in L^2(X)$ generiche, fissato $\epsilon>0$ esistono $\alpha$ e $\beta$ semplici tali che
$\norm{\phi-\alpha}_2,\norm{\psi-\beta}_2<\epsilon$. Sappiamo che definitivamente
\[ \abs{\ang{U_f^n\alpha,\beta}-\ang{\alpha,1}-\ang{1,\beta}}<\epsilon \]
e scrivendo
\[ \ang{U_f^n\phi,\psi}-\ang{U_f^n\alpha,\beta}
=\ang{U_f^n\phi,\psi}-\ang{U_f^n\phi,\beta}+\ang{U_f^n\phi,\beta}-\ang{U_f^n\alpha,\beta} \]
e usando il fatto che $U_f$ è un'isometria si trova che $\abs{\ang{U_f^n\phi,\psi}-\ang{U_f^n\alpha,\beta}}<O(\epsilon)$.
Le differenze $\abs{\ang{\phi,1}-\ang{\alpha,1}}$ e $\abs{\ang{1,\phi}-\ang{1,\alpha}}$ si stimano entrambe con $\epsilon$. \\
Quindi definitivamente $\abs{\ang{U_f^n\phi,\psi}-\ang{\phi,1}-\ang{1,\psi}}<O(\epsilon)$.
\end{proof}

\begin{esempio}Gli automorfismi lineari \emph{iperbolici} di $\mathbb{T}^2=\R^2/\Z^2$, cioè
gli elementi di $GL(2,\Z)$ con autovalori di modulo $\neq 1$, sono tutti mescolanti. \\
Sia infatti $A\in GL(2,\Z)$ una matrice iperbolica. Essendo $\abs{\det(A)}=1$, $A$ preserva la misura di Lebesgue su $\mathbb{T}^2$.
Basta ora vedere che vale l'ipotesi dell'ultima proposizione per funzioni
$\phi,\psi\in\set{e_k(x):=e^{2\pi i(k_1 x_1+k_2 x_2)}\mid k\in\Z^2}$, perché queste generano un sottospazio denso in $L^2(\mathbb{T}^2)$
e si conclude come nella dimostrazione della proposizione. \\
Dobbiamo verificare che $\ang{U_A^n e_k,e_{k'}}\to\ang{e_k,1}\ang{1,e_{k'}}$; per $k=0$ è ovvio, quindi supponiamo $k\in\Z^2\nonzero$.
Osserviamo che $U_A^n e_k=e_{(A^t)^n k}$ e che $(A^t)^n k=k'$ per al più un valore di $n$,
altrimenti avremmo $(A^t)^n k=(A^t)^{n'} k$ (con $n<n'$) $\implica$ $(A^t)^{n'-n}-I$ avrebbe nucleo non banale
$\implica$ $A^t$ avrebbe un autovalore di modulo $1$, assurdo.
Perciò definitivamente $(A^t)^n k\neq k'$ $\implica$ definitivamente $\ang{U_A^n e_k,e_{k'}}=0=\ang{e_k,1}\ang{1,e_{k'}}$.
\end{esempio}