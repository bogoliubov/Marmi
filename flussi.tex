\section{Flussi e Equazione Coomologica}

Consideriamo il toro $d$-dimensionale $\T^d=\R^d/\Z^d$ per $d\geq 2$, $\alpha\in\R^d\smallsetminus \{0\}$. In $\T^d$ abbiamo la dinamica continua data dal flusso integrale 
\[S^t_\alpha x=x+t\alpha \qquad \mbox{dell'equazione} \qquad \dot x=\alpha\]
in $\R^d$ al quoziente sul toro, e la dinamica discreta ad essa strettamente collegata data dalla traslazione 
$S_\alpha: x\mapsto x+\alpha$ in $\R^d$ al quoziente sul toro (in entrambi i casi parleremo di traslazioni sul toro). 
Osserviamo che la misura di Haar è invariante per entrambe, e se non meglio specificato ci riferiremo a quella.

\begin{defi} $\alpha=(\alpha_1,\dots,\alpha_d)$ è \emph{non risonante} se i reali $1,\alpha_1,\dots,\alpha_d$ sono linearmente indipendenti su $\Q$, ovvero
 \[\forall k\in \Z^{d+1} \qquad (1,\alpha)\cdot k=0 \Rightarrow k=0,\]
 diremo altrimenti che $\alpha$ è \emph{risonante}, e in tal caso sarà non nullo $R=\{k\in\Z^d:(1,\alpha)\cdot k=0\}$
\end{defi}

\begin{teo} Le dinamiche $S^t_\alpha$ e $S_\alpha$ sono minimali se e solo se $\alpha$ è non risonante. 
Inoltre sono ergodiche se e solo se $\alpha$ è non risonante, e in tal caso sono anche unicamente ergodiche.\end{teo}

\begin{proof}[Dimostrazione (facoltativa).]
 Proveremo il teorema solo per tempo discreto: la controparte a tempo continuo è analoga o ne discende immediatamente. 
 Anzitutto mostriamo con un unico esempio che se $\alpha$ è risonante la dinamica non è transitiva nè ergodica: 
 infatti in tal caso esiste $(k_1,\dots k_d)\in\Z^d$ tale che $\sum\alpha_j k_j=k_0\in\Z$, e allora abbiamo l'osservabile continuo integrabile non costante
 \[\phi:\T^d\rightarrow \T^d \qquad \phi(x)=\sin 2\pi(k_1x_1+\dots+k_dx_d).\]
 Supponiamo $S_\alpha$ non minimale, ovvero (per un esercizio nella parte di Dinamica Topologica) non transitivo, 
 allora abbiamo due aperti disgiunti invarianti $U,V\subset \T^d$. L'indicatrice $f=\uno_U$ allora non è q.c. costante, ma allora i suoi coefficienti di Fourier
 non sono tutti nulli, e quindi da
 \[f(x)=\sum_{m\in \Z^d} f_m e^{2\pi i (m\cdot x)}\]
 \[f(S_\alpha x )=\sum_{m\in \Z^d} f_m e^{2\pi i (m\cdot x)} e^{2\pi i (m\cdot \alpha)}\]
 (intendendo $x\in \T^d$) imponendo l'invarianza per $S_\alpha$ si ha che esiste $m\in\Z^d$ tale che $e^{2\pi i (m\cdot \alpha)}=1$, da cui $\alpha$ risonante.
 Se $\alpha$ è non risonante, è facile mostrare, sviluppando ancora in serie di Fourier, che le funzioni integrabili invarianti sono quasi certamente costanti,
 e quindi che il sistema è ergodico.
\end{proof}

























Lo stesso risultato vale se riparametrizziamo il tempo, ovvero se consideriamo invece il flusso dato da $\dot x=\phi(x)\alpha$ con $\phi\in C^1(\T^d,\R^+)$, che ha le stesse orbite di $S^t$. La misura invariante è però ora $\phi(x)dx$, con $dx$ la misura di Haar.

Sia ora $\beta=(\alpha,1)\in \R^{d+1}$ non risonante e $\psi\in C^1(\T^d,\R^+)$: consideriamo il flusso in $\T^{d+1}$ dato da 
\[(x,s)\mapsto(x,s+t) \qquad \mbox{in }\T^d\times \R\]
al quoziente su $(x,s+\psi(x))~(S_\alpha,s)$. Questo flusso $S^t_{\alpha,\psi}$ è detto \emph{flusso speciale} su $\T^{d+1}$, e conserva la misura $\frac{dxds}{\int_{\T^d}\psi(x)dx}$ ($dx$ su $\T^d$ e $ds$ su $\T^1$).

\begin{teo}Il flusso speciale $S^t_{\alpha,\psi}$ è $C^r$-coniugato a $S^t_\beta$, ovvero
 \[\exists h\in C^r(\T^{d+1},\T^{d+1}):\quad \forall t \quad S^t_{\alpha,\psi}h=hS^t_\beta\]
 se e solo se esiste una soluzione $\chi\in C^r(\T^d,\R^+)$ dell'\emph{equazione coomologica}
 \[\psi(x)-\int_{\T^d}\psi(x)dx=\chi(x)-\chi(x+\alpha)\]
 (soluzione che sarà unica se $\chi$ ha media nulla).
\end{teo}

Studieremo per $d=1$ il legame tra l'equazione coomologica e le proprietà di $\alpha$. Premettiamo un lemma sulle serie di Fourier.

\begin{teo}Sia $\chi\in L^2\T^1$, $\hat\chi_n=\int_0^1\chi(x)e^{-2\pi inx}dx$ i suoi coefficienti di Fourier. Allora
\begin{lista}
 \item $\chi\in C^\infty$ se e solo se i suoi coefficienti sono \emph{a decrescenza rapida}, ovvero
      \[\forall N\in \N\quad \exists C_N>0: \quad \forall n \quad \left|\hat\chi_n\right|\leq\frac{C_N}{(1+|n|)^N}\]
 \item $\chi$ è analitica, ovvero esiste un intorno di $T^1=\R/\Z$ in $\C/\Z$ su cui $\chi$ si estende a una funzione olomorfa, se e solo se esistono $M,\delta>0$ tali che $\forall n \quad \left|\hat\chi_n\right|\leq Me^{-\delta |n|}$.
\end{lista}
\end{teo}
\begin{proof}
 Per il primo punto, se $\chi\in C^\infty$ integrando per parti si ottiene
 \[\hat\chi_n=\frac{1}{2\pi in}\int_0^1\chi'(x)e^{-2\pi inx}dx \quad \Rightarrow \quad \left|\hat\chi_n\right|\leq \frac{1}{2\pi |n|}\left|\widehat{\chi_n'}\right|\]
 da cui la tesi iterando; il viceversa è immediato perchè le stime sui $\hat\chi_n$ danno convergenza uniforme per le derivate.
 Il secondo punto. Se $\chi$ è analitica, sia $\chi(z)$ olomorfa sul rettangolo di vertici $0, 1,1+i\delta,i\delta$: orientiamo il bordo $\Gamma$ in senso antiorario e applichiamo il teorema di Cauchy ottenendo
 \[\int_\Gamma \chi(z)e^{-2\pi inz}dz=0 \quad \Rightarrow \quad \hat\chi_n=\int_0^1\chi(x)e^{-2\pi inx}dx=\int_0^1\chi(x+i\delta)e^{-2\pi in(x+i\delta)}dx\]
 ricordando che in $\C/\Z$ i bordi verticali del rettangolo coincidono, da cui il controllo cercato. Per gli $n$ pari invece si prende il rettangolo \emph{sotto} il segmento $[0,1]$, orientandolo in senso orario. Per il viceversa basta osservare che dalle stime si ha che $\chi$ è somma convergente di funzioni olomorfe. 
\end{proof}

Usiamo questa caratterizzazione per dare due criteri aritmetici su $\alpha$ che garantiscono
che l'equazione coomologica $\psi-\psi\circ R_\alpha$ ha soluzione,
nelle categorie $C^\infty(\mathbb{T}^1)$ e $C^\omega(\mathbb{T}^1)$.

\begin{teo}Per $\alpha$ irrazionale valgono i seguenti criteri.
\begin{lista}
\item $\alpha$ diofanteo $\sse$ per ogni $\psi\in\C^\infty(\mathbb{T}^1)$ a media nulla
l'equazione coomologica ha una (unica) soluzione $\chi\in C^\infty(\mathbb{T}^1)$.
\item $\alpha$ soddisfa $\lim_{n\to\infty}\frac{\ln q_{n+1}}{q_n}<\infty$ (dove $\frac{p_n}{q_n}$ sono i suoi convergenti)
$\sse$ per ogni $\psi\in C^\omega(\mathbb{T}^1)$ a media nulla
l'equazione coomologica ha una (unica) soluzione $\chi\in C^\omega(\mathbb{T}^1)$ a media nulla.
\end{lista}
\end{teo}

\begin{proof}Vediamo il punto $(1)$. $(\Rightarrow)$: trasformando l'equazione coomologica troviamo
$\pa{1-e^{2\pi i k\alpha}}\widehat{\chi}(k)=\widehat{\psi}(k)$. Inoltre per ipotesi
$\norm{k\alpha}\ge\gamma k^{-(1+\tau)}$, quindi
\[ \abs{\widehat{\chi}(k)}=\frac{\abs{\widehat{\psi}(k)}}{\abs{1-e^{2\pi i k\alpha}}}=\abs{\widehat{\psi}(k)}O(k^{1+\tau}) \]
e la decrescenza rapida di $\widehat{\psi}(k)$ implica quella di $\widehat{\chi}(k)$;
dunque la funzione $\chi$ avente questi coefficienti di Fourier (e $\widehat{\chi}(0)=0$) è $C^\infty$ e risolve l'equazione. \\
$(\Leftarrow)$: supponiamo per assurdo $\alpha$ non diofanteo, ovvero esista una successione $k_j\uparrow\infty$
per cui $\abs{1-e^{2\pi i k_j\alpha}}^{-1}\ge k_j^j$. Costruiamo una $\psi$ con una serie di Fourier ``lacunare'': sia
\[ \psi(x):=\sum_j \pa{1-e^{2\pi i k_j\alpha}}e^{2\pi i k_j x}+\sum_j \pa{1-e^{-2\pi i k_j\alpha}}e^{-2\pi i k_j x}. \]
$\psi$ ha valori reali ed è $C^\infty$ grazie alla decrescenza rapida dei coefficienti.
Però non può esistere una $\chi\in C^\infty$ che risolve l'equazione coomologica perché come visto prima
dovrebbe avere tutti i coefficienti di indice $k_j$ pari a $1$. \\
Passiamo al punto $(2)$ (facoltativo). $(\Rightarrow)$: fissato $n>0$ (il caso $n<0$ è analogo), per la proprietà di migliore approssimazione ??
abbiamo $\norm{n\alpha}\ge\norm{q_k\alpha}$ dove $q_k\le n<q_{k+1}$. Inoltre
\[ \norm{q_k\alpha}=\frac{1}{q_{k+1}+q_k\alpha_{k+1}}\sim\frac{1}{q_{k+1}} \]
($0<\alpha_{k+1}<1$ si ottiene applicando $k+1$ volte la mappa di Gauss a $\set{\alpha}$).
L'ipotesi dice che $\abs{\widehat{\psi}(n)}\le Ce^{-\delta n}$, quindi come nel punto precedente otteniamo
\[ \abs{\widehat{\chi}(n)}=\frac{\abs{\widehat{\psi}(n)}}{\abs{1-e^{2\pi i n\alpha}}}\lesssim Ce^{-\delta n}\norm{n\alpha}^{-1}
\lesssim Ce^{-\delta n}q_{k+1} \]
e per concludere basta ad esempio $q_{k+1}\le e^{\frac{\delta}{2}n}$ definitivamente.
Ma per ipotesi $\ln q_{k+1}\le \frac{\delta}{2}q_k\le\frac{\delta}{2}n$ definitivamente. \\
$(\Leftarrow)$: per assurdo esista $\delta>0$ e una successione $k_j\uparrow\infty$ tale che
$\ln q_{k_j+1}\ge\delta q_{k_j}$, cioè $q_{k_j+1}\ge e^{\delta q_{k_j}}$. Come prima ricaviamo
\[ \abs{1-e^{2\pi i q_{k_j}\alpha}}\sim\frac{1}{q_{k_j+1}}\le e^{-\delta q_{k_j}}, \]
quindi definendo
\[ \psi(x):=\sum_j \pa{1-e^{2\pi i k_j\alpha}}e^{2\pi i k_j x}+\sum_j \pa{1-e^{-2\pi i k_j\alpha}}e^{-2\pi i k_j x} \]
otteniamo una funzione analitica a valori reali. Una soluzione $\chi$ dell'equazione coomologica però avrebbe
$\widehat{\chi}(q_{k_j})=1$ per ogni $j$, assurdo.
\end{proof}

L'equazione coomologica $\psi-\psi\circ f=\phi$ si può risolvere in una situazione molto più generale.
Consideriamo un sistema dinamico topologico $(X,d,f)$.

\begin{defi}Dato $(X,d)$ spazio metrico compatto, $f\in\homeo(X)$ è \emph{minimale} se tutte le orbite sono dense.
\end{defi}

\begin{oss}Ovviamente la minimalità implica la transitività.
\end{oss}

Poniamo $d\psi:=\psi-\psi\circ f$. Vedendo $d$ come un operatore lineare continuo,
$d:C(X)\to C(X)$, osserviamo che $\ker d=\set{\text{integrali primi}}$ e l'ortogonale all'immagine è $(dC(X))^\perp
=\mathcal{M}_f(X)$, lo spazio delle misure reali $f$-invarianti.

Cerchiamo di risolvere l'equazione coomologica almeno formalmente:
$\psi=\phi+\psi\circ f$ e sostituendo $\psi$ a secondo membro otteniamo
\[ \psi=\phi+\phi\circ f+\psi\circ f^2 \]
e iterando
\[ \psi=\phi+\phi\circ f+\dots+\phi\circ f^{n-1}+\psi\circ f^n, \]
cioè formalmente $\psi=\sum_{j=0}^\infty \phi\circ f^j$. Torniamo a fare discorsi rigorosi. \\
Supponendo che una soluzione $\psi$ esista, come abbiamo visto deve valere
$\sum_{j=0}^{n-1}\phi\circ f^j=\psi-\psi\circ f^n$. Quindi una condizione necessaria è che le somme parziali
di Birkhoff siano equilimitate. Se $f$ è minimale vale anche il viceversa, anzi vale addirittura:
\begin{teo}[Gottshalk-Hedlund]Se $f\in\homeo(X)$ è minimale, $\phi\in C(X)$ e per qualche $x_0\in X$, $C>0$
è $\abs{\sum_{j=0}^{n-1}\pa{\phi\circ f^j}(x_0)}\le C$ per tutti gli $n\ge 0$, allora l'equazione coomologica
$\psi-\psi\circ f=\phi$ ha soluzione.
\end{teo}

Cioè basta l'equilimitatezza in un solo punto $x_0$! Cambiando segno a $\psi$, scriviamo l'equazione coomologica
in questa forma: $\psi\circ f-\psi=\phi$.

\begin{proof}Lavoriamo sullo \emph{skew-product} $X\times\R$: definiamo $F\in\homeo(X\times\R)$
con $F(x,u):=\pa{f(x),u+\phi(x)}$. $(X,\phi)$ è un fattore di $(X\times\R,F)$.
Nelle iterate di $F$ compaiono proprio le somme di Birkhoff:
\[ F^n(x,u)=\pa{f^n(x),u+\sum_{j=0}^{n-1}(\phi\circ f^j)(x)}. \]
L'ipotesi dice che la chiusura dell'orbita positiva di $(x_0,0)$, in simboli
$M:=\obar{\set{F^n(x_0,0)\mid n\ge 0}}$, è un insieme compatto e $F$-invariante (nel senso che $F(M)\subseteq M$). \\
Per il lemma di Zorn esiste un $K\subseteq M$ compatto, non vuoto e $F$-invariante che è \emph{minimale} tra gli insiemi
con queste proprietà. La minimalità di $K$ ha diverse conseguenze:
\begin{itemize}
	\item $F(K)=K$, altrimenti $F(K)$ sarebbe compatto invariante e più piccolo di $K$;
	\item $\pi_X(K)=X$: infatti da $\pi_X\circ F=f\circ\pi_X$ segue che $\pi_X(K)$ è $f$-invariante (e compatto),
	quindi è tutto $X$ per la minimalità di $f$;
	\item $T_t(x,u):=(x,u+t)$ (il flusso verticale) commuta con $F$ per ogni $t$ fissato, dunque
	$T_t(K)$ è $F$-invariante. Se $t\neq 0$ abbiamo $T_t(K)\nsupseteq K$ (perché $\pi_\R\pa{T_t(K)}=\pi_\R(K)+t$),
	quindi $T_t(K)\cap K\subsetneq K$, da cui (per minimalità di $K$) $T_t(K)\cap K=\emptyset$.
\end{itemize}
In questo modo abbiamo dimostrato che $K$ è il grafico di una funzione $\psi:X\to\R$. La compattezza di $K$
implica la continuità di $\psi$: se per assurdo $x_n\to x$ e $\abs{\psi(x_n)-\psi(x)}\ge\epsilon$,
a meno di sottosuccessioni $\pa{x_n,\psi(x_n)}\to (x,u)\in K$ con $u\neq\psi(x)$, per cui
$(x,u),(x,\psi(x))\in K$, contraddicendo l'ultimo punto. \\
Infine l'$F$-invarianza di $K$ dice che $F(x,\psi(x))=\pa{f(x),\psi(x)+\phi(x)}\in K$, cioè (per definizione di $\psi$)
$\psi\pa{f(x)}=\psi(x)+\phi(x)$.
\end{proof}

