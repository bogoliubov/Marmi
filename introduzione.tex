\section{Sistemi Dinamici}

\begin{defi}[Sistema dinamico secondo Smale]
 Un \emph{sistema dinamico} è l'azione di un gruppo o semigruppo $G$ su uno spazio $X$. 
 Con spazio intendiamo un insieme $X$ con una struttura di spazio, ad esempio topologico, vettoriale, di misura 
 \emph{et cetera}, con le relative trasformazioni che ne conservano la struttura $\ndom X$ e $\aut X$, 
 queste ultime con inversa a sua volta in $\ndom X$. L'azione è dunque un omomorfismo di gruppo (risp. semigruppo) $G\rightarrow \aut X$ (risp. $G\rightarrow \ndom X$).
\end{defi}

La definizione di sistema dinamico è molto ampia. 
Il gruppo $G$, normalmente inteso come insieme dei tempi, è di solito $\R$, e in tal caso si parla di sistemi dinamici a tempo continuo, 
oppure $\Z$ o $\N$ per il tempo discreto. Nulla impedisce però di considerare altri (semi)gruppi. Ci occuperemo quasi esclusivamente di sistemi dinamici a tempo discreto, 
per cui indicheremo con $(X,f)$ il sistema, con $f$ l'immagine del generatore di $\Z$ o $\N$. 
Dove non specificato i fatti enunciati vanno intesi nel modo appropriato per entrambe le situazioni: 
ad esempio, quando scriviamo $f^n$, $n$ sarà in $\N$ oppure $\Z$ e nel secondo caso bisognerà considerare anche le iterate di $f^{-1}$. 
Esempi di sistemi a tempo discreto sono le successioni numeriche, come quelle di Collatz, Thue-Morse, Kolakowski (che sono legate a diversi problemi aperti).
Sistemi a tempo continuo sono invece ad esempio i flussi integrali di campi vettoriali
su varietà.
Una classe importante di sistemi dinamici è quella dei \emph{biliardi}, che descrivono traiettorie di punti che rimbalzano 
secondo la legge della riflessione ottica all'interno di un dominio in $\R^2$.

\begin{defi} Sia $G$ uno tra $\N$, $\Z$ e $\R$. L'\emph{orbita} di un punto $x$ è l'insieme 
\[O_f(x)=\left\{f^g(x): g\in G \right\}.\] 
Un'orbita è \emph{periodica} se per un certo $n>0$ si ha $f^n(x)=x$
e in tal caso $x$ si dice \emph{periodico}. Più in generale, se
per qualche $n>m>0$ si ha $f^n(x)=f^m(x)$ (o equivalentemente se $f^m(x)$ è periodico per qualche
$m>0$), $x$ si dice \emph{pre-periodico}. 
Un punto è \emph{fisso} se $f(x)=x$.\end{defi}

Si potrebbe pensare di studiare lo spazio delle orbite, ovvero il quoziente di $X$ tramite
la relazione $x\sim y\sse f^n(x)=y$ per qualche $n$, 
ma questo risulta difficile e infruttuoso nella maggior parte dei casi. 
Un'identificazione molto utile è invece quella dei sistemi dinamici \emph{coniugati} o equivalenti, 
ovvero sistemi dati da $f$ e $g$ su spazi dello stesso tipo e con gli stessi tempi tali che esista $h$ isomorfismo con $h\circ f=g\circ h$. 
Si parla di \emph{dinamica} quando si pensa al sistema dinamico a meno di coniugio piuttosto che a quello particolare. 
Più in generale se $h$ è un morfismo di spazi che fa commutare
\[ \xymatrix{ X \ar[r]^f \ar[d]_h & X \ar[d]^h \\ Y \ar[r]_g & Y } \]
se $h$ è surgettiva si dice che $g$ è un \emph{fattore} di $f$ e $f$ \emph{estende} $g$; 
se $h$ è iniettiva $f$ è un \emph{sottosistema} di $g$ (in questo caso infatti succede che $Y$ contiene un sottospazio $g$-invariante su cui la restrizione di $g$ è coniugata a $f$). 

