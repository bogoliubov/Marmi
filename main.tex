\documentclass[italian,course]{Notes}

%Pacchetti

%\usepackage[utf8]{inputenc}
\usepackage[italian]{babel}
\usepackage{amsthm}
\usepackage{amsmath}
\usepackage{enumitem}
\usepackage{bbm}
\usepackage{bm}\usepackage{xcolor}
\usepackage{url}
\usepackage{accents}
\usepackage{hyperref}
\usepackage{caption}
\usepackage{subcaption}
\usepackage{booktabs}
%\usepackage{amscd}
%\usepackage{amsfonts}
%\usepackage{amssymb}
%\usepackage{mathtools}
%\usepackage{setspace}
%\usepackage{xcolor}
%\usepackage{makeidx}
%\usepackage[italian]{varioref}
\usepackage[all]{xy}
\usepackage[a4paper, left=3.5cm, right=3.5cm, top=4cm, bottom=4cm]{geometry}

%Comandi comuni

\newcommand{\Q}{\mathbb{Q}}
\newcommand{\A}{\mathcal{A}}
\newcommand{\N}{\mathbb{N}}
\newcommand{\Z}{\mathbb{Z}}
\newcommand{\R}{\mathbb{R}}
\newcommand{\C}{\mathbb{C}}
\newcommand{\B}{\mathcal{B}}
\newcommand{\T}{\mathbb{T}}
\newcommand{\LL}{\mathcal{L}}
\newcommand{\irr}{\mathbb{R}\smallsetminus\mathbb{Q}}
\renewcommand{\phi}{\varphi}
\DeclareMathOperator{\homeo}{Homeo}
\DeclareMathOperator{\aut}{Aut}
\DeclareMathOperator{\ndom}{End}
\DeclareMathOperator{\per}{Per}
\DeclareMathOperator{\diam}{diam}
\newcommand{\lcap}{{\fontencoding{T1}\selectfont\guillemotleft}}
\newcommand{\rcap}{{\fontencoding{T1}\selectfont\guillemotright}}
\newcommand{\disco}{\mathbb{D}}

%Comandi di Pig
\newenvironment{sistema}{\left\lbrace\begin{array}{@{}l@{}}}{\end{array}\right.}
\newenvironment{lista}{\begin{enumerate}[label=(\arabic*)]}{\end{enumerate}}
\newcommand{\floor}[1]{\left\lfloor #1\right\rfloor}
\newcommand\restr[2]{{
  \left.\kern-\nulldelimiterspace
  #1
  \vphantom{\big|}
  \right|_{#2}
  }}
\newcommand{\sse}{\Leftrightarrow}
\newcommand{\implica}{\Rightarrow}
\newcommand{\de}{\partial}
\newcommand{\interna}[1]{\accentset{\circ}{#1}}
\newcommand{\mz}{\frac{1}{2}}
\newcommand{\ang}[1]{\left\langle#1\right\rangle}
\newcommand{\uno}{\bm{1}}
\newcommand{\nin}{\not\in}
\newcommand{\nonzero}{\setminus\set{0}}
\newcommand{\Eacc}{È }
\newcommand{\fantasma}[1]{\leavevmode\phantom{#1}}
\newcommand{\acapo}{\fantasma{a} \\}
\renewcommand{\obar}[1]{\overline{#1}}
\newcommand{\ubar}[1]{\underline{#1}}
\newcommand{\set}[1]{\left\{#1\right\}}
\newcommand{\pa}[1]{\left(#1\right)}
\newcommand{\bra}[1]{\left[#1\right]}
\newcommand{\abs}[1]{\left|#1\right|}
\newcommand{\norm}[1]{\left\|#1\right\|}

%Comandi di Martino
\newcommand{\eps}{\varepsilon}
\newcommand{\imm}{\mathrm{Im} \,}
\newcommand{\imp}{\;\Rightarrow \;} 
\newenvironment{mice}{\left(\!\! \begin{array}}{\end{array}\!\!\right)}
\newenvironment{sist}[1][l]{\left\{\begin{array}{#1}}{\end{array} \right.}
\newcommand{\fun}[1][]{\stackrel{#1}{\rightarrow}}
\newcommand{\map}[1][]{\stackrel{#1}{\mapsto}}
\newcommand{\uni}{\cup}
\newcommand{\inter}{\cap}
\newcommand{\biguni}{\bigcup}
\newcommand{\biginter}{\bigcap}
\newcommand{\norma}[1]{\left\| #1 \right\|}
\newcommand{\id}{\mathrm{id}}
\newcommand{\scal}[2]{\langle #1, #2 \rangle}

%ambienti e teoremi

\newenvironment{soluz}{\emph{Soluzione.} }{\hfill\qedsymbol}
\newtheorem{teo}{Teorema}
\newtheorem{defi}[teo]{Definizione}
\newtheorem{cor}[teo]{Corollario}
\newtheorem{lem}[teo]{Lemma}
\newtheorem{prop}[teo]{Proposizione}
\newtheorem{oss}[teo]{Osservazione}
\newtheorem{esempio}[teo]{Esempio}
\newtheorem{esercizio}[teo]{Esercizio}
\everymath{\displaystyle}


\title{Sistemi Dinamici}
\subject{Appunti del Corso}
\author{Francesco Grotto, Martino Ottolini, Alessandro Pigati}
%\email{EMAIL}
\speaker{Stefano Marmi}
\date{01}{10}{2014}
\dateend{17}{12}{2014}
\place{Scuola Normale Superiore}


\begin{document}

\newpage

\begin{abstract}
 In questo scritto sono raccolti gli appunti del corso di Sistemi Dinamici. 
 Abbiamo preferito modificare in parte l'ordine di esposizione seguito a lezione per accorpare argomenti simili tra loro. 
 Abbiamo inoltre arricchito le dispense con alcuni fatti e dimostrazioni che non sono stati visti a lezione, segnalandoli come facoltativi. 
 La parte del corso tenuta da Carlo Carminati si trova, insieme ad altri argomenti che vi fanno riferimento, nella sezione sulle Frazioni Continue.
\end{abstract}

\section{Sistemi Dinamici}

\begin{defi}[Sistema dinamico secondo Smale]
 Un \emph{sistema dinamico} è l'azione di un gruppo o semigruppo $G$ su uno spazio $X$. 
 Con spazio intendiamo un insieme $X$ con una struttura di spazio, ad esempio topologico, vettoriale, di misura 
 \emph{et cetera}, con le relative trasformazioni che ne conservano la struttura $\ndom X$ e $\aut X$, 
 queste ultime con inversa a sua volta in $\ndom X$. L'azione è dunque un omomorfismo di gruppo (risp. semigruppo) $G\rightarrow \aut X$ (risp. $G\rightarrow \ndom X$).
\end{defi}

La definizione di sistema dinamico è molto ampia. 
Il gruppo $G$, normalmente inteso come insieme dei tempi, è di solito $\R$, e in tal caso si parla di sistemi dinamici a tempo continuo, 
oppure $\Z$ o $\N$ per il tempo discreto. Nulla impedisce però di considerare altri (semi)gruppi. Ci occuperemo quasi esclusivamente di sistemi dinamici a tempo discreto, 
per cui indicheremo con $(X,f)$ il sistema, con $f$ l'immagine del generatore di $\Z$ o $\N$. 
Dove non specificato i fatti enunciati vanno intesi nel modo appropriato per entrambe le situazioni: 
ad esempio, quando scriviamo $f^n$, $n$ sarà in $\N$ oppure $\Z$ e nel secondo caso bisognerà considerare anche le iterate di $f^{-1}$. 
Esempi di sistemi a tempo discreto sono le successioni numeriche, come quelle di Collatz, Thue-Morse, Kolakowski (che sono legate a diversi problemi aperti).
Sistemi a tempo continuo sono invece ad esempio i flussi integrali di campi vettoriali
su varietà.
Una classe importante di sistemi dinamici è quella dei \emph{biliardi}, che descrivono traiettorie di punti che rimbalzano 
secondo la legge della riflessione ottica all'interno di un dominio in $\R^2$.

\begin{defi} Sia $G$ uno tra $\N$, $\Z$ e $\R$. L'\emph{orbita} di un punto $x$ è l'insieme 
\[O_f(x)=\left\{f^g(x): g\in G \right\}.\] 
Un'orbita è \emph{periodica} se per un certo $n>0$ si ha $f^n(x)=x$
e in tal caso $x$ si dice \emph{periodico}. Più in generale, se
per qualche $n>m>0$ si ha $f^n(x)=f^m(x)$ (o equivalentemente se $f^m(x)$ è periodico per qualche
$m>0$), $x$ si dice \emph{pre-periodico}. 
Un punto è \emph{fisso} se $f(x)=x$.\end{defi}

Si potrebbe pensare di studiare lo spazio delle orbite, ovvero il quoziente di $X$ tramite
la relazione $x\sim y\sse f^n(x)=y$ per qualche $n$, 
ma questo risulta difficile e infruttuoso nella maggior parte dei casi. 
Un'identificazione molto utile è invece quella dei sistemi dinamici \emph{coniugati} o equivalenti, 
ovvero sistemi dati da $f$ e $g$ su spazi dello stesso tipo e con gli stessi tempi tali che esista $h$ isomorfismo con $h\circ f=g\circ h$. 
Si parla di \emph{dinamica} quando si pensa al sistema dinamico a meno di coniugio piuttosto che a quello particolare. 
Più in generale se $h$ è un morfismo di spazi che fa commutare
\[ \xymatrix{ X \ar[r]^f \ar[d]_h & X \ar[d]^h \\ Y \ar[r]_g & Y } \]
se $h$ è surgettiva si dice che $g$ è un \emph{fattore} di $f$ e $f$ \emph{estende} $g$; 
se $h$ è iniettiva $f$ è un \emph{sottosistema} di $g$ (in questo caso infatti succede che $Y$ contiene un sottospazio $g$-invariante su cui la restrizione di $g$ è coniugata a $f$). 


\section{Dinamica Topologica}

Consideriamo $X$ spazio metrico compatto, con $\ndom X=C(X,X)$ le funzioni continue e $\aut X=\homeo X$ gli omeomorfismi. 
Denoteremo il sistema dinamico con $(X,d,f)$, dove $d$ è la metrica dello spazio. Iniziamo con alcune definizioni generali.

\begin{defi} $x\in X$ è un \emph{punto errante} se ne esiste un intorno $U$ tale che 
\[\bigcup_n f^n(U)\cap U=\varnothing.\]
L'insieme di tali punti è detto l'\emph{insieme errante} del sistema. Un insieme $A$ è \emph{invariante} se $f(A)\subseteq A$, 
e per omeomorfismi si dice \emph{totalmente invariante} se lo è anche negativamente, cioè per $f^{-1}$, ovvero si ha $f^{-1}(A)=A=f(A)$. 
Gli insiemi di $\alpha$ e $\omega$-limite ($\alpha$ solo per omeo) sono
\[\alpha(x)=\bigcap_{n<0}\obar{\bigcup_{m\leq n}f^m(x)}\qquad \omega(x)=\bigcap_{n>0}\obar{\bigcup_{m\geq n}f^m(x)} \]
ovvero un $\omega$-limite è un punto $y$ limite di una sottosuccessione di $f^n(x)$. 
Se $x\in\omega(x)$ (risp. $x\in\alpha(x)$) si dice positivamente (risp. negativamente) ricorrente, e nel caso accadano entrambe le cose si dice che $x$ è \emph{ricorrente}, 
il che equivale a dire $x$ non errante. I punti ricorrenti sono quindi l'insieme non errante.
\end{defi}

\begin{esercizio}
 Gli insiemi $\omega(x)$, $\alpha(x)$ e l'insieme non errante sono chiusi e invarianti.
\end{esercizio}

%%%%%%%%%%%%%%%%%%%%%%%%%%%%%%%%%%%%%%%%%%%%%%%%%%%%%%%%%%%%%%%%%%%%%%%%%%%%%%%%%%%%%%
\subsection{Transitività}

\begin{defi} $(X,d,f)$ si dice \emph{topologicamente transitivo} se un punto $x\in X$ ha orbita densa, si dice \emph{minimale} se tutte le orbite sono dense.
\end{defi}

\begin{prop}Se $f\in\homeo (X)$ sono equivalenti:
\begin{lista}
	\item $f$ topologicamente transitivo;
	\item se $U$ è un aperto $f$-invariante (ovvero $U=f(U)=f^{-1}(U)$), allora $U=\emptyset$ oppure $\obar{U}=X$;
	\item se $U,V$ sono aperti non vuoti, esiste $N\in\Z$ tale che $f^N(U)\cap V\neq\emptyset$;
	\item $\set{x:\obar{O_f(x)}=X}$ è un $G_\delta$ denso, ovvero è \emph{generico}.
\end{lista}
\end{prop}

\begin{proof}$(1)\implica (2)$: sia $x$ un punto con l'orbita densa e $U$ un aperto $f$-invariante, che supponiamo non vuoto.
Da $O_f(x)\cap U\neq\emptyset$ segue che per qualche $j\in\Z$ $f^j(x)\in U$, quindi (invarianza di $U$) $O_f(x)\subseteq U$,
perciò $X=\obar{O_f(x)}\subseteq\obar{U}$. \\
$(2)\implica (3)$: poniamo $U':=\cup_{n\in\Z}f^n(U)$; $U'$ è non vuoto e invariante $\implica\obar{U'}=X$
$\implica \obar{U'}\cap V\neq\emptyset \implica U'\cap V\neq\emptyset \implica \exists N\ f^N(U)\cap V\neq\emptyset$. \\
$(3)\implica(4)$: sia $\set{U_n}$ una base numerabile di aperti. Ora $\obar{O_f(x)}=X\sse \forall n\ \exists j\ f^j(x)\in U_n
\sse \forall n\ \exists m\ x\in f^m(U_n)\sse x\in\cap_n\cup_{m\in\Z} f^m(U_n)$.
Per ipotesi, per ogni $n$ l'insieme $\cup_{m\in\Z} f^m(U_n)$ è un aperto denso.
Quindi $\set{x:\obar{O_f(x)}=X}$ è un $G_\delta$ e per il teorema di Baire è anche denso. \\
$(4)\implica (1)$: ovvio perché l'insieme $\set{x:\obar{O_f(x)}=X}$, essendo denso, è anche non vuoto.
\end{proof}

\begin{oss}\label{osstransitivita} Se $f:X\to X$ è solo continua (cioè è un endomorfismo), è facile dimostrare che le condizioni $(1)$ e $(3)$ sono
  ancora equivalenti se assumiamo che $X$ non abbia punti isolati.\newline
  Osserviamo inoltre che nel caso $f\in\homeo (X)$ la terza condizione del teorema è equivalente alla \emph{decomponibilità}: 
  $f$ non transitiva se e solo se esistono due aperti disgiunti (totalmente) invarianti.
\end{oss}

\begin{prop}Se $f:X\to X$ è topologicamente transitivo, gli unici \emph{integrali primi} continui (cioè
funzioni $F:X\to\R$ continue tali che $F\circ f=F$) sono le costanti. Il viceversa, in generale, è falso (cfr. gli Esempi).
\end{prop}

\begin{proof}Una tale $f$ deve essere costante sull'orbita di $x$ (il solito punto con orbita densa), quindi per continuità anche
sulla chiusura $\obar{O_f(x)}=X$.
\end{proof}

\begin{defi} $(X,d,f)$ è \emph{topologicamente mescolante} se dati aperti $U,V$ esiste $N\in\Z$ tale che per ogni $n\geq N$ vale $f^n(U)\cap V\neq \emptyset$.\end{defi}

\begin{oss} Per la caratterizzazione vista, mescolante implica transitivo. Non vale il viceversa, e un esempio sono le rotazioni di $S^1$. 
 
\end{oss}



%%%%%%%%%%%%%%%%%%%%%%%%%%%%%%%%%%%%%%%%%%%%%%%%%%%%%%%%%%%%%%%%%%%%%%%%%%%%%%%%%%%%
\subsection{Esempi}

I seguenti esempi di sistemi dinamici sono qui presentati come dinamiche topologiche, ma torneranno anche nello studio di quelle misurabili.

\begin{esempio}[Rotazioni Irrazionali] Consideriamo le rotazioni su $\T^1=\R/\Z$: $f(x)=x+\alpha$. 
Non è difficile vedere che se $\alpha\in\Q$ tutti i punti sono periodici, mentre se $\alpha\in\irr$ ogni orbita è densa, ovvero il sistema è minimale 
(cfr. la versione generalizzata al toro $d$-dimensionale nella parte sui Flussi). 
Questo non è un caso, come mostra il risultato seguente. \end{esempio}
\begin{lemma}[facoltativo] Sia $G$ un gruppo topologico: se $g_0\in G$ è tale che l'applicazione $g\mapsto g_0\cdot g$ (oppure $g\mapsto g\cdot g_0$) è topologicamente transitiva, 
allora è anche minimale. Se inoltre $G$ è metrizzabile, è un gruppo abeliano. 
\end{lemma}
\begin{proof}
 Se $\{g_0^kg_1\}_k$ è densa, poichè $g_0^kg=(g_0^kg_1)(g_1^{-1}g)$ si ha che l'orbita di $g$ si ottiene con una traslazione, che per ipotesi è un omeomorfismo, di un'orbita densa. 
 Per il secondo punto, abbiamo che il sottogruppo ciclico $G'=\{g_0^kg_1\}_k$ è denso, quindi poichè lo spazio è metrico si usa il criterio di continuità per successioni: 
 se $g_n\rightarrow g$ e $h_n\rightarrow h$ con $g_n, h_n \in G'$, si ha $[g,h]\leftarrow [g_n,h_n]=e$, e quindi per continuità $[g,h]=e$.
\end{proof}




\begin{esempio}[Raddoppiamento e Dinamica Simbolica] Le dinamiche date da successioni di numeri sono importanti perchè sono un oggetto a cui 
spesso ci si riconduce nello studio dei sistemi dinamici. Consideriamo
\[ \xymatrix{ \{0,1\}^\N \ar[r]^\sigma \ar[d]_\pi & \{0,1\}^\N \ar[d]^\pi \\ [0,1] \ar[r]_S & [0,1] } \]
 dove $S(x)=2x(\mbox{mod } 1)$ è il \emph{raddoppiamento} o \emph{dente di sega}, $\sigma(a_1,a_2,\dots)=(a_2,a_3,\dots)$ è lo \emph{shift} e 
 \[\pi(a_1,a_2,\dots)=\sum_1^\infty a_k 2^{-k}.\]
 La mappa $\pi$ è iniettiva e dunque $S$ risulta essere un sottosistema di $\sigma$. In questo modo si vede subito che il raddoppiamento è transitivo non minimale: 
 lavorando in $\{0,1\}^\N$, la successione ottenuta concatenando tutte le possibili parole finite di soli $0$ e $1$ (che sono numerabili) corrisponde a un punto con orbita densa; 
 invece le successioni periodiche corrispondono a punti periodici (se ne trovano di ogni ordine). 
 \Eacc importante osservare che i due sistemi non possono essere coniugati perchè $[0,1]$ e $\{0,1\}^\N$ sono rispettivamente omeomorfi a $\R$, connesso, e all'insieme di Cantor, 
 totalmente sconnesso. \newline
 Osserviamo inoltre che il raddoppiamento è un esempio di sistema topologicamente mescolante.
\end{esempio}



\begin{esempio}[Tenda e Logistic Map] Un altro esempio di dinamiche coniugate: 
su $[0,1]$ sono coniugate tramite $h(x)=\sin^2\left(\frac{2\pi}{2}\right)$ la mappa \emph{tenda} 
$f(x)=\begin{cases}2x \quad x\leq 1/2 \\ 2-2x \quad x\geq 1/2\end{cases}$ e la \emph{mappa logistica} $Q_4(x)=4x(1-x)$, $Q_4\circ h=h\circ f$.
La mappa tenda è transitiva, infatti per gli intervalli della forma $\bra{\frac{i}{2^k},\frac{i+1}{2^k}}$ l'immagine
tramite $T^n$ è definitivamente tutto $[0,1]$ (si dice che $T$ è \emph{espansiva}). Per cui anche la mappa logistica è transitiva (vedi anche il prossimo esempio).
\end{esempio}



\begin{esempio}[Transitività e fattori] Se $g$ è un fattore di $f$ e $f$ è topologicamente transitiva, allora lo è anche $g$.
Infatti per qualche $h:X\to Y$ surgettiva è $gh=hf$, quindi $g^k h=h f^k$, perciò
se $x\in X$ ha orbita densa abbiamo $O_g(h(x))=h(O_f(x))$, che è un insieme denso essendo $h$ surgettiva.\newline
Un esempio (un po' improprio perchè $S$ è discontinua) è il seguente: se
\[S(x):=\begin{cases}2x &\text{se }0\le x<\mz \\ 2x-1 &\text{se }\mz\le x\le 1\end{cases}\]
allora commuta il diagramma
\[ \xymatrix{X \ar[r]^S \ar[d]_T & X \ar[d]^T \\ X \ar[r]_T & X} \]
per cui $T$ è un fattore di $S$, e dunque abbiamo un'altra dimostrazione che $T$ è transitivo. 
\end{esempio}



\begin{esempio} Avere solo integrali primi costanti non implica transitività. Sia $T:=\R^2/\Z^2$ il $2$-toro e $A\in SL(2,\Z)$ una trasformazione ergodica
su $T$, ad esempio il \emph{gatto di Arnold},
\[A=\begin{pmatrix} 2 & 1 \\ 1 & 1 \end{pmatrix}\]
Costruiamo $X$ partendo da due copie del toro, $T$ e $T'$ e identificando (quoziente topologico) i loro elementi neutri. 
La mappa su $T\sqcup T'$ che si ottiene accostando $A:T\to T$ e $A:T'\to T'$ induce una mappa $f:X\to X$.
$f$ non è topologicamente transitiva essendo i sottospazi $T$ e $T'$ invarianti (stiamo identificando $T$ con la sua proiezione a quoziente,
come pure $T'$).
Un integrale primo $F:X\to \R$ però è costante: infatti che $\restr{f}{T}:T\to T$ è topologicamente transitivo 
(essendo ergodico, vedi il relativo capitolo),
perciò $F$ è costante su $T$ e lo stesso vale per $T'$. Siccome (in $X$) $T$ e $T'$ si intersecano, deduciamo che $F$ è costante su tutto $X$. 
\end{esempio}



%%%%%%%%%%%%%%%%%%%%%%%%%%%%%%%%%%%%%%%%%%%%%%%%%%%%%%%%%%%%%%%%%%%%%%%%%%%%%%%%%%%%%%%%%%%%%%%%%%%%%%%%%%%%%
\subsection{Sistemi Caotici}


\begin{defi} $(X,d,f)$ è \emph{caotico secondo Devaney} se è topologicamente transitiva e l'insieme dei punti periodici $\per(f)$ è denso.
\end{defi}

% DEFINIZIONE RIMPIAZZATA DA QUELLA SEGUENTE
% \begin{defi}$f$ ha \emph{dipendenza sensibile} dalle condizioni iniziali se c'è un $\delta>0$ tale che
% per ogni $x\in X$ e per ogni suo intorno $U$ si possano sempre trovare un $y\in U$ e un $n\ge 0$ per cui
% $d\pa{f^n(x),f^n(y)}>\delta$. In altre parole, pensando l'intorno $U$ come un errore di approssimazione di $x$,
% per quanto sia buona l'approssimazione c'è sempre un $y\approx x$ la cui orbita (a qualche istante $n$) si separa da quella di $x$
% più della tolleranza $\delta$ fissata.
% \end{defi}

\begin{defi}$(X,d,f)$ ha \emph{dipendenza sensibile dai dati in} $x_0\in X$ se 
 \[\exists \epsilon>0:\quad \forall U \mbox{ intorno di } x_0 \quad \exists y_0\in U:\quad\exists n\geq 0: \quad d(f^nx_0,f^ny_0)>\epsilon\]
 e se questo vale $\forall x\in X$ diremo che la dinamica ha \emph{dipendenza sensibile}.
 Viceversa $(X,d,f)$ è \emph{stabile secondo Lyapunov} in $x_0\in X$ se
 \[\forall \epsilon>0 \quad \exists U \mbox{ intorno di } x_0: \quad \forall y_0\in U\quad\forall n\geq 0 \quad d(f^nx_0,f^ny_0)<\epsilon\]
 (contronominale della precedente) ovvero se le iterate $\{f^n\}$ sono equicontinue.
\end{defi}

\begin{teo}Se $f$ è caotica e $X$ non ha punti isolati, $f$ ha dipendenza sensibile dalle condizioni iniziali.
\end{teo}

\begin{lem}Se ci sono due orbite periodiche distinte, esiste una costante $c>0$ tale che per ogni $x\in X$
c'è un punto periodico $p$ la cui orbita dista tutta più di $c$ da $x$. In altre parole
$d(x,f^k(p))>c$ per ogni $k\ge 0$.
\end{lem}

\begin{proof}Siano $y$ e $z$ periodici con orbite distinte (e quindi disgiunte).
Sia $c>0$ tale che $d\pa{O_f(y),O_f(z)}>2c$. Per ogni $x\in X$ abbiamo
\[ d\pa{x,O_f(y)}+d\pa{x,O_f(z)}\ge d\pa{O_f(y),O_f(z)}, \]
quindi almeno uno tra $d\pa{x,O_f(y)}$ e $d\pa{x,O_f(z)}$ è $>c$.
\end{proof}

\begin{proof}[Dimostrazione del teorema]Se esiste una sola orbita periodica,
da $\obar{\per(f)}=X$ segue che $X$ è finito e quindi tutti i punti sarebbero isolati, assurdo.\newline
Perciò siamo nelle ipotesi del lemma: poniamo $\delta:=\frac{c}{4}$, con $c$ la costante che il lemma ci fornisce.
Sia $x\in X$ e $U'$ un suo intorno; definiamo $U:=U'\cap B(x,\delta)$. Per densità di $\per(f)$ esiste $q\in U$ periodico,
diciamo di periodo $n$. Per il lemma esiste un punto $p$ periodico con $d\pa{x,O_f(p)}>c=4\delta$. Inoltre per transitività
esiste un $z\in U$ che nella sua vita resta vicino all'orbita di $p$ per almeno $n$ passi consecutivi:
infatti posto $W_i:=B\pa{f^i(p),\delta}$ (per $i=1,\dots,n$) e $V:=\cap_{i=1}^n f^{-i}(W_i)$ (non vuoto perché $p\in V$)
sappiamo che esistono $z\in U$ e $k\ge 0$ tali che $f^k(z)\in V$ (cfr. Osservazione \ref{osstransitivita}). $z$ soddisfa
$f^{k+i}(z)\in W_i$ per $i=1,\dots,n$. Per uno di questi indici $i$, $k+i$ è multiplo di $n$,
perciò $f^{k+i}(q)=q$, da cui otteniamo
\[ \begin{split} &d\pa{f^{k+i}(q),f^{k+i}(z)}\ge d\pa{q,f^i(p)}-d\pa{f^{k+i}(z),f^i(p)}\ge \\
&\ge d\pa{x,f^i(p)}-d\pa{x,q}-d\pa{f^{k+i}(z),f^i(p)}>4\delta-\delta-\delta=2\delta \end{split} \]
usando tante volte la disuguaglianza triangolare. Ma allora almeno uno tra
$d\pa{f^{k+i}(x),f^{k+i}(q)}$ e $d\pa{f^{k+i}(x),f^{k+i}(z)}$ è maggiore di $\delta$ (sempre per la triangolare), che è la tesi
essendo $q,z\in U$.
\end{proof}




\section{Dinamica Misurabile}

Sia $(X,\A,\mu)$ uno spazio di probabilità. Un sistema dinamico misurabile è dato da una mappa $f:X\rightarrow X$ che preserva la misura, 
ovvero tale che $f^{-1}\A\subseteq \A$ (misurabile) e per ogni $A\in\A$ vale $\mu(f^{-1}A)=\mu(A)$. (Scegliamo questa definizione con $f^{-1}$ coerentemente con la nozione di misura immagine.)
I sistemi dinamici topologici considerati finora sono anche sistemi misurabili, come prova il seguente

\begin{teo}[Krylov-Bogoliubov] Dati $X$ spazio metrico compatto e $f:X\to X$ continua, esiste sempre una probabilità $f$-invariante.
\end{teo}

\begin{proof}Sia $C(X)$ lo spazio (di Banach) delle funzioni continue a valori reali su $X$. Sappiamo che il suo duale è lo spazio $C(X)'=\mathcal{M}(X)$ delle misure finite con segno.
Sia $\mu$ una qualsiasi misura di probabilità su $X$ (ad esempio una delta). ``Spalmiamo'' questa misura tramite $f$, ottenendo misure sempre più vicine all'essere invarianti:
poniamo $\mu_n:=\frac{1}{n}\sum_{k=0}^{n-1}(f^k)_*\mu$ (cioè $\mu_n$ è la media dei pushforward di $\mu$ tramite le prime $n$ iterate di $f$). \\
Le misure $\mu_n$ sono tutte positive e di probabilità, quindi per ogni $n$ $\norm{\mu_n}=1$. Ma la palla unitaria (chiusa) di $\mathcal{M}(X)$, dotata
della topologia debole*, è compatta (per Banach-Alaoglu) ed è anche metrizzabile (essendo $C(X)$ separabile). Quindi esiste una sottosuccessione convergente
debole*: $\mu_{n_k}\overset{*}{\rightharpoonup}\nu$. Il limite $\nu$ è una misura positiva (essendo per ogni $\phi\in C(X)$ nonnegativa $\int_X \phi\,d\nu=
\lim_{n\to\infty}\int_X\phi\,d\mu_n\ge 0$) e di probabilità (perché $\nu(X)=\int_X 1\,d\nu=\lim_{n\to\infty}\int_X 1\,d\mu_n=1$). \\
Verifichiamo che $\nu$ è $f$-invariante: per ogni $\phi\in C(X)$ vale
\[ \begin{split}&\int_X \phi\,d\nu-\int_X \phi\,d(f_*\nu)=\int_X \pa{\phi-\phi\circ f}\,d\nu=\lim_{n\to\infty}\int_X \pa{\phi-\phi\circ f}\,d\mu_n \\
&=\lim_{n\to\infty}\pa{\int_X \frac{\phi+\phi\circ f+\dots+\phi\circ f^{n-1}}{n}\,d\mu-\int_X \frac{(\phi\circ f)+\dots+(\phi\circ f)\circ f^{n-1}}{n}\,d\mu} \\
&=\lim_{n\to\infty}\int_X\frac{\phi-\phi\circ f^n}{n}\,d\mu_n=0 \end{split} \]
essendo $\abs{\int_X\frac{\phi-\phi\circ f^n}{n}\,d\mu_n}\le\frac{2}{n}\norm{\phi}_\infty\to 0$. Questo dice che $\nu=f_*\nu$ come elementi del duale di $C(X)$,
quindi (siccome $C(X)'=\mathcal{M}(X)$) le misure $\nu$ e $f_*\nu$ coincidono.
\end{proof}

Una bigezione che coniuga un sistema misurabile con uno qualsiasi induce una misura invariante: è questo il caso dei seguenti esempi.

\begin{esempio} La mappa tenda $T$ preserva la misura di Lebesgue. \Eacc coniugata alla mappa logistica $Q_4(x)=4x(1-x)$ tramite $\sin^2\left(\frac{\pi x}{2}\right)$,
 e da ciò si ricava la misura invariante $\frac{dx}{\pi\sqrt{x(1-x)}}$. 
\end{esempio}

\begin{esempio} La mappa logistica $Q_4$ è coniugata alla \emph{mappa di Ulam} 
 \[U:\R\rightarrow\R \qquad x\mapsto \ln \left(\frac{4 e^x}{(1-e^x)^2}\]
tramite 
 \[k:[0,1]\rightarrow \R \qquad x\mapsto \ln\left(\frac{x}{1-x} \]
 
\end{esempio}










\begin{defi}Dato $A\in\mathcal{A}$ e $T$ intero positivo poniamo $\nu_T(x,T,A):=\frac{1}{T}\sum_{j=0}^{T-1}\chi_A\pa{f^j(x)}$.
Questa è la \emph{frequenza media delle visite} ad $A$ dell'orbita di $x$ tra $0$ e $T$. \\
Poniamo anche $\obar{\nu}(x,A):=\obar{\lim}_{T\to\infty}\nu(x,T,A)$ e $\ubar{\nu}(x,A):=\ubar{\lim}_{T\to\infty}\nu(x,T,A)$.
\end{defi}

\begin{defi}Più in generale, data un'osservabile $\phi\in L^1(X)$, definiamo la \emph{somma di Birkhoff} $(S_T\phi)(x):=\sum_{j=0}^{T-1}(\phi\circ f^j)(x)$
e la \emph{media temporale} $\frac{(S_T\phi)(x)}{T}$.
\end{defi}

Dimostreremo questo risultato, che per ora diamo per buono:

\begin{teo}[Birkhoff]Per ogni $A\in\mathcal{A}$ esiste q.o. $\nu(x,A):=\lim_{T\to\infty}\nu(x,T,A)$. \\
Più in generale per ogni $\phi\in L^1(X)$ esiste q.o. il $\lim_{T\to\infty}\frac{(S_T\phi)(x)}{T}$.
\end{teo}

\begin{defi}$(X,\mathcal{A},\mu,f)$ è \emph{ergodico} se per ogni $A\in\mathcal{A}$ vale $\nu(x,A)=\mu(A)$ q.o.
Cioè \lcap la statistica delle orbite coincide con la probabilità a priori\rcap.
\end{defi}

\begin{teo}Sono equivalenti:
\begin{lista}
	\item $X$ ergodico 
	\item per ogni $A\in\mathcal{A}$ con $f(A)\subseteq A$, uno tra $A$ e $A^c$ è trascurabile 
	\item ogni integrale primo $\phi$ (cioè $\phi\in L^1(X)$ tale che $\phi=\phi\circ f$ q.o.) è costante q.o. 
	\item per ogni $\phi\in L^1(X)$ vale $\frac{1}{T}(S_T\phi)(x)\to\int_X \phi\,d\mu$ per q.o. $x$ 
	\item per ogni $A,B\in\mathcal{A}$ vale $\lim_{n\to\infty}\frac{1}{n}\sum_{j=0}^{n-1}\mu\pa{f^{-j}(A)\cap B}=\mu(A)\mu(B)$. 
\end{lista}
\end{teo}

\begin{proof}$(1)\implica(2)$: sia $A\in\mathcal{A}$ tale che $f(A)\subseteq A$; possiamo assumere $\mu(A)>0$. Chiaramente per ogni $x\in A$
è $\nu(x,A)=1$ (perché $f^j(x)\in A$ per ogni $j\ge 0$), ma $\nu(x,A)$ può essere diverso da $\mu(A)$ solo su un insieme trascurabile, quindi $\mu(A)=1$. \\
$(2)\implica(3)$: per chiarire l'idea vediamo prima il caso in cui $\phi=\phi\circ f$ ovunque.
Poniamo $A_\gamma:=\set{x:\phi(x)\le\gamma}$ e osserviamo che
\[ f^{-1}(A_\gamma)=\set{x:\phi\circ f(x)\le\gamma}=A_\gamma. \]
Quindi $f\pa{f^{-1}(A_\gamma)}\subseteq A_\gamma=f^{-1}(A_\gamma)$,
cioè $f^{-1}(A_\gamma)$ è invariante (nel senso del punto $(2)$),
da cui $\mu(A_\gamma)=\mu\pa{f^{-1}(A_\gamma)}\in\set{0,1}$. \\
Ma la funzione $\gamma\mapsto\mu(A_\gamma)$ è crescente e ha limiti $0$ per $\gamma\to -\infty$ e $1$ per $\gamma\to +\infty$,
perciò $\obar{\gamma}:=\inf\set{\gamma:\mu(A_\gamma)=1}\in\R$. Essendo anche $A_{\obar{\gamma}}=\cap_n A_{\obar{\gamma}+\frac{1}{n}}$
e $\set{x:\phi(x)<\obar{\gamma}}=\cup_n A_{\obar{\gamma}-\frac{1}{n}}$ otteniamo $\mu(A_{\obar{\gamma}})=1$
e $\mu\pa{\set{x:\phi(x)<\obar{\gamma}}}=0$, cioè $\phi\equiv\obar{\gamma}$ q.o. \\
Il caso in cui $\phi=\phi\circ f$ vale q.o. si sistema facilmente: sia $B_\gamma:=\set{x:\phi\circ f\le\gamma}$;
siccome $\mu\pa{A_\gamma\Delta B_\gamma}=0$ possiamo scrivere $A_\gamma=B_\gamma\Delta N$ con $N$ trascurabile.
Ora $B_\gamma=f^{-1}(A_\gamma)=f^{-1}(B_\gamma)\Delta f^{-1}(N)$, da cui
(posto $\tilde{N}:=f^{-1}(N)$, trascurabile) $f\pa{B_\gamma\Delta\tilde{N}}\subseteq B_\gamma=\pa{B_\gamma\Delta\tilde{N}}\Delta\tilde{N}$,
ovvero $B_\gamma\Delta\tilde{N}$ è ``quasi invariante''. Per la ?? e l'ipotesi $(2)$ segue
$\mu(A_\gamma)=\mu(B_\gamma)=\mu(B_\gamma\Delta\tilde{N})\in\set{0,1}$ e si conclude come nel caso precedente. \\
$(3)\implica (4)$: possiamo assumere $\phi\ge 0$.
Essendo $\norm{\phi\circ f^j}_1=\norm{\phi}_1$ per ogni $j\ge 0$, abbiamo $\norm{\frac{S_T\phi}{T}}_1\le \norm{\phi}_1$ per ogni $T>0$.
Il teorema di Birkhoff dice che è ben definito q.o. $\tilde{\phi}:=\lim_{T\to\infty}\frac{S_T\phi}{T}$ e per il lemma di Fatou $\tilde{\phi}\in L^1(X)$.
Ora $\tilde{\phi}$ è un integrale primo, quindi è costante q.o., diciamo $\tilde{\phi}=c$ q.o. Integrando otteniamo $c=\int_X\tilde{\phi}
\le\ubar{\lim}_{n\to\infty}\int_X\frac{S_T\phi}{T}\,d\mu=\int_X\phi\,d\mu$ (dato che $f$ conserva $\mu$). Quindi $\lim_{T\to\infty}\frac{S_T\phi}{T}\le \int_X\phi\,d\mu$. \\
Per i troncamenti $\phi\wedge N$, per convergenza dominata, vale di più: $\lim_{T\to\infty}\frac{S_T(\phi\wedge N)}{T}=\int_X \phi\wedge N\,d\mu$. Dunque
\[ \int_X\phi\wedge N=\lim_{T\to\infty}\frac{S_T(\phi\wedge N)}{T}\le\lim_{T\to\infty}\frac{S_T\phi}{T}\le \int_X\phi\,d\mu \]
e mandando $N\to\infty$ otteniamo la tesi per convergenza monotona. \\
$(4)\implica (1)$: basta scegliere $\phi:=\chi_A$. \\
$(4)\implica(5)$: ponendo $\phi:=\chi_A$ otteniamo $\phi\circ f^j=\chi_{f^{-j}(A)}$, da cui
\[ \frac{1}{n}\sum_{j=0}^{n-1}\mu\pa{f^{-j}(A)\cap B}=\int_B\frac{S_n\phi}{n}\,d\mu\to\int_B\pa{\int_X\phi\,d\mu}\,d\mu=\mu(B)\mu(A) \]
(abbiamo usato il teorema di convergenza dominata). \\
$(5)\implica (2)$: dato $A$ invariante, scegliamo $B:=A^c$ e osserviamo che $A\subseteq f^{-j}(A)$. Perciò
\[ \mu(A)=\mu\pa{f^{-j}(A)}=\mu(A)+\mu\pa{f^{-j}(A)\cap B}, \]
da cui segue che $\mu\pa{f^{-j}(A)\cap B}=0$ per ogni $j\ge 0$. Ma allora, per quanto afferma $(5)$, $\mu(A)\mu(B)=0$.
\end{proof}

\begin{oss}Se un insieme $A\in\mathcal{A}$ è ``quasi invariante'', nel senso che $f(A)\subseteq A\cup N$ per qualche
$N$ trascurabile, $A$ è equivalente a un insieme $A'$ invariante (cioè $\mu(A\Delta A')=0$ e $f(A')\subseteq A'$):
basta porre $N':=\cup_{j=0}^\infty$ e $A':=A\setminus N'$. Ora se $x\in A'$ abbiamo
$f(x)\in A\cup N$ e $f(x)\nin N'$ (perché $f^{-1}(N')\subseteq N'$), quindi $f(A')\subseteq (A\cup N)\setminus N'=A\setminus N'=A'$.
\end{oss}

?? sarebbe meglio mettere l'oss dopo la def. dei s.d. misurabili

\begin{oss}Se il sistema dinamico è \emph{strongly mixing}, ovvero per ogni $A,B\in\mathcal{A}$ vale
$\lim_{j\to\infty}\mu\pa{f^{-j}(A)\cap B}=\mu(A)\mu(B)$ (\lcap decadimento delle correlazioni\rcap),
la condizione $(5)$ del teorema è banalmente verificata. \\
Quindi strongly mixing $\implica$ ergodico.
\end{oss}

\begin{oss}Sia $(X,d,f)$ uno spazio metrico compatto con una trasformazione $f:X\to X$ continua che preserva
una misura $\mu$. $X$ ha contemporaneamente le strutture di sistema dinamico topologico
e misurabile. Ci chiediamo se ci sono relazioni tra la transitività topologica e l'ergodicità. \\
Assumendo che ogni aperto non vuoto abbia misura positiva, l'ergodicità implica la transitività:
dati $U,V$ aperti non vuoti è $\lim_{n\to\infty}\frac{1}{n}\sum_{j=0}^{n-1}\mu\pa{f^{-j}(U)\cap V}=\mu(U)\mu(V)>0$,
quindi per qualche $N\ge 0$ abbiamo $\mu\pa{f^{-N}(U)\cap V}>0$. \\
%Dato un aperto $W$ vale sempre $\mu\pa{f(W)}\ge\mu(W)$: infatti $f(W)$ è un boreliano (essendo $W$ $\sigma$-compatto)
%e $\mu\pa{f(W)}=\mu\pa{f^{-1}\pa{f(W)}}\ge\mu(W)$ (dato che $f^{-1}\pa{f(W)}\supseteq W$). \\
%Applicando questo a $f^{-N}(U)\cap V$ e alla mappa $f^N$ (che preserva $\mu$) otteniamo
%\[ \mu\pa{U\cap f^N(V)}\ge\mu\pa{f^N\pa{f^{-N}(U)\cap V}}\ge\mu\pa{f^{-N}(U)\cap V}>0, \]
Dunque $f^{-N}(U)\cap V\neq\emptyset$ e la sua immagine tramite $f^N$ è inclusa in $U\cap f^N(V)$, perciò
quest'ultimo insieme è non vuoto e questa è la tesi (v. anche l'Osservazione ??).
\end{oss}

\begin{esempio}Sia $\alpha\in\R\setminus\Q$ un irrazionale. Il toro $\mathbb{T}^1$
con la misura di Lebesgue e la trasformazione $R_\alpha(x)=x+\alpha$ (la solita rotazione irrazionale)
è un sistema ergodico. Infatti il teorema precedente vale anche se nei punti $(3)$ e $(4)$ ci restringiamo a osservabili in $L^2$
(con la stessa dimostrazione: abbiamo usato solo il fatto che le funzioni caratteristiche sono $L^1$, ma
sono anche $L^p$ per ogni $p$). Ora, dato un integrale primo $\phi\in L^2(X)$, i suoi coefficienti di Fourier soddisfano
\[ \widehat{\phi}(k)=\widehat{\phi\circ R_\alpha}(k)=e^{2\pi i k\alpha}\widehat{\phi}(k) \]
per ogni $k\in\Z$. Ma se $k\neq 0$ abbiamo $e^{2\pi i k\alpha}\neq 1$ (perché $\alpha$ è irrazionale), quindi
$\widehat{\phi}(k)=0$ per $k\neq 0$ e questo dice che $\phi$ è costante q.o.
\end{esempio}


\begin{teo}[ergodico di von Neumann] Sia $U$ un'isometria di $H$ spazio di Hilbert, allora
 \[\forall\phi\in H \quad \exists \lim_{n\rightarrow\infty}\frac{1}{n}\sum_{j=0}^{n-1}U^j\phi:=\hat\phi\]
 e inoltre $U\hat\phi=\phi$. Per cui in una dinamica misurabile $(X,\A,\mu,f)$, considerando nello spazio di Hilbert $L^2_\C(X,\A,\mu)$ l'\emph{operatore di Koopman} $U_f$ tale che $U_f \phi=\phi\circ f$, che è un'isometria essendo $\mu$ $f$-invariante, si ha che le somme di Birkhoff convergono in $L^2$. 
 \end{teo}

\begin{oss}
 In dimensione finita, cioè per $U$ unitario in un $\C$-spazio $k$-dimensionale, il teorema spettrale permette di diagonalizzare $U$ riconducendosi al caso monodimensionale sugli autospazi: $Uz=e^{i\alpha}z$ per $z\in\C$. Allora
 \[\frac{1}{n}\sum_{j=0}^{n-1}e^{ij\alpha}z=\begin{cases} z \qquad \alpha=0 \\ \frac{1}{n}\frac{1-e^{in\alpha}}{1-e^{i\alpha}}z \overset{n\rightarrow\infty}{\longrightarrow}0 \qquad \alpha\neq 0\end{cases}   .                                                                                                                                                                                            \]
 Per applicare questo ragionamento in dimensione infinita servirebbe il teorema spettrale relativo (Stone). La dimostrazione che daremo è di Riesz.
\end{oss}

\begin{proof}
 Sia $D=\{\phi\in H|\exists\psi\in H : \phi=U\psi-\psi\}$ (insieme dei \emph{cobordi}); anzitutto mostriamo che $\forall \theta \in D^\bot \quad U\theta=\theta$. Infatti se $\theta \in D^\bot$, si ha
 \[(\theta, U\theta-\theta)=0\]
 \[(U\theta,U\theta-\theta)=(U\theta,U\theta)-(U\theta,\theta)=(\theta-U\theta,\theta)=0\]
 sommando quindi $(U\theta-\theta,U\theta-\theta)=0$ da cui quanto detto.
 Sia ora $I=\{\phi\in H | U\phi=\phi\}$, allora $(D\cup I)^\bot=\{0\}$, e quindi $D\cup I$ è denso in $H$. Osserviamo che in $D\cup I$ vale la tesi, per cui concludiamo vedendo che la tesi vale anche in $\overline{D\cup I}$, ovvero che la tesi passa al limite. Sia $\phi_k\rightarrow\phi$ di Cauchy in $D\cup I$, ovvero $h,k$-definitivamente $\|\phi_h-\phi_k\|<\epsilon$, allora anche $\|\hat\phi_h-\hat\phi_k\|<\epsilon$, cioè anche $\hat\phi_k$ è di Cauchy, e quindi ha limite $\hat\phi$, e allora da
 \[\|\hat\phi-\frac{1}{n}\sum_{j=0}^{n-1}U^j\phi\|\leq \|\frac{1}{n}\sum_{j=0}^{n-1}U^j(\phi-\phi_k)\|+
  \|\frac{1}{n}\sum_{j=0}^{n-1}(U^j\phi_k-\hat\phi_k)\|+\|\hat\phi_k-\hat\phi\|\]
 si conclude che vale proprio $\|\hat\phi-\frac{1}{n}\sum_{j=0}^{n-1}U^j\phi\|\rightarrow 0$.
\end{proof}

\begin{defi}In un sistema dinamico misurabile $(X,\A,\mu,S)$, per $A\in\A$, $x\in X$ diciamo somma di Birkhoff ($n$-esima) \[T(x,A,n)=\sum_{j=0}^{n-1}\chi_A(S^jx)\]
inoltre saranno $\nu(x,A,n)=\frac{1}{n}T(x,A,n)$, $\bar\nu(x,A)=\limsup_{n\rightarrow\infty}\frac{1}{n}T(x,A,n)$ e rispettivamente $\underline\nu(x,A)$, e infine, se esiste, $\nu(x,A)=\lim_{n\rightarrow\infty}\frac{1}{n}T(x,A,n)$.
\end{defi}


\begin{teo}[di Birkhoff per frequenze di visita] Per $\mu$-quasi ogni $x\in X$ esiste $\nu(x,A)$.\end{teo}
\begin{proof}
 Fissiamo $\epsilon>0$ e definiamo
 \[\bar\tau(x,A,\epsilon)=\min\{n\in\N:\nu(x,A,n)\geq\bar\nu(x,A)-\epsilon\}\]
 e analogamente $\underline\tau$ per $\underline\nu$. Supponiamo dapprima che $\forall x\in X \quad \bar\tau(x,A,\epsilon)\leq M$ (con $M$ che dipenderà eventualmente da $\epsilon$). Fissiamo $n$ più grande di $M$ e definiamo ricorsivamente una sottosuccessione di $\{S^jx\}_0^{n-1}$
 \[\begin{cases}
 x_0=x \qquad \tau_0=\bar\tau(x_0,A,\epsilon)\\
 x_{k+1}=S^{\bar\tau(x_k,A,\epsilon)}x_k=S^{\tau_k}x_0 \qquad \tau_k=\sum^{k-1}\bar\tau(x_h,A,\epsilon)
 \end{cases} \]
 che quindi deve concludersi con $x_{K}$ tale che $\bar\tau_{K}<n$ e $\bar\tau_{K+1}\geq n$. Osserviamo che, poichè $\nu(x,A,\bar\tau(x,A,\epsilon))\geq\bar\nu(x,A)-\epsilon$ e $\bar\nu(x,A)=\bar\nu(Sx,A)$, si hanno le seguenti disuguaglianze
 \[\begin{split}
  T(x_k,A,\bar\tau(x_k,A,\epsilon))&=\bar\tau(x_k,A,\epsilon)\nu(x_k,A,\bar\tau(x_k,A,\epsilon)) \\
  &\geq \bar\tau(x_k,A,\epsilon) (\bar\nu(x_k,A)-\epsilon)=\bar\tau(x_k,A,\epsilon)(\bar\nu(x_0,A)-\epsilon)
 \end{split}\]
 
 \[\begin{split}
  T(x,A,n)&=\sum^{K-1}T(x_k,A,\bar\tau(x_k,A,\epsilon))+\underbrace{T(x_k,A,n-K)}_{\geq 0}   \\
	  &\geq \underbrace{\sum^{K-1} \bar\tau(x_k,A,\epsilon)}_{\bar\tau_K}(\bar\nu(x,A)-\epsilon)   \\
	  &\geq (n-M)(\bar\nu(x,A)-\epsilon)
 \end{split}\]
da cui, integrando, si ha
\[n \mu(A)=\int_X T(x,A,n) d\mu(x)\geq (n-M)\left(\int_X \bar\nu(x,A)d\mu(x)-\epsilon\right)\]
e dunque per arbitrarietà di $\epsilon$ abbiamo $\mu(A)\geq \int_X \bar\nu(x,A)d\mu(x)$. Analogamente supponendo $\underline\tau$ limitato, si ottiene l'altra disuguaglianza:
\[\int_X \underline\nu(x,A)d\mu\geq\mu(A)\geq \int_X \bar\nu(x,A)d\mu\]
per cui $\underline\nu(x,A)\geq \bar\nu(x,A)$ per $\mu$-quasi ogni $x\in X$, e quindi la tesi.

Rimuoviamo infine l'ipotesi $\bar\tau$ limitato: fissato $\epsilon>0$, scegliamo $C$ tale che
\[\mu(A_C)=\mu\{x\in X: \bar\tau(x,A,\epsilon)>C\}<\epsilon\]
poniamo quindi $\bar A=A\cup A_C$ e
\[\bar\tau'(x,A,\epsilon)=\begin{cases}\bar\tau(x,A,\epsilon) \qquad x\notin A_C\\
					1 \qquad x\in A_C
  \end{cases}\]
e definiamo analogamente a prima la successione $x_{k+1}'=S^{\bar\tau'(x_k,A,\epsilon)}x_k'$. Abbiamo allora che
\[T(x_k,\bar A,\bar\tau'(x_k,A,\epsilon))
    \begin{cases} =T(x_k',\bar A,1)=1\geq \bar\nu(x_0,A)-\epsilon \qquad x\in A_C \\
		  \begin{split}\geq T(x_k,A,\bar\tau'(x_k,A,\epsilon))&=T(x_k,A,\bar\tau(x_k,A,\epsilon))\\
			&\geq \bar\tau(x_k,A,\epsilon)(\bar\nu(x_0,A)-\epsilon) \qquad x\in A \end{split}
    \end{cases}\]
\[T(x_k,\bar A,\bar\tau'(x_k,A,\epsilon))\geq \bar\tau'(x_k,A,\epsilon)(\bar\nu(x_0,A)-\epsilon)\]
per cui concludiamo esattamente come sopra sommando sulla successione e integrando, ottenendo
\[\mu(A)+\epsilon\geq\mu(\bar A)\geq \int_X \bar\nu(x,A)-\epsilon.\]
\end{proof}


\begin{esercizio}Sia $(X,\mathcal{A},\mu,S)$ ergodico e $\mu_1$ un'altra misura di probabilità $S$-invariante.
Sono equivalenti:
\begin{lista}
\item $\mu\neq\mu_1$
\item $\mu_1$ non è assolutamente continua rispetto a $\mu$
\item esiste un $A\in\mathcal{A}$ $S$-invariante (cioè $S(A)\subseteq A$) con $\mu(A)=0$ e $\mu_1(A)\neq 0$.
\end{lista}
\end{esercizio}

\begin{soluz}Ovviamente $(3)\implica (2)$ e $(2)\implica (1)$. \\
Resta da vedere che $(1)\implica (3)$: se $\mu\neq\mu_1$ esistono $A$ misurabili con $\mu_1(A)>\mu(A)$.
Consideriamo $M:=\sup\set{\mu_1(A)-\mu(A)\mid A\in\mathcal{A}}>0$: questo $\sup$ è in realtà un $\max$
e viene realizzato dal supporto $P$ di $(\mu_1-\mu)^+$ (la parte positiva della decomposizione di Hahn di $\mu_1-\mu$). \\
Se $(\mu_1-\mu)(A)=M$ allora $A$ coincide con $P$ a meno di insiemi $\abs{\mu_1-\mu}$-trascurabili e scriviamo
$A\sim P$ per risparmiare bytes.
L'invarianza di $\mu$ e $\mu_1$ dà $(\mu_1-\mu)\pa{S^{-1}(P)}=M$, quindi $S^{-1}(P)\sim P$; analogamente $S^{-n}(P)\sim P$,
quindi $P':=\cup_{n\ge 0}S^{-n}(P)\sim P$. $P'$ è $S$-invariante e $1\ge \mu_1(P')>\mu(P')$, perciò
(ergodicità di $\mu$) $\mu(P')=0$.
\end{soluz}

\begin{oss}$(2)\implica (3)$ si può anche mostrare così: per ipotesi c'è un $N\in\mathcal{A}$ tale che $\mu(N)=0$ e $\mu_1(N)\neq 0$.
Poniamo $A:=\limsup_{k\to\infty}f^{-k}(N)=\cap_{m\ge 0}\cup_{n\ge m}f^{-n}(N)$ (l'insieme dei punti che
cascano frequentemente in $A$). $A$ è $S$-invariante e $\mu(A)=0$, $\mu_1(A)\neq 0$ grazie all'invarianza di $\mu$ e $\mu_1$.
\end{oss}

\begin{oss}Se $\mu_1\neq\mu$ è anch'essa ergodica, tornando all'ultima parte della soluzione dell'esercizio
otteniamo $\mu_1(P')=1$, quindi $\mu\perp\mu_1$.\footnote{Questo risponde a una domanda fatta da Andrea Bianchi
a lezione}
\end{oss}

\begin{defi}$(X,\mathcal{A},\mu,S)$ è \emph{unicamente ergodico} se $\mu$ è l'unica misura di probabilità
$S$-invariante su $\mathcal{A}$.
\end{defi}

Nel seguito $X$ è uno spazio metrico compatto, $\mathcal{A}=\mathcal{B}(X)$ ($\sigma$-algebra dei boreliani)
e $S:X\to X$ è continua.

\begin{teo}Se $\mu$ è unicamente ergodica, $\mu$ è ergodica e per ogni $f\in C(X,\R)$
le somme di Birkhoff $\frac{1}{n}\sum_{j=0}^{n-1}f\circ S^j$ convergono uniformemente a $\widehat{f}:=\int_X f\,d\mu$.
\end{teo}

La seconda parte della dimostrazione assomiglia molto a quella di Krylov-Bogoliubov,
perciò evitiamo di ripetere tutti i dettagli nell'utilizzare la compattezza delle misure.

\begin{proof}Mostriamo che $\mu$ è ergodica: se $A$ è un boreliano invariante (cioè $S(A)\subseteq A$)
e $0<\mu(A)<1$, allora $\pa{\frac{\uno_A}{\mu(A)}}\mu$ è una misura invariante, di probabilità e diversa da $\mu$
(invarianza: $A\subseteq S^{-1}(A)$ e $\mu(A)=\mu\pa{S^{-1}(A)}$, quindi $\mu\pa{S^{-1}(A)\setminus A}=0$;
perciò, per ogni $B$, $(\uno_A\mu)\pa{S^{-1}(B)}=\mu\pa{S^{-1}(B)\cap A}=\mu\pa{S^{-1}(B)\cap S^{-1}(A)}=\mu(B\cap A)=(\uno_A\mu)(B)$)
e questo contraddice l'ipotesi. \\
Mostriamo per assurdo anche la seconda parte, cioè supponiamo che esistano $f:X\to\R$ continua e $\epsilon>0$ tali che
$\norm{\frac{1}{n_i}\sum_{j=0}^{n_i-1}f\circ S^j-\int_X f\,d\mu}_\infty\ge\epsilon$ per qualche sottosuccessione $n_i\uparrow\infty$.
Esiste allora una successione di punti $(x_i)$ tale che
\[ \abs{\frac{1}{n_i}\sum_{j=0}^{n_i-1}f\pa{S^j(x_i)}-\int_X f\,d\mu}\ge\epsilon. \]
Poniamo $\nu_i:=\frac{1}{n_i}\sum_{j=0}^{n_i-1}\delta_{S^j(x_i)}$: $\nu_i$ è una misura di probabilità che soddisfa (per costruzione)
$\abs{\int_X f\,d\nu_i-\int_X f\,d\mu}\ge\epsilon$. \\
Per la compattezza debole* delle misure (viste come il duale di $C(X)$), a meno di un'ulteriore sottosuccessione
abbiamo $\nu_i\overset{*}{\rightharpoonup}\nu$ per qualche misura di probabilità $\nu$. \\
Ora $\abs{\int_X f\,d\nu-\int_X f\,d\mu}=\lim_{i\to\infty}\abs{\int_X f\,d\nu_i-\int_X f\,d\mu}\ge\epsilon$, da cui $\mu\neq\nu$.
Per arrivare all'assurdo basta mostrare che $\nu$ è invariante, che equivale a $\int_X g\circ S\,d\nu=\int_X g\,d\nu$. Ma
\[ \int_X (g\circ S-g)\,d\nu=\lim_{i\to\infty}\frac{1}{n_i}\sum_{j=0}^{n_i-1}\pa{g\pa{S^{j+1}(x_i)}-g\pa{S^j(x_i)}}
=\lim_{i\to\infty}\frac{1}{n_i}\pa{g\pa{S^{n_i}(x_i)}-g(x_i)}=0 \]
e abbiamo finito.
\end{proof}

\begin{oss}[facoltativa] Vale anche il viceversa del teorema appena visto: se per ogni $f\in C(X,\R)$ le somme $\frac{1}{n}\sum_{j=0}^{n-1}f\circ S^j$ convergono puntualmente a $\int_X f\,d\mu$, allora $S$ è unicamente ergodica. Infatti se $\nu$ è un'altra misura invariante, per convergenza dominata si ha
\[\int_X f\,d\nu=\int_X \frac{1}{n}\sum_{j=0}^{n-1}f\circ S^j d\nu \xrightarrow{n\rightarrow\infty}\int_X \left(\int_X f d\mu\right)d\nu=\int_X f d\mu\]
per ogni $f\in C(X,\R)$, da cui $\mu$ e $\nu$ coincidono.
\end{oss}



\begin{esercizio}Consideriamo la cifra più significativa di $2^k$, per $k=1,2,\dots$.
Nella successione che si ottiene compare più spesso il $7$ o l'$8$?
\end{esercizio}

\begin{soluz}Sia $s$ tale che $10^s\le 2^k<10^{s+1}$. $c\in\set{1,\dots,9}$ è la cifra più significativa di $2^k$
quando $c\cdot 10^s\le 2^k<(c+1)10^s$, cioè quando $s+\log_{10}(c)\le k\log_{10}(2)<s+\log_{10}(c+1)$.
Inoltre $s=\floor{k\log_{10}(2)}$, quindi ci stiamo chiedendo se è
$\set{k\log_{10}(2)}\in\pa{\log_{10}(7),\log_{10}(8)}$ oppure $\set{k\log_{10}(2)}\in\pa{\log_{10}(8),\log_{10}(9)}$
(usiamo gli intervalli aperti perché per $k>3$ gli estremi degli intervalli non verranno mai realizzati). \\
Sia $\alpha:=\log_{10}(2)$ (irrazionale) e $R_\alpha(x):=x+\alpha$ la solita rotazione
su $\mathbb{T}^1=\R/\Z$; ci stiamo chiedendo chi è maggiore tra $\nu(0,A_1)$ e $\nu(0,A_2)$, dove $A_1$
e $A_2$ sono gli intervalli scritti sopra, proiettati su $\mathbb{T}^1$ (v. Birkhoff ?? per la definizione di $\nu(\cdot)$). \\
Birkhoff non garantisce neppure che queste frequenze asintotiche esistano (il singoletto $\set{0}$ è trascurabile!),
ma possiamo applicare invece l'ultimo teorema. \\
Intanto occorre verificare che la misura di Lebesgue è l'unica misura $R_\alpha$-invariante:
se $\mu$ è una misura di probabilità tale che $\pa{R_\alpha}_*\mu=\mu$ otteniamo
\[ \widehat{\mu}(k)=\widehat{\pa{R_\alpha}_*\mu}(k)=e^{-2\pi i k\alpha}\widehat{\mu}(k), \]
da cui $\widehat{\mu}(k)=0$ se $k\neq 0$, mentre $\widehat{\mu}(0)=\mu(\mathbb{T}^1)=1$
(ricordiamo che $\widehat{\mu}(k):=\int_0^1 e^{-2\pi i kx}\,d\mu(x)$
e che vale l'unicità: se due misure hanno gli stessi coefficienti di Fourier, allora coincidono). \\
Quindi i coefficienti di Fourier di $\mu$ sono univocamente determinati, ovvero c'è una sola misura invariante,
che deve essere quella di Lebesgue. \\
Siamo nelle ipotesi dell'ultimo teorema; le funzioni $\uno_{A_1}$ e $\uno_{A_2}$ non sono continue, ma possiamo cavarcela
approssimando dal basso. Siano per esempio $\phi_k\uparrow \uno_{A_1}$ con $0\le\phi_k\le 1$ e $\phi_k\in C(\mathbb{T}^1)$.
Il teorema ci dice che
\[ \int_{\mathbb{T}^1} \phi_k\,dx=\ubar{\lim}_{T\to\infty}\frac{(S_T\phi_k)(0)}{T}
\le\ubar{\lim}_{T\to\infty}\frac{(S_T\uno_{A_1})(0)}{T}=\ubar{\nu}(0,A_1) \]
e mandando $k\to\infty$ otteniamo $\abs{A_1}\le\ubar{\nu}(0,A_1)\le\obar{\nu}(0,A_1)$.
Chiamando $J_1$ la parte interna di $\mathbb{T}^1\setminus A_1$,
analogamente vale $\abs{J_1}\le\ubar{\nu}(0,J_1)$. Sommando le due disuguaglianze arriviamo a
\[ 1=\abs{A_1}+\abs{J_1}\le\obar{\nu}(0,A_1)+\ubar{\nu}(0,J_1)\le \obar{\lim}_{T\to\infty}\nu(0,T,A_1\cup J_1)\le 1, \]
per cui devono valere tutte le uguaglianze: in particolare $\ubar{\nu}(0,A_1)=\obar{\nu}(0,A_1)=\abs{A_1}$. \\
Allo stesso modo $\nu(0,A_2)=\abs{A_2}$. Ma $\abs{A_1}<\abs{A_2}$, dunque il $7$ compare più spesso dell'$8$.
\end{soluz}

\begin{defi}$(X,\mathcal{A},\mu,f)$ è \emph{mescolante} o \emph{strongly mixing} se
per ogni $A,B\in\mathcal{A}$ abbiamo $\mu\pa{f^{-n}(A)\cap B}\to\mu(A)\mu(B)$ per $n\to\infty$.
\end{defi}

Ricordiamo che mescolante $\implica$ ergodico, ma non vale il viceversa:

\begin{esempio}La rotazione irrazionale $R_\alpha\in\homeo(\mathbb{T}^1)$ non è mescolante.
Infatti, preso un piccolo intervallino $I$ centrato in $x_0$, esistono infiniti $n>0$ tali che $f^{-n}(x_0)\approx x_0+\mz$,
quindi per questi $n$ abbiamo $f^{-n}(I)\cap I=\emptyset$, contraddicendo la definizione di mescolanza con $A,B:=I$.
\end{esempio}

Sia $U_f:L^2(X)\to L^2(X)$, $U_f(\phi):=\phi\circ f$ l'operatore
di Koopman già definito prima.

\begin{prop}$(X,\mathcal{A},\mu,f)$ è mescolante $\sse$ $\forall\phi,\psi\in L^2(X)\ \ang{U_f^n\phi,\psi}\to\ang{\phi,1}\ang{1,\psi}$.
(Scritto in questo modo, l'enunciato funziona anche per funzioni a valori complessi.)
\end{prop}

\begin{proof}$(\Leftarrow)$: ovvio restringendosi alle funzioni caratteristiche e ricordando che $U_f(\uno_A)=\uno_{f^{-1}(A)}$. \\
$(\Rightarrow)$: la proprietà vale per funzioni caratteristiche e per bilinearità anche per funzioni semplici.
Per $\phi,\psi\in L^2(X)$ generiche, fissato $\epsilon>0$ esistono $\alpha$ e $\beta$ semplici tali che
$\norm{\phi-\alpha}_2,\norm{\psi-\beta}_2<\epsilon$. Sappiamo che definitivamente
\[ \abs{\ang{U_f^n\alpha,\beta}-\ang{\alpha,1}-\ang{1,\beta}}<\epsilon \]
e scrivendo
\[ \ang{U_f^n\phi,\psi}-\ang{U_f^n\alpha,\beta}
=\ang{U_f^n\phi,\psi}-\ang{U_f^n\phi,\beta}+\ang{U_f^n\phi,\beta}-\ang{U_f^n\alpha,\beta} \]
e usando il fatto che $U_f$ è un'isometria si trova che $\abs{\ang{U_f^n\phi,\psi}-\ang{U_f^n\alpha,\beta}}<O(\epsilon)$.
Le differenze $\abs{\ang{\phi,1}-\ang{\alpha,1}}$ e $\abs{\ang{1,\phi}-\ang{1,\alpha}}$ si stimano entrambe con $\epsilon$. \\
Quindi definitivamente $\abs{\ang{U_f^n\phi,\psi}-\ang{\phi,1}-\ang{1,\psi}}<O(\epsilon)$.
\end{proof}

La quantità $c_n(\phi,\psi):=\ang{U_f^n\phi,\psi}-\ang{\phi,1}\ang{1,\psi}$ si chiama $n$-\emph{esimo coefficiente di correlazione}
tra $\phi$ e $\psi$.

\begin{esempio}Gli automorfismi lineari \emph{iperbolici} di $\mathbb{T}^2=\R^2/\Z^2$, cioè
gli elementi di $GL(2,\Z)$ con autovalori di modulo $\neq 1$, sono tutti mescolanti. \\
Sia infatti $A\in GL(2,\Z)$ una matrice iperbolica. Essendo $\abs{\det(A)}=1$, $A$ preserva la misura di Lebesgue su $\mathbb{T}^2$.
Basta ora vedere che vale l'ipotesi dell'ultima proposizione per funzioni
$\phi,\psi\in\set{e_k(x):=e^{2\pi i(k_1 x_1+k_2 x_2)}\mid k\in\Z^2}$, perché queste generano un sottospazio denso in $L^2(\mathbb{T}^2)$
e si conclude come nella dimostrazione della proposizione. \\
Dobbiamo verificare che $\ang{U_A^n e_k,e_{k'}}\to\ang{e_k,1}\ang{1,e_{k'}}$; per $k=0$ è ovvio, quindi supponiamo $k\in\Z^2\nonzero$.
Osserviamo che $U_A^n e_k=e_{(A^t)^n k}$ e che $(A^t)^n k=k'$ per al più un valore di $n$,
altrimenti avremmo $(A^t)^n k=(A^t)^{n'} k$ (con $n<n'$) $\implica$ $(A^t)^{n'-n}-I$ avrebbe nucleo non banale
$\implica$ $A^t$ avrebbe un autovalore di modulo $1$, assurdo.
Perciò definitivamente $(A^t)^n k\neq k'$ $\implica$ definitivamente $\ang{U_A^n e_k,e_{k'}}=0=\ang{e_k,1}\ang{1,e_{k'}}$.
\end{esempio}
\section{Frazioni Continue e Mappa di Gauss}

Osserviamo che un $\alpha\in\R$  \`e irrazionale se e solo se esistono tutte le iterate della mappa di Gauss $G(x)=\{\frac{1}{x}\}$ (ovvero $G^n(\alpha)$ non  \`e mai nulla). Prendiamo allora $\alpha\in\irr$ e definiamo l'$n$\emph{-esimo quoziente completo}
\[\alpha_n=G(\alpha_{n-1}) \quad\mbox{con}\quad \alpha_0=G\left(\frac{1}{\alpha}\right)\]
e l'$n$\emph{-esimo quoziente parziale}, che sar\`a sempre un intero,
\[a_n=\frac{1}{\alpha_{n-1}}-\alpha_n=\frac{1}{\alpha_{n-1}}-G(\alpha_{n-1})=\left[\frac{1}{\alpha_{n-1}}\right] \quad \mbox{con} \quad a_0=\alpha-\alpha_0\]
per cui abbiamo una scrittura di $\alpha$ detta \emph{frazione continua}:
\[\begin{split}
\alpha&=a_0+\frac{1}{a_1+\frac{1}{a_2+\frac{1}{a_3+\dots}}}\\
      &:=[a_0,a_1,a_2,a_3,\dots]\\
      &:=a_0+\frac{1}{a_1+}\frac{1}{a_2+}\frac{1}{a_3+}\cdots
\end{split}\]
Di fatto abbiamo stabilito una bigezione tra gli irrazionali e le successioni (unilatere) di interi, per molti versi analoga agli sviluppi in base. 
Possiamo far corrispondere in modo naturale (non unico) le successioni troncate a dei razionali, che diremo \emph{convergenti}:
\[\frac{p_n}{q_n}=[a_0,a_1,a_2,\dots,a_n]=a_0+\frac{1}{a_1+}\frac{1}{a_2+}\cdots\frac{1}{a_{n-1}+}\frac{1}{a_n}\]
Il problema della rappresentazione non unica di numeri razionali in frazione continua, per cui ad esempio $[2,2,3]=[2,2,2,1]$, 
si pu\`o risolvere semplicemente imponendo che il numero di termini della frazione continua sia sempre pari o sempre dispari, 
ovvero che l'ultimo denominatore sia sempre 1, o non sia mai 1. 
Per il momento non adotteremo convenzioni, essendo interessati invece a passare dalle frazioni ai razionali.

\begin{teo}[Propriet\`a dei convergenti] Per ogni $n$ valgono le seguenti
\[\begin{cases}
   p_0=a_0 \quad p_1=a_1a_0+1 \quad p_{n}=a_np_{n-1}+p_{n-2}\\
   q_0=1 \quad q_1=a_1 \quad q_{n}=a_nq_{n-1}+q_{n-2}
  \end{cases}\]
\[p_nq_{n-1}-p_{n-1}q_n=(-1)^{n-1} \qquad p_nq_{n-2}-p_{n-2}q_n=(-1)^na_n\]
\[\alpha=\frac{p_n+p_{n-1}\alpha_n}{q_n+q_{n-1}\alpha_n} \qquad \alpha_n=-\frac{q_n\alpha-p_n}{q_{n-1}\alpha-p_{n-1}}\]
\end{teo}

\begin{esempio}
 Con le propriet\`a suddette, non \`e difficile vedere che $\alpha\in\irr$ \`e quadratico (su $\Q$) se e solo se la sua frazione continua  \`e periodica. 
 Si pu\`o mostrare inoltre che se $\alpha,\alpha'\in\irr$, esistono $h,k$ tali che
 \[\alpha=[a_0,\dots,a_k,c_1,\dots,c_n,\dots] \qquad \alpha'=[a'_0,\dots,a'_h,c_1,\dots,c_n,\dots]\]
 se e solo se esiste $M\in PSL(2,\Z)$ tale che $\alpha=M\alpha'$, ovvero 
 \[\alpha=\frac{A\alpha'+B}{C\alpha'+D}\quad \mbox{con}\quad |AD-BC|=1.\]
\end{esempio}

I convergenti approssimano $\alpha$ molto bene, come vediamo con i seguenti risultati. Anzitutto definiamo
\[\beta_{-1}=1\qquad \beta_n=(-1)^n(q_n\alpha-p_n)>0 \quad \Rightarrow \quad \beta_n=\alpha_0\alpha_1\cdots \alpha_n\]

\begin{teo}[Migliore approssimazione]\label{miglioreappr}\begin{enumerate} Se $\alpha\in\irr$, valgono
  \item $|q_n\alpha-p_n|=(q_{n+1}+q_n\alpha_{n+1})^{-1}$ da cui $1/2<\beta_nq_{n+1}<1$
  \item $\beta_n\leq g^n$ e $q_n\geq G^{n-1}$, dove $G,g=\frac{\sqrt 5\pm 1}{2}$ sono i numeri d'oro
  \item se $0<q<q_{n+1}$ allora per ogni $p\in\Z$ si ha $|q\alpha-p|\geq |q_n\alpha-p_n|$, e se vale l'uguaglianza $p=p_n$ e $q=q_n$
  \item se $\left|\alpha-\frac{p}{q}\right|<\frac{1}{2q^2}$ allora $p/q$  \`e un convergente.
  \end{enumerate}\end{teo}

La prova di questo importante teorema non  \`e difficile, ma laboriosa. Le frazioni continue permettono anche di caratterizzare i numeri diofantei come segue:
\[\begin{split} CD(\tau)&=\{\alpha\in\irr:q_{n+1}=O(q_n^{1+\tau})\}\\
			&=\{\alpha\in\irr:a_{n+1}=O(q_n^{\tau})\}\\
			&=\{\alpha\in\irr:\alpha_n^{-1}=O(\beta_{n-1}^{-\tau})\}\\
			&=\{\alpha\in\irr:\beta_n^{-1}=O(\beta_{n-1}^{-1-\tau})\}
\end{split}\]
Un altro esercizio non difficile  \`e provare che convergono $\sum^\infty\frac{\ln q_k}{q_k}$ e $\sum^\infty\frac{1}{q_k}$.



\subsection{La mappa punto interrogativo di Minkowski}

Consideriamo la mappa a tenda e la mappa di Farey, definite sull'intervallo $[0,1]$:

$$T(x):=\begin{sist}[cc] 2x & 0\leq x\leq 1/2 \\ 2-2x & 1/2\leq x\leq 1 \end{sist} \qquad F(x):=\begin{sist}[cc] x/(1-x) & 0\leq x\leq 1/2 \\ (1-x)/x & 1/2\leq x\leq 1 \end{sist}$$

\begin{prop} Esiste un unico omeomorfismo $h:[0,1]\fun[] [0,1]$ tale che $F\circ h = h\circ T$.\end{prop}

\begin{proof} Dimostriamo per prima cosa l'unicit\`a. Prima di tutto, si nota  che necessariamente $h(0)=0$. 
Infatti $h(0)$ deve essere un estremo dell'intervallo e un punto fisso per la mappa $F$, dato che $0$ lo \`e per $T$. 
Dunque vale anche $h(1)=1$ e $h$ deve essere strettamente crescente.

Si verifica inoltre che: 
$$\biguni_k T^{-k}(0)=\Q_D \inter [0,1] \quad \biguni_k F^{-k}(0)=\Q \inter [0,1] \quad h\left(T^{-n}(0)\right)\subset F^{-n}(0) \text{ per ogni } n\in\N$$
dove $\Q_D$ sono i razionali diadici.

Per ogni $x\in [0,1]$, $\#T^{-1}(x)=\#F^{-1}(x)<\infty$. Perci\`o $\# T^{-n}(0)=\# F^{-n}(0)<\infty$ per ogni $n\in \N$. 
Dato che $h$ \`e strettamente crescente e $h\left(T^{-n}(0)\right)\subset F^{-n}(0)$, i valori di $h$ su $T^{-n}(0)$ sono determinati in modo unico. 
Quindi $h$ \`e determinata in modo unico su $\Q_D \inter [0,1]$ e dunque abbiamo l'unicit\`a.

\begin{oss}\begin{enumerate}
\item Si verifica facilmente che $\# T^{-n}(0)=\# F^{-n}(0)=2^{n-1}+1$. Gli elementi di $T^{-n}(0)$ sono i razionali diadici con denominatore al pi\`u $2^{n-1}$. 
\item Si verifica facilmente che $F^{-1}\left(\frac{p}{q}\right)=\set{\frac{p}{p+q},\,\frac{q}{p+q}}$. 
\item Se $a_1<\dots<a_{2^{n-1}-1}$ sono gli elementi di $F^{-n}(0)\setminus \set{0,1}$, vale che:
$$a_i=\frac{m_{i-1}+m_{i+1}}{n_{i-1}+n_{i+1}} \text{ dove } m_i,\, n_i\in \N : a_i=\frac{m_i}{n_i}$$
\item Se inoltre $b_1<\dots<b_{2^{n-1}}$ sono gli elementi di $F^{-n-1}(0)\setminus F^{-n}(0)$, allora:
$$\frac{1}{n+1}=b_1<a_1<b_2<\dots<a_{2^{n-1}-1}<b_{2^{n-1}}=\frac{n}{n+1}$$

Per esempio i primi termini sono:
$$\begin{array}{cccccccc} F^{-2}(0)\setminus F^{-1}(0):\:&&&&1/2&&&\\
F^{-3}(0)\setminus F^{-2}(0):\:&&1/3&&&&2/3&\\
F^{-4}(0)\setminus F^{-3}(0):\:&1/4&&2/5&&3/5&&3/4\end{array}$$
\end{enumerate}
\end{oss}

Dimostriamo ora l'esistenza di $h$. Cominciamo notando che la mappa $T$ si comporta bene con lo sviluppo binario. Infatti si verifica che:
$$\text{se }x=0.\eps_1\eps_2\eps_3\dots \text{ con } \eps_i\in\set{0,1}, \text{ allora } 
T(x)=\begin{sist}[lr] 0.\eps_2\eps_3\dots & \text{ se } \eps_1=0 \\ 0.\hat{\eps}_2\hat{\eps}_3\dots & \text{ se } \eps_1=1\end{sist} \; \text{ con } \hat{\eps}_i=1-\eps_i$$

Invece la mappa $F$ si comporta bene con lo sviluppo in frazioni continue. Si verifica che:
$$\text{se }y=[0; a_1,a_2, a_3, \dots], \text{ allora } 
F(y)=\begin{sist}[lcr] [0; a_2, a_3, \dots] &\text{ se } a_1=1 & (1/2<y\leq 1) 
\\ \left[0; a_1 -1, a_2, a_3, \dots \right]  &\text{ se } a_1\geq 2 & (0\leq y\leq 1/2)\end{sist}$$

Sia $Q:[0,1]\fun[] [0,1]$ la mappa:
$$y=[0;a_1,a_2,a_3,\dots] \map Q(y):=0.\underbrace{0\dots 0}_{a_1-1}\underbrace{1\dots 1}_{a_2 }\underbrace{0\dots 0}_{a_3 \text{ volte}}1\dots$$
con gli $0$ e gli $1$ che si alternano in blocchi lunghi $a_1-1, a_2, a_3, \dots$.

C\`e un piccolo ostacolo nella definizione si $Q(y)$ quando $y\in\Q$. Infatti lo sviluppo in frazioni continue non \`e unico per i numeri razionali: ci sono due possibili rappresentazioni. Tuttavia le due rappresentazioni $y$ in frazioni continue vengono mandate attraverso $Q$ nelle due diverse rappresentazioni binarie dello stesso razionale diadico. Per esempio:

$$\frac{2}{5}=[0;2,2,\infty]=[0;2,1,1,\infty] \quad Q[0;2,2,\infty]=0.011 \quad Q[0;2,1,1,\infty]=0.010\bar{1}$$

Dunque $Q$ \`e ben definita. Inoltre si verifica che $Q$ \`e biunivoca, continua e $Q\circ F=T\circ Q$.
Si definisce $h=Q^{-1}$.
\end{proof}

\begin{defi} $Q:[0,1]\fun[] [0,1]$ \`e  la mappa \emph{punto interrogativo} di Minkoski.\end{defi}


\subsection{Sistemi dinamici esatti}

L'obbiettivo di questa sezione e della prossima è dimostrare l'ergodicità della mappa di Gauss. Per farlo, torniamo a parlare di dinamica misurabile.

\begin{defi} Un sistema dinamico misurabile $(X,\mathcal{A},\mu,T)$ \`e \emph{mixing} se:
$$\int_X \left(f\circ T^k \right) \cdot g \, d\mu \longrightarrow 0 \quad\text{per ogni } f,g\in L^2_0(X)$$
dove $L^2_0(X,\mathcal{A})$ \`e l'insieme delle funzioni in $L^2$ a media nulla.

Un sistema dinamico misurabile $(X,\mathcal{A},\mu,T)$ \`e esatto se $\biginter_n T^{-n}\mathcal{A}\subset \mathcal{N}$, dove $\mathcal{N}$ \`e la $\sigma$-algebra composta dagli insiemi di misura $0$ o $1$.
\end{defi}

\begin{prop} Sia $(X,\mathcal{A},\mu,T)$ un sistema dinamico misurabile. 
$$X \text{ esatto } \imp X \text{ mixing } \imp X \text{ ergodico }$$\end{prop}
\begin{proof} Dimostriamo per prima cosa che mixing $\imp$ ergodico. Sia $A\in\mathcal{A}$ un insieme invariante. Applichiamo la definizione di mixing a $f=g=\chi_A-\mu(A)$.
$$\int_X \left(f\circ T^k \right) \cdot g \, d\mu=\int_X \left(\chi_A\circ T^k \right) \cdot \chi_A \, d\mu-\mu(A)^2=\mu(A)-\mu(A)^2\rightarrow 0$$
Dunque $\mu(A)=0$ o $\mu(A)=1$ e il sistema \`e ergodico.

Dimostriamo ora che esatto $\imp$ mixing.

Siano $f,g\in L^2_0(X,\mathcal{A})$. Sia $U:L^2(X,\mathcal{A})\fun L^2(X,\mathcal{A})$, $f\map f\circ T$. Dobbiamo mostrare che $\scal{U^k f}{g}\rightarrow 0$. Prima di tutto notiamo alcune propriet\`a di $U$:\begin{enumerate}
\item \`e un isometria, dato che $T$ preserva la misura. In particolare $U^*U=I$, $P_k:=U^k(U*)^k$ \`e autoaggiunto e idempotente;
\item $\imm U^k=L^2(X,T^{-k}\mathcal{A})$, per il criterio di misurabilit\`a di Doob;
\item $P_k$ \`e una proiezione di $L^2(X,\mathcal{A})$ su $\imm U^k=L^2(X,T^{-k}\mathcal{A})$.
\end{enumerate}

Sfruttiamo queste propriet\`a:
$$\scal{U^k f}{g}=\scal{P_k U^k f}{g}=\scal{U^k f}{P_k g}\leq \norma{f} \cdot\norma{P_k g}$$

Ci basta ora dimostrare che $\norma{P_k g}\rightarrow 0$. 

$P_k g$ converge in $L^2$. Infatti $g-P_k g=\sum_{i=0}^{k-1} P_i g - P_{i+1} g$  converge poich\'e gli addendi della somma sono ortogonali e la serie delle norme al quadrato $\sum_i \norma{P_i g - P_{i+1} g}^2\leq\norma{g}^2$ \`e finita. 

Sia $\hat{g}=\lim_k P_k g$. Abbiamo:\begin{enumerate}
\item $P_k g\in L^2_0(X,\mathcal{A})$, infatti $L^2_0(X,\mathcal{A})$ \`e lo spazio ortogonale alle costanti e $\scal{P_k g}{1}=\scal{g}{P_k 1}=\scal{g}{1}=0$. Dunque $\hat{g}\in L^2_0(X,\mathcal{A})$. 
\item $\hat{g}\in \biginter_k L^2(X,T^{-k}\mathcal{A})\subset L^2(X,\mathcal{N})$, dato che, per ogni $k$,  $P_j g$ sta definitivamente in $L^2(X,T^{-k}\mathcal{A})$.
\end{enumerate}
Mettendo insieme le cose, $\hat{g}\in L^2_0(X,\mathcal{N})$, cio\`e $\hat{g}=0$.
\end{proof}


\subsection{Esattezza della mappa di Gauss}
Consideriamo nuovamente su $[0,1]$ la mappa di Gauss $G(x)=\left\{\frac{1}{x}\right\}$. Vediamo ora un po' di cose su $G$ con l'obiettivo di mostrare che \`e una mappa esatta rispetto alla sua misura invariante $d\mu(x)=\frac{1}{\log 2 (1+x)}d\lambda(x)$.

Definiamo la seguente azione di $\mathrm{PSL}(2,\Z)$ su $\R$:
$$\begin{mice}{cc} a&b\\c&d\end{mice} \cdot x=\frac{ax+b}{cx+d}$$

Siano ora $a_1,\dots,a_n\in \N_+$. Possiamo pensarli come i primi $n$ quozienti parziali di un $\alpha\in \R\setminus \Q$. Siano $p_i/q_i:=[0;a_1,\dots,a_i]$ per ogni $i\leq n$. Siano:
$$S_n:=S_n(a_1,\dots,a_n)=\begin{mice}{cc} 0&1\\1&a_1\end{mice}\dots \begin{mice}{cc} 0&1\\1&a_n\end{mice}$$
$$\Delta_n:=\Delta_n(a_1,\dots,a_n)=\set{x\in [0,1]: \text{ i primi $n$ quozienti parziali di $x$ sono } a_1,\dots, a_n}$$
Al variare degli $a_i$ in $\N_+$, a $n$ fissato, i $\Delta_n$ partizionano quasi tutto $[0,1]$ (possono mancare infatti alcuni razionali).

Valgono i seguenti fatti: \begin{enumerate}
\item Se $\beta=[0;b_1,b_2,\dots]$, allora $S_n\cdot \beta =[0;a_1,\dots,a_n,b_1,b_2,\dots]$. Si pu\`o verificare direttamente. In particolare $S_n\left([0,1)\right)=\Delta_n$.
\item Per quanto appena detto, $G^n\circ S_n=\id$ su tutto $[0,1)$ e $S_n\circ G^n=\id$ su $\Delta_n$. Quindi $S_n=(G^n|_{\Delta_n})^{-1}$
\item $S_n=\begin{mice}{cc} p_{n-1}&p_n\\q_{n-1}&q_n\end{mice}$. Si pu\`o verificare per induzione.
\item La mappa su $[0,1]$ $x\map S_n\cdot x$ \`e decrescente se $n$ \`e dispari e crescente se $n$ \`e pari. Infatti la mappa su $[0,1]$ $x\map 1/(a+x)$ \`e decrescente (se $a>0$) e $S_n$ \`e la composizione di $n$ mappe di questo tipo.
\item $S_n \cdot 0=\frac{p_n}{q_n}$ e $S_n \cdot 1=\frac{p_n+p_{n-1}}{q_n+q_{n-1}}$. Dunque, dato che $S_n$ \`e monotona:
$$\Delta_n=\imm S_n=\begin{sist}[cc] \left[\frac{p_n}{q_n}, \frac{p_n+p_{n-1}}{q_n+q_{n-1}} \right) & \text{per $n$ pari}\\ \left(\frac{p_n+p_{n-1}}{q_n+q_{n-1}}, \frac{p_n}{q_n} \right] & \text{per $n$ dispari}\end{sist}$$

\item Si calcola che il diametro di $\Delta_n$ \`e $|\Delta_n|=\frac{1}{\abs{q_n(q_{n-1}+q_n)}}$.
\end{enumerate}

\begin{lem} Sia $B\subset [0,1]$ boreliano e sia $\Delta$ un cilindro di rango $n$ qualsiasi. Allora:

$$\frac{1}{C}\mu(B)\mu(\Delta)\leq \mu(G^{-n}(B)\inter\Delta)\leq C \mu(B)\mu(\Delta).$$

Si pu\`o prendere $C=8\log 2$.
\end{lem}

\begin{proof} Dimostriamo il lemma in tre passi: \begin{enumerate}
\item Vediamo che se $B=[x,y]$, allora 
$$\frac{1}{2}\lambda(B)\lambda(\Delta)\leq \lambda(G^{-n}(B)\inter\Delta)\leq 2 \lambda(B)\lambda(\Delta).$$

Sia $\Phi:[0,1]\fun \Delta$ tale che $\Phi=(G^n|_{\Delta})^{-1}$. Facendo un po' di conti:
$$\lambda(G^{-n}(B)\inter\Delta)=\lambda(\Phi(B))=\abs{\Phi(x)-\Phi(y)}=\abs{x-y}\cdot \abs{\Delta}\frac{q_n(q_{n-1}+q_n)}{(q_{n-1}x+q_n)(q_{n-1}y+q_n)}$$
 Consideriamo la frazione. Ponendo $x=y=1$ essa raggiunge il minimo, che \`e minorabile con $1/2$. Ponendo $x=y=0$ essa raggiunge il massimo, che \`e maggiorabile con $2$. Si hanno cos\`i le disuguaglianze.

\item Le disuguaglianze del punto (1) valgono anche per $B$ boreliano: basta approssimare $B$ con degli intervalli.

\item Si giunge alla tesi utilizzando le semplice disuguaglianze $\frac{1}{2\log 2}\lambda\leq\mu\leq\frac{1}{\log 2}\lambda$.

\end{enumerate}
\end{proof}

\begin{prop} Il sistema dinamico misurabile $([0,1],\mathcal{B},\mu,G)$ \`e esatto.\end{prop}

\begin{proof} Sia $E\in\biginter_n G^{-n}(\mathcal{B})$. Fissiamo per un momento $n$. Sia $B\in\mathcal{B}$ tale che $E=G^{-n}(B)$. Allora, usando il lemma e il fatto che $\mu$ \`e $G$-invariante, si ottiene:
$$\mu(E)\mu(\Delta)=\mu(B)\mu(\Delta)\leq C\,\mu(T^{-n}(B)\inter \Delta)=C\,\mu(E\inter \Delta)$$
per ogni $\Delta$ cilindro di rango $n$.
Abbiamo dunque che la disuguaglianza $\mu(E)\mu(\Delta)\leq C\,\mu(E\inter \Delta)$ vale per ogni $\Delta$ cilindro, di rango qualsiasi. Dato che i cilindri generano i boreliani, la disuguaglianza vale per ogni $\Delta\in\mathcal{B}$. Ponendo infine $\Delta=E^c$, si ha che $\mu(E)\mu(E^c)=0$, cio\`e $E\in\mathcal{N}$.
\end{proof}

Sia $a(x)=\left[\frac{1}{x}\right]$. Si ha che $x=G(x)+a(x)$. Inoltre $a_n(x)=a(G^{n-1}(x))$ \`e l'$n$-esimo quoziente parziale di $x$.

\begin{prop} Per quasi ogni $x\in [0,1]$, vale che: \begin{enumerate}
\item $\displaystyle\lim_n \frac{a_1(x)+\dots+a_n(x)}{n}=\infty$
\item $\displaystyle\lim_n \sqrt[n]{a_1(x)\cdot\dots\cdot a_n(x)}=K$ costante, detta costante di Khinchin.
\end{enumerate}
\end{prop}

\begin{proof}\leavevmode\begin{enumerate}
\item Intuitivamente per il teorema di Birkhoff:
$$\frac{a_1(x)+\dots+a_n(x)}{n}=\frac{1}{n}\sum_{k=0}^n a(T^k(x))\longrightarrow \int_0^1 a(x) d\mu(x)=\infty$$
dato che $a\sim \frac{1}{x}$ non \`e $L^1$. Proprio perch\'e $a\not\in L^1$, il teorema di Birkhoff non si pu\`o applicare. Il problema si risolve applicando il teorema a delle $\phi_n\uparrow a$ integrabili e usando il teorema di convergenza monotona.
\item Applichiamo il teorema di Birkhoff a $\log a(x)$:
$$\log \sqrt[n]{a_1(x)\cdot\dots\cdot a_n(x)}=\frac{1}{n}\sum_{k=0}^n \log a(T^k(x))=\int_0^1 \log a(x) d\mu(x)$$
che \`e finito poich\'e $\log a(x) \sim -\log(x)\in L^1$.

\end{enumerate}
\end{proof}

\section{Dinamica Olomorfa}

\subsection{Linearizzazione di germi olomorfi}
Indicheremo con $\C\{x\}\subset\C[[x]]$ il sottoanello delle serie convergenti, ovvero i \emph{germi olomorfi}, ovvero
\[\C\{x\}=\{ f=\sum_0^\infty f_n x^n\in \C[[x]]:\limsup_{n\rightarrow \infty}|f_n|^\frac{1}{n}<\infty \}\]
inoltre ricordiamo che se $\lambda\in\C*$ allora ogni $f=\lambda x+f_2 x^2+\dots \in \C[[x]]$ è invertibile formalmente, ovvero nel senso dell'anello $\C[[x]]$. Definiamo allora
\[G_\lambda=\{f\in\C\{x\}:f=\lambda x+f_2 x^2+\dots\},\]
\[\hat G_\lambda=\{f\in\C[[x]]:f=\lambda x+f_2 x^2+\dots\}.\]
Studieremo le dinamiche a tempo discreto su $\C*$ date dai $G_\lambda$, e il nostro scopo sarà quello di ricondursi per coniugio alla rotazione $R_\lambda(z)=\lambda z$, operazione che diremo \emph{linearizzazione}.

\begin{defi} $f\in \hat G_\lambda$ è \emph{formalmente linearizzabile} se $\exists \hat h_f\in \hat G_1$ tale che $f\circ\hat h_f=\hat h_f\circ R_\lambda$. Analogamente $f\in G_\lambda$ è \emph{linearizzabile} (analiticamente) se $\exists h_f\in G_1$ tale che $f\circ h_f= h_f\circ R_\lambda$. 
\end{defi}

\begin{teo}Se $\lambda$ è una radice $q$-esima primitiva dell'unità, allora $f\in G_\lambda$ (risp. $f\in \hat G_\lambda$) è linearizzabile (risp. formalmente linearizzabile) se e solo se $f^q=Id$.\end{teo}
\begin{proof}
 Da $R_\lambda=h_f^{-1}fh_f$ abbiamo $Id=R_\lambda^q=h_f^{-1}f^qh_f$, per cui $f^q=Id$. Per il viceversa
 \[h_f^{-1}:=\frac{1}{q}\sum_0^{q-1}\lambda^{-j}f^j\]
 \[h_f^{-1}f=\frac{1}{q}\sum_0^{q-1}\lambda^{-j}f^{j+1}=\frac{\lambda}{q}\sum_0^{q-1}\lambda^{-j-1}f^{j+1}=R_\lambda h_f^{-1}.\]
\end{proof}

Il seguente teorema completa lo studio della linearizzazione in $\hat G_\lambda$, fornendo la formula formale ricorsiva per $\hat h_f$ che sarà centrale (per il principio di identità delle serie) anche nello studio di $G_\lambda$.

\begin{teo} Se $\lambda^q\neq 1 \quad \forall q\in\N$ (non risonanza), $f\in \hat G_\lambda$ è formalmente linearizzabile.\end{teo}
\begin{proof} Esplicitando le scritture in $f\circ\hat h_f=\hat h_f\circ R_\lambda$ si ottiene
 \[\sum_{n\geq 1} \hat h_n \lambda^nz^n=\lambda\sum_1^\infty\hat h_nz^n+\sum_2^\infty z^n\sum_2^n f_j \sum_{n_1+\dots+n_j=1}\hat h_{n_1}\cdots\hat h_{n_j} \]
 in cui intenderemo sempre che $n_k\geq 1$. Per il principio di identità, otteniamo la ricorrenza cercata:
 \[\hat h_n=\frac{1}{\lambda^n-\lambda}\sum_2^n f_j\sum_{n_1+\dots+n_j=1}\hat h_{n_1}\cdots\hat h_{n_j}\]
 con $\hat h_1=1$, che ha senso per l'ipotesi di non risonanza.
\end{proof}

\begin{teo}[K\"onigs-Poincarè] Se $|\lambda|\neq 1$ allora $f\in G_\lambda$ è linearizzabile.\end{teo}
\begin{proof}
Mostreremo che in questo caso è analitica la $\hat h_f$ data dal teorema precedente.
 $f\in G_\lambda$ è analitica, quindi abbiamo le stime
 \[\exists c_1>1 \quad \exists r<1 : \quad \forall j\geq 2 \quad |f_j|\leq c_1 r^{1-j}\]
 e inoltre dalle ipotesi su $\lambda$ si ha
 \[\exists c_2>1: \quad \forall n\geq 2 \quad |\lambda^n-\lambda|^{-1}\leq c_2.\]
 Se ora definiamo ricorsivamente
 \[\begin{cases}
       \sigma_1=1\\
       \sigma_{n+1}=\sum_2^n f_j\sum_{n_1+\dots+n_j=1}\sigma_{n_1}\cdots\sigma_{n_j} 
   \end{cases}\]
 induttivamente si vede che $|\hat h_n|\leq (c_1 c_2 r^{-1})^{n-1}\sigma_n$: per cui $\hat h_f$ se e solo se converge la funzione generatrice $\sigma(z)=\sum_1^\infty \sigma_n z^n$. Poichè
 \[\begin{split}
 \sigma(z)&=z+\sum_2^\infty z^n \sum_2^n f_j\sum_{n_1+\dots+n_j=1}\sigma_{n_1}\cdots\sigma_{n_j}\\
	  &=z+\sum_2^\infty \left(\sum_1^\infty \sigma_k z^k\right)^n=z+\sum_2^\infty \sigma(z)^n\\
	  &=z+\frac{\sigma(z)^2}{1-\sigma(z)}\\
	  &\Longrightarrow \sigma(z)=\frac{1+z+\sqrt{1-6z+z^2}}{4}
 \end{split}\]
 abbiamo che $\sigma$ è analitica in un intorno di $0$, da cui la tesi. 
\end{proof}

Il teorema risolve il problema della linearizzabilità fuori dal cerchio unitario, e inoltre ci dà che per $|\lambda|<1$ l'origine è stabile secondo Lyapunov. Nel caso $|\lambda|=1$,
abbiamo già caratterizzato la linearizzabilità quando $\lambda$ è una radice dell'unità: discuteremo ora il caso in cui $\abs{\lambda}=1$ ma $\lambda$ non è radice dell'unità, arrivando a risolvere completamente il problema nell'ultima sezione. Come vedremo, per q.o. $\lambda$ (nel senso della misura di Lebesgue) tutti i germi di $G_\lambda$ si linearizzano, tuttavia esistono molti $\lambda$ per cui ciò non accade. La dimostrazione di questo fatto, non costruttiva, si basa sul noto

\begin{teo}[lemma di Baire]Se $(X,d)$ è uno spazio metrico completo e $(U_n)$ è una famiglia numerabile di aperti densi,
allora l'intersezione $\bigcap_n U_n$ è un $G_\delta$ denso (e in particolare non vuoto). Se inoltre $X$ non contiene punti isolati, l'intersezione $\bigcap_n U_n$ è più che numerabile.
\end{teo}

\begin{proof} (Seconda parte dell'enunciato.) Se per assurdo fosse $\bigcap_n U_n=\set{x_1,x_2,\dots}$ (con finiti o numerabili $x_i$),
posto $V_i:=X\setminus\set{x_i}$ (aperto denso per ipotesi) avremmo
\[ \emptyset=\pa{\bigcap_n U_n}\cap\pa{\bigcap_i V_i}. \]
Ma a secondo membro c'è ancora un'intersezione numerabile di aperti densi, assurdo.
\end{proof}

\begin{teo}[Controesempi di Cremer] Esiste un $G_\delta$ denso di valori di $\lambda\in\T^1$ tali che esiste $f\in G_\lambda$ non linearizzabile. Dunque, per Baire, tale $G_\delta$ non contiene solo radici dell'unità.\end{teo}


\begin{proof}[Primo controesempio.]
Ci sono più che numerabili $\lambda$ tali che $Q_\lambda(z):=\lambda z+z^2$ non è linearizzabile.
Infatti una condizione necessaria per la linearizzabilità è che l'origine sia lontana dall'unione di tutti
i punti periodici non nulli (perché $z\mapsto \lambda z$ non ne ha), quindi per avere $\lambda$ come richiesto basta garantire che ci siano punti periodici arbitrariamente piccoli. L'iterata $n$-esima è della forma $Q_\lambda^n(z)=\lambda^n z+\dots+z^{2^n}$, quindi i punti periodici non nulli di periodo (che divide) $n$ sono le radici di
\[ \frac{Q_\lambda^n(z)-z}{z}=(\lambda^n-1)+\dots+z^{2^n-1}. \]
Il termine noto $\lambda^n-1$ è il prodotto delle $2^n-1$ radici, quindi ce n'è una di modulo $\le\abs{\lambda^n-1}^{\frac{1}{2^n-1}}$.
Quindi basta garantire che l'$\inf$ al variare di $n>0$ dell'ultima espressione sia $0$. \\
L'insieme dei $\lambda$ che soddisfano questa richiesta è
\[ S:=\bigcap_{N>0}\set{\lambda\in\mathbb{T}^1:\inf_n\abs{\lambda^n-1}^{\frac{1}{2^n-1}}<\frac{1}{N}}. \]
Abbiamo espresso l'insieme voluto come intersezione numerabile di aperti; questi aperti sono densi perché ognuno di essi contiene tutte le radici dell'unità, quindi $S$ è un $G_\delta$ denso e più che numerabile. Le radici dell'unità sono numerabili, quindi rimuovendole rimangono più che numerabili $\lambda\in S$ irrazionali.
\end{proof} 

\begin{proof}[Secondo controesempio.]
In termini di $\alpha$, l'insieme $S$ del primo metodo corrispondeva alla richiesta
$\obar{\lim}\pa{-\frac{1}{2^n}\ln\norm{n\alpha}}=\infty$. Indeboliamola chiedendo solo
\[ \obar{\lim}\pa{-\frac{1}{n}\ln\norm{n\alpha}}=\infty \]
e cerchiamo germi $f(z)=\lambda z+f_2 z^2+f_3 z^3+\dots$ non linearizzabili (fatto questo, l'insieme dei $\lambda$ cercati
sarà più che numerabile in quanto sovrainsieme di $S$).
Ricordiamo la formula dei coefficienti del ``linearizzatore'' formale $h(z)$:
\[ \begin{split} h_n&=\frac{1}{\lambda^n-\lambda}\sum_{j=2}^n f_j\sum_{\substack{n_1+\dots+n_j=n,\\n_i\ge 1}}h_{n_1}\cdots h_{n_j} \\
&=\frac{1}{\lambda^n-\lambda}\pa{f_n+\sum_{j=2}^{n-1}f_j\sum_{\substack{n_1+\dots+n_j=n,\\n_i\ge 1}}h_{n_1}\cdots h_{n_j}} \end{split} \]
(per $n\ge 2$); scegliamo ricorsivamente $f_n$ di modulo $1$ e con lo stesso argomento del risultato della somma.
In questo modo $f(z)$ converge e d'altro canto
\[ \abs{h_n}\ge\frac{1}{\abs{\lambda^n-\lambda}}=\frac{1}{\abs{\lambda^{n-1}-\lambda}}, \]
quindi $h(z)$ diverge perché
\[ \sqrt[n]{\abs{h_{n+1}}}\ge\abs{\lambda^n-1}^{-\frac{1}{n}}=\exp\pa{-\frac{\ln\pa{2\abs{\sin(\pi n\alpha)}}}{n}}\to\infty \]
lungo una sottosuccessione (abbiamo usato l'identità $\abs{\lambda^n-1}=\sqrt{2-2\cos\pa{2\pi n\alpha}}=2\abs{\sin(\pi n\alpha)}$).
\end{proof}

Enunciamo ora i risultati che risolvono il problema della linearizzabilità.

\begin{teo}[Siegel] Se $\alpha$ è irrazionale e diofanteo, ogni germe in $G_\lambda$ è linearizzabile analiticamente.\end{teo}

\begin{teo}[Brjuno] Detti $\frac{p_n}{q_n}$ i convergenti di $\alpha$, se $\alpha$ è irrazionale
e $\sum\frac{\ln q_{n+1}}{q_n}<\infty$ allora ogni germe in $G_\lambda$ è linearizzabile analiticamente.
\end{teo}

\begin{teo}[Yoccoz] Se i convergenti di $\alpha$ non soddisfano $\sum\frac{\ln q_{n+1}}{q_n}<\infty$, allora $f(z):=\lambda z+z^2$ non è analiticamente linearizzabile.\end{teo}

I numeri i cui convergenti soddisfano la condizione di Brjuno sono appunto detti \emph{numeri di Brjuno}. Non è difficile vedere che contengono i numeri diofantei (cioè il teorema di Brjuno generalizza quello di Siegel) e quindi hanno misura di Lebesgue $1$, come già annunciato. Tornando al controesempio di Cremer, osserviamo che è uno dei casi in cui un insieme è generico nel senso della topologia (cioè è un $G_\delta$ denso) ma è l'esatto opposto nel senso della misura (è trascurabile). 

Facciamo qualche conto per apprezzare la difficoltà del teorema di Siegel. Ricordiamo la formula per i coefficienti di $h$:
\[ h_1=1 \qquad h_n=\frac{1}{\lambda^n-\lambda}\sum_{j=2}^n f_j\sum_{\substack{n_1+\dots+n_j=n,\\n_i\ge 1}}h_{n_1}\cdots h_{n_j} \]
Con un po' di fatica si ottiene 

\[h_2=\frac{f_2}{\lambda^2-\lambda}\]
\[h_3=\frac{1}{\lambda^3-\lambda}\pa{\frac{2f_2^2}{\lambda^2-\lambda}+f_3}\]
\[\begin{split}
   h_4 &=\frac{1}{\lambda^4-\lambda}\pa{f_4+3f_3h_2+2f_2h_3+f_2h_2^2} \\
       &=\frac{1}{\lambda^4-\lambda}\pa{f_4+\frac{3f_3f_2}{\lambda^2-\lambda}+\frac{4f_2^3}{(\lambda^3-\lambda)(\lambda^2-\lambda)} +\frac{2f_2f_3}{\lambda^3-\lambda}+\frac{f_2^3}{(\lambda^2-\lambda)^2}}
  \end{split}\]

In generale in $h_n$ compare il termine $\frac{2^{n-2}f_2^{n-1}}{(\lambda^n-\lambda)\cdots(\lambda^2-\lambda)}$,
quindi una stima brutale dei moduli di tutti i termini di $h_n$ difficilmente funzionerebbe: ad esempio stimando
$\abs{\lambda^n-\lambda}\sim\norm{(n-1)\alpha}\ge\gamma(n-1)^{-(1+\tau)}$ otteniamo
\[ \abs{\frac{2^{n-2}f_2^{n-1}}{(\lambda^n-\lambda)\cdots(\lambda^2-\lambda)}}\le C^n (n-1)!^{1+\tau} \]
che è troppo poco per garantire la convergenza di $h$.

Prima di passare alla dimostrazione del teorema di Brjuno (non mostreremo invece quello di Yoccoz), ambientiamo il problema in un contesto un poco più generale, per motivare ulteriormente lo sforzo e legare il problema della stabilità topologica a quello della linearizzabilità anche nel caso $|\lambda|=1$.

\subsection{Stabilità e linearizzabilità}

Ricordiamo anzitutto alcune nozioni sulla sfera di Riemann $\obar{\C}$. 
\begin{itemize}
 \item Dato $U\subseteq\obar{\C}$ aperto, sia $\mathcal{F}_U$ l'insieme delle funzioni meromorfe su $U$
      (cioè l'insieme delle funzioni olomorfe $U\to\obar{\C}$, compresa la funzione costante $\infty$).
      Una famiglia $\mathcal{F}\subseteq\mathcal{F}_U$ è \emph{normale} se ogni successione di funzioni in $\mathcal{F}$
      ha una sottosuccessione convergente uniformemente sui compatti.
 \item Un noto teorema di Weierstrass garantisce che se una successione di funzioni
      in $\mathcal{F}_U$ converge uniformemente sui compatti il suo limite è ancora in $\mathcal{F}_U$.
 \item $\obar{\C}$ è uno spazio compatto e metrizzabile (è omeomorfo a $S^2$);
      possiamo definire esplicitamente una metrica riemanniana tramite le due carte standard $z:\C\to\C$
      e $w:\obar{\C}\nonzero\to\C$ ($w=\frac{1}{z}$), ponendo
      \[ ds=\frac{2\abs{dz}}{1+\abs{z}^2}=\frac{2\abs{dw}}{1+\abs{w}^2}, \]
      che equivale a dire che la lunghezza di una curva $\gamma:[0,T]\to\obar{\C}$ è
      (prendendo come esempio il caso in cui l'immagine di $\gamma$ sia inclusa in $\obar{\C}\nonzero$)
      \[ \ell(\gamma)=\int_0^T \frac{2\abs{(w\circ\gamma)'(t)}}{1+\abs{(w\circ\gamma)'(t)}^2}\,dt. \]
      Questa metrica coincide con quella di $S^2$ letta tramite le proiezioni stereografiche.
 \item $\aut(\obar{\C})=\set{\frac{az+b}{cz+d}\mid ad-bc\neq 0}$
 \item $\ndom(\obar{\C})=\set{\frac{p(z)}{q(z)}\mid p(z),q(z)\in\C[z]}\cup\set{\infty}$
\end{itemize}

Studiamo gli endomorfismi: poniamo $R(z):=\frac{p(z)}{q(z)}$ e definiamo due insiemi complementari.

\begin{defi}L'\emph{insieme di Fatou} di $R$, che indichiamo con $F(R)$, è
l'insieme dei punti $z\in\obar{\C}$ tali che $\set{R^n\mid n\ge 0}$ è normale in un intorno di $z$
(qui $R^n$ è l'iterata $n$-esima). L'\emph{insieme di Julia} di $R$ è $J(R):=\obar{\C}\setminus F(R)$.
\end{defi}

\begin{oss}L'insieme di Fatou è aperto per definizione, mentre Julia è chiuso (e quindi compatto).
\end{oss}

\begin{esempio}Sia $R(z):=z^2$. In questo caso $F(R)=\obar{\C}\setminus S^1$, mentre $J(R)=S^1$
(infatti, se $z\in S^1$ e $U$ è un suo intorno, ogni elemento di $\C\nonzero$ sta definitivamente in $R^n(U)$:
questo si vede bene supponendo che $U$ sia della forma $\set{u:r<\abs{u}<R,\ \alpha<\arg(u)<\beta}$). \\
In generale, se $R(z)$ è un polinomio di grado $\ge 2$, $\infty\in F(R)$.
\end{esempio}

Nel caso di un polinomio $P(z)$ di grado $\ge 2$, il complementare della componente connessa di $F(P)$ contenente $\infty$
viene chiamato \emph{insieme di Julia riempito} e si indica con $K(P)$ (nel caso di $P(z)=z^2$, $K(P)=\obar{\disco}$).

\begin{prop}Se il grado di $R$ (come funzione razionale) è $d\ge 2$, $J(R)\neq\emptyset$.
\end{prop}

\begin{proof}Il grado di $R$ come funzione razionale coincide con il grado di $R$ come funzione liscia da $\obar{\C}$ in sé.
Quindi l'iterata $R^n$ ha grado $d^n$. Se per assurdo fosse $F(R)=\obar{\C}$, per la compattezza di $\obar{\C}$
una sottosuccessione $R^{n_i}$ convergerebbe uniformemente su tutto $\obar{\C}$; ma allora il grado
di $R^{n_i}$ sarebbe definitivamente costante, assurdo.
\end{proof}

\begin{defi}Dati $(X,d)$ spazio metrico e $f:X\to X$, $x_0$ è \emph{stabile}
se per ogni $\epsilon>0$ esiste un intorno $V$ di $x_0$ tale che $\forall x\in V\ \forall n\ge 0$
$\pa{f^n(x_0),f^n(x)}\le\epsilon$ (cioè le iterate $f^n$ sono equicontinue in $x_0$).
\end{defi}

Consideriamo su $\obar{\C}$ la distanza $d$ definita prima tramite la metrica riemanniana
(ma una qualsiasi distanza che induca la topologia di $\obar{\C}$ va bene).

\begin{prop}$z_0\in\obar{\C}$ è \emph{stabile} per $R$ $\sse$ $z_0\in F(R)$.
\end{prop}

\begin{proof}$(\Rightarrow)$: data una successione $(R^{n_i})$, a meno di sottosuccessioni possiamo supporre che
$R^{n_i}(z_0)$ converga a un punto $p\in\obar{\C}$ (compattezza della sfera). Esiste un automorfismo $h\in\aut(\obar{\C})$
con $h(p)=0$ e per stabilità (e la continuità uniforme di $h$) esiste un intorno $V$ di $z_0$ tale che,
per ogni $z\in V$ e ogni $i$ grande, $h\circ R^{n_i}(z)\in\C$ e anzi $\abs{h\circ R^{n_i}(z)}\le 1$. \\
Un risultato noto di analisi complessa (che è una conseguenza di Ascoli-Arzelà) afferma che una successione di funzioni olomorfe
(da un aperto di $\obar{\C}$ a $\C$) e limitate ammette una sottosuccessione convergente sui compatti.
Quindi (a meno di ulteriori sottosuccessioni) $h\circ R^{n_i}$ converge sui compatti di $V$ e così anche $R^{n_i}$. \\
$(\Leftarrow)$: se per assurdo $z_0$ non fosse stabile, esisterebbero due successioni $n_k$ e $z_k$ tali che
$d(z_0,z_k)<\frac{1}{k}$ e $d\pa{R^{n_k}(z_0),R^{n_k}(z_k)}\ge\epsilon$.
Essendo $z_0\in F(R)$, troveremmo una sottosuccessione convergente uniformemente su un intorno compatto $\obar{V}$ di $z_0$;
perciò le funzioni che la compongono sarebbero equicontinue su $\obar{V}$, assurdo.
\end{proof}

Diamo ora per buoni tre risultati; il primo è un lemma topologico, mentre gli altri vengono dall'analisi complessa.

\begin{lemma}Sia $W\subseteq\C$ un aperto connesso. $W$ è semplicemente connesso se e solo se,
per ogni curva semplice chiusa $\gamma\subset W$, la componente connessa limitata di $\C\setminus\gamma$ è inclusa in $W$.
\end{lemma}

\begin{lemma}[Schwarz]Data $f:\disco_r\to\disco_r$ olomorfa con $f(0)=0$, abbiamo $\abs{f(z)}\le\abs{z}$
e $\abs{f'(0)}\le 1$. Se vale una delle due uguaglianze (cioè $\abs{f(z')}=\abs{z'}$ per qualche $z'$ o $\abs{f'(0)}=1$),
allora $f(z)=\lambda z$ con $\abs{\lambda}=1$.
\end{lemma}

\begin{teo}[Riemann]Ogni aperto di $\C$ semplicemente connesso (eccetto tutto $\C$) è biolomorfo al disco.
Più precisamente, per ogni $z_0\in W$ esistono unici $r>0$ e $h:\disco_r\to W$ con $h$ biolomorfismo
che soddisfa $h(0)=z_0$ e $h'(0)=1$.
\end{teo}

Usiamo questi strumenti per dedurre un'altra caratterizzazione della linearizzabilità:

\begin{teo}Sia $\abs{\lambda}\le 1$. Dato $f\in G_\lambda$, $f$ è stabile in $0$ se e solo se è linearizzabile.
\end{teo}

La stabilità significa che, per ogni $V$ intorno di $0$, esiste un intorno più piccolo $U$ tale che
le iterate $f^n$ sono tutte definite su $U$ e $f^n(U)\subseteq V$ per ogni $n$.

\begin{proof}Abbiamo già visto che per $\abs{\lambda}<1$ sono vere entrambe le condizioni,
quindi supponiamo $\abs{\lambda}=1$. \\
Se $f$ è linearizzabile, chiaramente è anche stabile. Viceversa, se $f$ è stabile, esiste un disco $D$
centrato nell'origine su cui sono definite tutte le iterate $f^n$. \\
Per la stabilità c'è anche un intorno $V$ di $0$ tale che $f^n(V)\subseteq D$ per ogni $n\ge 0$.
Poniamo $L:=\bigcap_{n\ge 0}f^{-n}(D)$: $L$ è a sua volta un intorno di $0$ perché $L\supseteq V$. \\
Inoltre $f(L)\subseteq L$, quindi (essendo $f$ aperta) $f\pa{\interna{L}}\subseteq\interna{L}$.
Chiamando $W$ la componente connessa di $\interna{L}$ che contiene $0$, vale pure $f(W)\subseteq W$. \\
Basta dimostrare che $W$ è semplicemente connesso: fatto questo, il teorema di Riemann ci dà
il biolomorfismo $h:\disco_r\to W$ con $h(0)=0$ e $h'(0)=0$; la composizione $g:=h^{-1}\circ f\circ h:\disco_r\to\disco_r$
soddisfa $g(0)=0$ e $\abs{g'(0)}=\abs{\lambda}=1$, quindi per il lemma di Schwarz $g(z)=\lambda z$:
abbiamo linearizzato $f$! \\
Applichiamo il lemma topologico: sia $\gamma\subset W$ una curva semplice chiusa. Poiché $W\subseteq D$,
chiamando $B$ la componente limitata di $\C\setminus\gamma$, è $\obar{B}\subseteq D$.
Ma allora, per ogni $n\ge 0$, $f^n$ è definita su $\obar{B}$ e per il principio del massimo
il modulo di $f^n$ ha massimo su $\de\obar{B}=\gamma$. Per definizione di $L$, $f^n(\gamma)\subseteq D$;
dunque $\max_{z\in\obar{B}}\abs{f^n(z)}<1$, cioè $f^n(\obar{B})\subseteq D$.
Questo dà $\obar{B}\subseteq\interna{L}$; essendo $\obar{B}$ connesso e $\de\obar{B}\subset W$, arriviamo a $\obar{B}\subseteq W$.
\end{proof}

\subsection{Il teorema di Siegel-Brjuno}

Come annunciato, dimostriamo in questa sezione il teorema di Brjuno. Seguiremo da vicino l'articolo originale, abbandonando le notazioni usate fino a questo punto in favore di quelle di Brjuno. Iniziamo con l'enunciato.

\begin{teo}[Siegel-Brjuno] Sia $\lambda\in\irr$ con convergenti $p_k/q_k$ tali che 
   \[\sum_o^\infty \frac{\ln q_{k+1}}{q_k}<\infty,\]
  e sia $\Lambda=e^{2\pi i\lambda}$; allora la dinamica data da
   \[F:x\mapsto \Lambda x + f_1 x^2+f_2 x^3+\dots=\Lambda x+xf(x)\]
  sulla sfera di Riemann, con $f$ olomorfa in $0$ e $f(0)=0$, è coniugata alla dinamica $R_\Lambda:y\mapsto \Lambda y$ tramite
   \[H:y\mapsto y+h_1 y^2+h_2 y^3+\dots =y(1+h(y))=x\]
  con $h$ olomorfa e nulla in $0$.
\end{teo}

L'idea è sempre quella di esplicitare ricorsivamente i coefficienti $h_n$ e verificare che la serie di potenze relativa converge, tramite stime opportune, che si ricondurranno all'ipotesi sui convergenti. Imponendo il coniugio $F\circ H=H\circ R_\Lambda$ si ottiene l'equazione (\emph{di Schr\"oder})
\[\Lambda h(\Lambda y)=\Lambda h(y) +(1+h(y))f(y+yh(y))\]
da cui, indicando con $[\cdot]_n$ il coefficiente $n$-esimo di una serie, si ha
\[h_n=\underbrace{\frac{1}{\Lambda^{n+1}-\Lambda}}_{(1)}\underbrace{\left[(1+h(y))f(y+yh(y))\right]_n}_{(2)}.\]
Controlliamo il fattore risonante $(1)$ con 
\[\omega(n)=\min_{m\in \Z} |n\lambda-m| \quad \Rightarrow \quad |\Lambda^n-1|\geq 4\omega(n)\]
mentre per il termine ricorsivo $(2)$ usiamo che per opportune costanti $c_1,c_2$ i coefficienti dello sviluppo in serie di $\frac{c_1 x}{c_2-x}$ maggiorano quelli dello sviluppo di una funzione analitica, nel nostro caso $f$, ottenendo in definitiva
\[|h_n|\leq \frac{1}{4\omega(n)} \left[\frac{c_1 y (1+h(y)^2)}{c_2-y(1+h(y))}\right]_n.\]
A questo punto definiamo ricorsivamente, posto $g=\sum_1^\infty g_n x^n$,
\[\omega(n) g_n=\frac{1}{4} \left[\frac{c_1 y (1+g(y)^2)}{c_2-y(1+g(y))}\right]_n\geq 0;\]
si verifica per induzione che $g_n\geq|h_n|$, per cui siamo ricondotti alla convergenza di $g$. Sommando su $n$ l'ultima equazione si ottiene
\[\sum_1^\infty \omega(n) g_n y^n=\frac{c_1 y (1+g(y)^2)}{4(c_2-y(1+g(y)))} \]
da cui, usando che $\omega(n)<1/2$, con semplici stime si ottiene
\[g_n<\frac{C}{\omega(n)}\underset{n_1+n_2+1=n}{\sum_{n_1,n_2\geq 0}}g_{n_1}g_{n_2}\]
dove $C$ è una costante opportuna. Ancora una volta dobbiamo stimare un termine risonante e uno ricorsivo, e lo facciamo rispettivamente con le successioni
\begin{eqnarray*}
  \sigma_0=1 \qquad
  \sigma_n=C \underset{n_1+n_2+1=n}{\sum_{n_1,n_2\geq 0}}\sigma_{n_1}\sigma_{n_2}
 \\
  \delta_0=1 \qquad
  \delta_n=\frac{1}{\omega(n)}\underset{n_1+n_2+1=n}{\max_{n_1,n_2\geq 0}}\delta_{n_1}\delta_{n_2}
\end{eqnarray*}
in modo che $g_n\leq \sigma_n \delta_n$. Siamo ricondotti allo studio della convergenza delle relative serie di potenze.
Per la prima il calcolo è analogo a quello fatto per K\"onigs-Poincarè: 
\[\sigma(y)-1=\sum_1^\infty \sigma_n y^n=Cy\sigma(y)^2 \quad \Rightarrow \quad \sigma(y)=\frac{1-\sqrt{1-4Cy}}{2Cy}\]
per cui $\sigma$ è analitica in $0$.
La difficoltà sta tutta nel fattore risonante: come avevamo visto in precedenza nel caso diofanteo, non funzionano stime semplici, per cui dobbiamo raffinare il conteggio dei piccoli divisori. Osserviamo infatti che dalla definizione dei $\delta_n$ si ha che ognuno si scrive come prodotto di fattori $1/\omega(j)$ (con molteplicità): basta infatti nella definizione considerare $n_1$ e $n_2$ che realizzano il minimo e iterare il ragionamento. Siamo interessati a stimare quanti di questi fattori sono ``grandi''.\newline\indent
Iniziamo definendo $\omega_k=\omega(q_k)$ i minimi successivi di $\omega:\N\rightarrow\R$, ovvero mettiamo $\omega_0=\omega(1)$ e induttivamente $q_{k+1}$ il primo numero dopo $q_k$ tale che $\omega_{k+1}=\omega(q_{k+1})$ è minimo di $\omega|_{0,\dots,q_{k+1}}$ ovvero di $\omega|_{q_k,\dots,q_{k+1}}$. La notazione $q_k$ è suggestiva del fatto che tali numeri risulteranno essere i denominatori dei convergenti di $\lambda$. Eventualmente useremo $\omega_{-1}=1$. \newline
Suddividiamo ora la retta in ``scatole'' successive $[\omega_{k+1}/2,\omega_k/2)$: risulteranno essere di larghezza (più che) esponenzialmente decrescente. Anzitutto costruiamo una funzione che controlla \emph{quali} piccoli divisori stanno in una ``scatola'': per $n\in\N^+$ e $k=-1,0,1,\dots$ poniamo
\[\psi_k(n)=\begin{cases}
   1 \qquad \omega(n)<\omega_k/2\\
   0 \qquad \omega(n)\geq \omega_k/2
  \end{cases}\]
\begin{lemma} Se $\psi_k(n)=1$, allora $\psi_k(n-l)=0$ per tutti gli $0<l<q_{k+1}$. In altre parole, nelle scatole oltre la $k$-esima si cade al più ogni $q_{k+1}$ numeri.\end{lemma}
\begin{proof}
 Indichiamo con $m_n$ l'intero che realizza il minimo nella definizione di $\omega$;
 \[\begin{split}
  \omega(n)+\omega(n-l) &=|n\lambda-m_n|+|(n-l)\lambda-m_{n-l}|\\
			&\geq |l\lambda+m_{n-l}-m_n|\geq \omega(l)
\end{split}\]
e ora però $\omega(l)\geq \omega_k$ e $\omega(n)<\omega_k/2$, per cui
\[\omega(n-l)\geq \omega(l)-\omega(n)>\omega_k/2.\] 
\end{proof}

Definiamo ora la funzione $\phi_k(n)$ come il numero di fattori $1/\omega(j)$ nella decomposizione di $\delta_n$ tali che $\omega_{k+1}/2\leq\omega(j)<\omega_k/2$. In altri termini, $\phi$ conta \emph{quanti} piccoli divisori stanno in una ``scatola''. Dalla definizione ricorsiva di $\delta_n$ si ha che per $n\in\N^+$ e $k=-1,0,1,\dots$
\[0\leq \phi_k(n)\leq\psi_k(n)+\underset{n_1+n_2+1=n}{\max_{n_1,n_2\geq 0}}(\phi_k(n_1)+\phi_k(n_2)).\tag{*}\]
Nel prossimo risultato ci basterà quest'ultima disuguaglianza, e non ricorreremo alla definizione di $\phi_k$, per cui lo enunciamo nel modo seguente:

\begin{lemma} Se $\phi:\N\rightarrow\N$ soddisfa la \emph{(*)}, per $n\in\N^+$ e $k=-1,0,1,\dots$
 \[\phi_k(n)\begin{cases} =0 \qquad 0<n<q_{k+1}\\
    \leq 2\left\lfloor\frac{n}{q_{k+1}}\right\rfloor-1 \qquad n\geq q_{k+1}
   \end{cases}\]
   e dunque $\phi_k(n)<\frac{2n}{q_{k+1}}$.
   \end{lemma}
 
\begin{proof}
 Se $n<q_{k+1}$, $\psi_k(n)=0$, da (*) la tesi. Se $n=q_{k+1}$, devono essere $n_i<n$ e quindi $\phi(n_i)=0$, inoltre $\psi(n)\leq 1$ e perciò ancora per (*) vale la tesi. \newline
 Per il caso $n>q_{k+1}$ procediamo per induzione, e assumiamo la tesi fino a $n$. Siano $n_1\geq n_2$ che realizzano l'uguaglianza in (*). Se $\psi(n)=0$ si conclude facilmente studiando i tre casi $q_{k+1}\leq n_2$, $n_2<q_{k+1}\leq n_1$ e $n_1<q_{k+1}$. Se invece $\psi(n)=1$, come prima sono facili i casi laterali $q_{k+1}\leq n_2$ e $n_1<q_{k+1}$, come anche il caso $n_2<q_{k+1}\leq n_1$ se vale anche $\left\lfloor\frac{n_1}{q_{k+1}} \right\rfloor<\left\lfloor\frac{n}{q_{k+1}} \right\rfloor$. Se invece $\left\lfloor\frac{n_1}{q_{k+1}} \right\rfloor=\left\lfloor\frac{n}{q_{k+1}} \right\rfloor$, allora dev'essere $n-n_1<q_{k+1}$, e quindi $\psi_k(n_1)=0$, per cui applicando la (*) a $\phi(n_1)$ otteniamo, poichè era $\phi(n_2)=0$,
 \[\phi\leq 1+\phi(n_3)+\phi(n_4)\]
 con $n_3+n_4+1=n_1$. Ripetendo lo studio dei tre casi per i nuovi $n_i$, si giunge ancora alla situazione $n_4<q_{k+1}\leq n_3$ e $\left\lfloor\frac{n_3}{q_{k+1}} \right\rfloor=\left\lfloor\frac{n_1}{q_{k+1}} \right\rfloor=\left\lfloor\frac{n}{q_{k+1}} \right\rfloor$, da cui si procede in modo analogo, giungendo dopo $l$ passi a
 \begin{eqnarray*}
  \phi(n)\leq 1+\phi(n_{2l-1})+\phi(n_{2l})\\
  n_{2l-1}+n_{2l}=n_{2l-3}<n_{2l-5}<\dots<n_1<n\\
  q>n_{2l} \qquad \left\lfloor\frac{n_{2l-1}}{q_{k+1}} \right\rfloor=\left\lfloor\frac{n}{q_{k+1}} \right\rfloor.
 \end{eqnarray*}
Iterando la $n_{2l+1}\leq n_{2l-1}-1$ si ottiene che $n-n_{2l-1}\geq l$, e tuttavia l'ultima equazione elencata può verificarsi solo fichè $n-n_{2l-1}\leq q_{k+1}-1$, per cui il caso sfortunato non può presentarsi per più di $q_{k+1}-1$ volte di fila. A questo punto, è facile condludere l'induzione. 
\end{proof}
 
Abbiamo ora tutti gli strumenti per stimare $\delta_n$: sia $K$ tale che $q_K\leq n<q_{K+1}$ ($n$ sta nella $K$-esima ``scatola''), allora per quanto visto
\begin{eqnarray*}
 \delta_n\leq\prod_{k=-1}^K \left(\frac{2}{\omega_{k+1}}\right)^{\phi_k(n)}\\
 \ln\delta_n\leq \sum_{k=-1}^K \frac{2n}{q_{k+1}}\ln\left(\frac{2}{\omega_{k+1}}\right)<2n\sum_{k=0}^\infty \frac{1}{q_k}\ln\left(\frac{2}{\omega_k}\right).
\end{eqnarray*}

A questo punto, come anticipato, osserviamo che i nostri $q_k$ e i relativi interi $p_k$ che realizzano la definizione di $\omega(k)$, 
sono buone approssimazioni di $\lambda$ nel senso precisato nella parte sulle Frazioni Continue, e dunque, per il teorema di buona approssimazione,
coincidono con i convergenti della frazione continua di $\lambda$. 
Allora, ricordando che $q_k>(\sqrt{2})^{k-1}$ e che $\omega_k=|q_k\lambda-p_k|>\frac{1}{2q_{k+1}}$ (cfr. ancora il capitolo suddetto), dalla formula sopra abbiamo
\[\ln\delta_n <2n\left(\ln(4)\sum_{k=0}^\infty \frac{1}{q_k}+\sum_{k=0}^\infty \frac{\ln q_{k+1}}{q_k} \right)\]
in cui la prima serie converge (esercizio visto in precedenza) e la seconda è quella dell'ipotesi del teorema. 
Per cui abbiamo finalmente concluso la nostra dimostrazione, avendo provato che per una costante $C$ vale $\delta_n<C^{2n}$.
\section{Flussi e Equazione Coomologica}

Consideriamo il toro $d$-dimensionale $\T^d=\R^d/\Z^d$ per $d\geq 2$, $\alpha\in\R^d\smallsetminus \{0\}$. In $\T^d$ abbiamo la dinamica continua data dal flusso integrale 
\[S^t_\alpha x=x+t\alpha \qquad \mbox{dell'equazione} \qquad \dot x=\alpha\]
in $\R^d$ al quoziente sul toro, e la dinamica discreta ad essa strettamente collegata data dalla traslazione 
$S_\alpha: x\mapsto x+\alpha$ in $\R^d$ al quoziente sul toro (in entrambi i casi parleremo di traslazioni sul toro). 
Osserviamo che la misura di Haar è invariante per entrambe, e se non meglio specificato ci riferiremo a quella.

\begin{defi} $\alpha=(\alpha_1,\dots,\alpha_d)$ è \emph{non risonante} se i reali $1,\alpha_1,\dots,\alpha_d$ sono linearmente indipendenti su $\Q$, ovvero
 \[\forall k\in \Z^{d+1} \qquad (1,\alpha)\cdot k=0 \Rightarrow k=0,\]
 diremo altrimenti che $\alpha$ è \emph{risonante}, e in tal caso sarà non nullo $R=\{k\in\Z^d:(1,\alpha)\cdot k=0\}$
\end{defi}

\begin{teo} Le dinamiche $S^t_\alpha$ e $S_\alpha$ sono minimali se e solo se $\alpha$ è non risonante. 
Inoltre sono ergodiche se e solo se $\alpha$ è non risonante, e in tal caso sono anche unicamente ergodiche.\end{teo}

\begin{proof}[Dimostrazione (facoltativa).]
 Proveremo il teorema solo per tempo discreto: la controparte a tempo continuo è analoga o ne discende immediatamente. 
 Anzitutto mostriamo con un unico esempio che se $\alpha$ è risonante la dinamica non è transitiva nè ergodica: 
 infatti in tal caso esiste $(k_1,\dots k_d)\in\Z^d$ tale che $\sum\alpha_j k_j=k_0\in\Z$, e allora abbiamo l'osservabile continuo integrabile non costante
 \[\phi:\T^d\rightarrow \T^d \qquad \phi(x)=\sin 2\pi(k_1x_1+\dots+k_dx_d).\]
 Supponiamo $S_\alpha$ non minimale, ovvero (per un esercizio nella parte di Dinamica Topologica) non transitivo, 
 allora abbiamo due aperti disgiunti invarianti $U,V\subset \T^d$. L'indicatrice $f=\uno_U$ allora non è q.c. costante, ma allora i suoi coefficienti di Fourier
 non sono tutti nulli, e quindi da
 \[f(x)=\sum_{m\in \Z^d} f_m e^{2\pi i (m\cdot x)}\]
 \[f(S_\alpha x )=\sum_{m\in \Z^d} f_m e^{2\pi i (m\cdot x)} e^{2\pi i (m\cdot \alpha)}\]
 (intendendo $x\in \T^d$) imponendo l'invarianza per $S_\alpha$ si ha che esiste $m\in\Z^d$ tale che $e^{2\pi i (m\cdot \alpha)}=1$, da cui $\alpha$ risonante.
 Se $\alpha$ è non risonante, è facile mostrare, sviluppando ancora in serie di Fourier, che le funzioni integrabili invarianti sono quasi certamente costanti,
 e quindi che il sistema è ergodico.
\end{proof}

























Lo stesso risultato vale se riparametrizziamo il tempo, ovvero se consideriamo invece il flusso dato da $\dot x=\phi(x)\alpha$ con $\phi\in C^1(\T^d,\R^+)$, che ha le stesse orbite di $S^t$. La misura invariante è però ora $\phi(x)dx$, con $dx$ la misura di Haar.

Sia ora $\beta=(\alpha,1)\in \R^{d+1}$ non risonante e $\psi\in C^1(\T^d,\R^+)$: consideriamo il flusso in $\T^{d+1}$ dato da 
\[(x,s)\mapsto(x,s+t) \qquad \mbox{in }\T^d\times \R\]
al quoziente su $(x,s+\psi(x))~(S_\alpha,s)$. Questo flusso $S^t_{\alpha,\psi}$ è detto \emph{flusso speciale} su $\T^{d+1}$, e conserva la misura $\frac{dxds}{\int_{\T^d}\psi(x)dx}$ ($dx$ su $\T^d$ e $ds$ su $\T^1$).

\begin{teo}Il flusso speciale $S^t_{\alpha,\psi}$ è $C^r$-coniugato a $S^t_\beta$, ovvero
 \[\exists h\in C^r(\T^{d+1},\T^{d+1}):\quad \forall t \quad S^t_{\alpha,\psi}h=hS^t_\beta\]
 se e solo se esiste una soluzione $\chi\in C^r(\T^d,\R^+)$ dell'\emph{equazione coomologica}
 \[\psi(x)-\int_{\T^d}\psi(x)dx=\chi(x)-\chi(x+\alpha)\]
 (soluzione che sarà unica se $\chi$ ha media nulla).
\end{teo}

Studieremo per $d=1$ il legame tra l'equazione coomologica e le proprietà di $\alpha$. Premettiamo un lemma sulle serie di Fourier.

\begin{teo}Sia $\chi\in L^2\T^1$, $\hat\chi_n=\int_0^1\chi(x)e^{-2\pi inx}dx$ i suoi coefficienti di Fourier. Allora
\begin{lista}
 \item $\chi\in C^\infty$ se e solo se i suoi coefficienti sono \emph{a decrescenza rapida}, ovvero
      \[\forall N\in \N\quad \exists C_N>0: \quad \forall n \quad \left|\hat\chi_n\right|\leq\frac{C_N}{(1+|n|)^N}\]
 \item $\chi$ è analitica, ovvero esiste un intorno di $T^1=\R/\Z$ in $\C/\Z$ su cui $\chi$ si estende a una funzione olomorfa, se e solo se esistono $M,\delta>0$ tali che $\forall n \quad \left|\hat\chi_n\right|\leq Me^{-\delta |n|}$.
\end{lista}
\end{teo}
\begin{proof}
 Per il primo punto, se $\chi\in C^\infty$ integrando per parti si ottiene
 \[\hat\chi_n=\frac{1}{2\pi in}\int_0^1\chi'(x)e^{-2\pi inx}dx \quad \Rightarrow \quad \left|\hat\chi_n\right|\leq \frac{1}{2\pi |n|}\left|\widehat{\chi_n'}\right|\]
 da cui la tesi iterando; il viceversa è immediato perchè le stime sui $\hat\chi_n$ danno convergenza uniforme per le derivate.
 Il secondo punto. Se $\chi$ è analitica, sia $\chi(z)$ olomorfa sul rettangolo di vertici $0, 1,1+i\delta,i\delta$: orientiamo il bordo $\Gamma$ in senso antiorario e applichiamo il teorema di Cauchy ottenendo
 \[\int_\Gamma \chi(z)e^{-2\pi inz}dz=0 \quad \Rightarrow \quad \hat\chi_n=\int_0^1\chi(x)e^{-2\pi inx}dx=\int_0^1\chi(x+i\delta)e^{-2\pi in(x+i\delta)}dx\]
 ricordando che in $\C/\Z$ i bordi verticali del rettangolo coincidono, da cui il controllo cercato. Per gli $n$ pari invece si prende il rettangolo \emph{sotto} il segmento $[0,1]$, orientandolo in senso orario. Per il viceversa basta osservare che dalle stime si ha che $\chi$ è somma convergente di funzioni olomorfe. 
\end{proof}

Usiamo questa caratterizzazione per dare due criteri aritmetici su $\alpha$ che garantiscono
che l'equazione coomologica $\psi-\psi\circ R_\alpha$ ha soluzione,
nelle categorie $C^\infty(\mathbb{T}^1)$ e $C^\omega(\mathbb{T}^1)$.

\begin{teo}Per $\alpha$ irrazionale valgono i seguenti criteri.
\begin{lista}
\item $\alpha$ diofanteo $\sse$ per ogni $\psi\in\C^\infty(\mathbb{T}^1)$ a media nulla
l'equazione coomologica ha una (unica) soluzione $\chi\in C^\infty(\mathbb{T}^1)$.
\item $\alpha$ soddisfa $\lim_{n\to\infty}\frac{\ln q_{n+1}}{q_n}<\infty$ (dove $\frac{p_n}{q_n}$ sono i suoi convergenti)
$\sse$ per ogni $\psi\in C^\omega(\mathbb{T}^1)$ a media nulla
l'equazione coomologica ha una (unica) soluzione $\chi\in C^\omega(\mathbb{T}^1)$ a media nulla.
\end{lista}
\end{teo}

\begin{proof}Vediamo il punto $(1)$. $(\Rightarrow)$: trasformando l'equazione coomologica troviamo
$\pa{1-e^{2\pi i k\alpha}}\widehat{\chi}(k)=\widehat{\psi}(k)$. Inoltre per ipotesi
$\norm{k\alpha}\ge\gamma k^{-(1+\tau)}$, quindi
\[ \abs{\widehat{\chi}(k)}=\frac{\abs{\widehat{\psi}(k)}}{\abs{1-e^{2\pi i k\alpha}}}=\abs{\widehat{\psi}(k)}O(k^{1+\tau}) \]
e la decrescenza rapida di $\widehat{\psi}(k)$ implica quella di $\widehat{\chi}(k)$;
dunque la funzione $\chi$ avente questi coefficienti di Fourier (e $\widehat{\chi}(0)=0$) è $C^\infty$ e risolve l'equazione. \\
$(\Leftarrow)$: supponiamo per assurdo $\alpha$ non diofanteo, ovvero esista una successione $k_j\uparrow\infty$
per cui $\abs{1-e^{2\pi i k_j\alpha}}^{-1}\ge k_j^j$. Costruiamo una $\psi$ con una serie di Fourier ``lacunare'': sia
\[ \psi(x):=\sum_j \pa{1-e^{2\pi i k_j\alpha}}e^{2\pi i k_j x}+\sum_j \pa{1-e^{-2\pi i k_j\alpha}}e^{-2\pi i k_j x}. \]
$\psi$ ha valori reali ed è $C^\infty$ grazie alla decrescenza rapida dei coefficienti.
Però non può esistere una $\chi\in C^\infty$ che risolve l'equazione coomologica perché come visto prima
dovrebbe avere tutti i coefficienti di indice $k_j$ pari a $1$. \\
Passiamo al punto $(2)$ (facoltativo). $(\Rightarrow)$: fissato $n>0$ (il caso $n<0$ è analogo), per la proprietà di migliore approssimazione ??
abbiamo $\norm{n\alpha}\ge\norm{q_k\alpha}$ dove $q_k\le n<q_{k+1}$. Inoltre
\[ \norm{q_k\alpha}=\frac{1}{q_{k+1}+q_k\alpha_{k+1}}\sim\frac{1}{q_{k+1}} \]
($0<\alpha_{k+1}<1$ si ottiene applicando $k+1$ volte la mappa di Gauss a $\set{\alpha}$).
L'ipotesi dice che $\abs{\widehat{\psi}(n)}\le Ce^{-\delta n}$, quindi come nel punto precedente otteniamo
\[ \abs{\widehat{\chi}(n)}=\frac{\abs{\widehat{\psi}(n)}}{\abs{1-e^{2\pi i n\alpha}}}\lesssim Ce^{-\delta n}\norm{n\alpha}^{-1}
\lesssim Ce^{-\delta n}q_{k+1} \]
e per concludere basta ad esempio $q_{k+1}\le e^{\frac{\delta}{2}n}$ definitivamente.
Ma per ipotesi $\ln q_{k+1}\le \frac{\delta}{2}q_k\le\frac{\delta}{2}n$ definitivamente. \\
$(\Leftarrow)$: per assurdo esista $\delta>0$ e una successione $k_j\uparrow\infty$ tale che
$\ln q_{k_j+1}\ge\delta q_{k_j}$, cioè $q_{k_j+1}\ge e^{\delta q_{k_j}}$. Come prima ricaviamo
\[ \abs{1-e^{2\pi i q_{k_j}\alpha}}\sim\frac{1}{q_{k_j+1}}\le e^{-\delta q_{k_j}}, \]
quindi definendo
\[ \psi(x):=\sum_j \pa{1-e^{2\pi i k_j\alpha}}e^{2\pi i k_j x}+\sum_j \pa{1-e^{-2\pi i k_j\alpha}}e^{-2\pi i k_j x} \]
otteniamo una funzione analitica a valori reali. Una soluzione $\chi$ dell'equazione coomologica però avrebbe
$\widehat{\chi}(q_{k_j})=1$ per ogni $j$, assurdo.
\end{proof}

L'equazione coomologica $\psi-\psi\circ f=\phi$ si può risolvere in una situazione molto più generale.
Consideriamo un sistema dinamico topologico $(X,d,f)$.

\begin{defi}Dato $(X,d)$ spazio metrico compatto, $f\in\homeo(X)$ è \emph{minimale} se tutte le orbite sono dense.
\end{defi}

\begin{oss}Ovviamente la minimalità implica la transitività.
\end{oss}

Poniamo $d\psi:=\psi-\psi\circ f$. Vedendo $d$ come un operatore lineare continuo,
$d:C(X)\to C(X)$, osserviamo che $\ker d=\set{\text{integrali primi}}$ e l'ortogonale all'immagine è $(dC(X))^\perp
=\mathcal{M}_f(X)$, lo spazio delle misure reali $f$-invarianti.

Cerchiamo di risolvere l'equazione coomologica almeno formalmente:
$\psi=\phi+\psi\circ f$ e sostituendo $\psi$ a secondo membro otteniamo
\[ \psi=\phi+\phi\circ f+\psi\circ f^2 \]
e iterando
\[ \psi=\phi+\phi\circ f+\dots+\phi\circ f^{n-1}+\psi\circ f^n, \]
cioè formalmente $\psi=\sum_{j=0}^\infty \phi\circ f^j$. Torniamo a fare discorsi rigorosi. \\
Supponendo che una soluzione $\psi$ esista, come abbiamo visto deve valere
$\sum_{j=0}^{n-1}\phi\circ f^j=\psi-\psi\circ f^n$. Quindi una condizione necessaria è che le somme parziali
di Birkhoff siano equilimitate. Se $f$ è minimale vale anche il viceversa, anzi vale addirittura:
\begin{teo}[Gottshalk-Hedlund]Se $f\in\homeo(X)$ è minimale, $\phi\in C(X)$ e per qualche $x_0\in X$, $C>0$
è $\abs{\sum_{j=0}^{n-1}\pa{\phi\circ f^j}(x_0)}\le C$ per tutti gli $n\ge 0$, allora l'equazione coomologica
$\psi-\psi\circ f=\phi$ ha soluzione.
\end{teo}

Cioè basta l'equilimitatezza in un solo punto $x_0$! Cambiando segno a $\psi$, scriviamo l'equazione coomologica
in questa forma: $\psi\circ f-\psi=\phi$.

\begin{proof}Lavoriamo sullo \emph{skew-product} $X\times\R$: definiamo $F\in\homeo(X\times\R)$
con $F(x,u):=\pa{f(x),u+\phi(x)}$. $(X,\phi)$ è un fattore di $(X\times\R,F)$.
Nelle iterate di $F$ compaiono proprio le somme di Birkhoff:
\[ F^n(x,u)=\pa{f^n(x),u+\sum_{j=0}^{n-1}(\phi\circ f^j)(x)}. \]
L'ipotesi dice che la chiusura dell'orbita positiva di $(x_0,0)$, in simboli
$M:=\obar{\set{F^n(x_0,0)\mid n\ge 0}}$, è un insieme compatto e $F$-invariante (nel senso che $F(M)\subseteq M$). \\
Per il lemma di Zorn esiste un $K\subseteq M$ compatto, non vuoto e $F$-invariante che è \emph{minimale} tra gli insiemi
con queste proprietà. La minimalità di $K$ ha diverse conseguenze:
\begin{itemize}
	\item $F(K)=K$, altrimenti $F(K)$ sarebbe compatto invariante e più piccolo di $K$;
	\item $\pi_X(K)=X$: infatti da $\pi_X\circ F=f\circ\pi_X$ segue che $\pi_X(K)$ è $f$-invariante (e compatto),
	quindi è tutto $X$ per la minimalità di $f$;
	\item $T_t(x,u):=(x,u+t)$ (il flusso verticale) commuta con $F$ per ogni $t$ fissato, dunque
	$T_t(K)$ è $F$-invariante. Se $t\neq 0$ abbiamo $T_t(K)\nsupseteq K$ (perché $\pi_\R\pa{T_t(K)}=\pi_\R(K)+t$),
	quindi $T_t(K)\cap K\subsetneq K$, da cui (per minimalità di $K$) $T_t(K)\cap K=\emptyset$.
\end{itemize}
In questo modo abbiamo dimostrato che $K$ è il grafico di una funzione $\psi:X\to\R$. La compattezza di $K$
implica la continuità di $\psi$: se per assurdo $x_n\to x$ e $\abs{\psi(x_n)-\psi(x)}\ge\epsilon$,
a meno di sottosuccessioni $\pa{x_n,\psi(x_n)}\to (x,u)\in K$ con $u\neq\psi(x)$, per cui
$(x,u),(x,\psi(x))\in K$, contraddicendo l'ultimo punto. \\
Infine l'$F$-invarianza di $K$ dice che $F(x,\psi(x))=\pa{f(x),\psi(x)+\phi(x)}\in K$, cioè (per definizione di $\psi$)
$\psi\pa{f(x)}=\psi(x)+\phi(x)$.
\end{proof}


\section{Entropia}

Supponiamo di fare un esperimento con $n$ possibili esiti mutualmente esclusivi.
Modelliamo la situazione con uno spazio di probabilità (spazio dei campioni) $(X,\mu)$
e una partizione $X=A_1\sqcup \dots\sqcup A_n$ (il tutto anche mod $0$), con $\mu(A_i)=p_i$. 

Con l'esperimento non conosceremo il preciso elemento $x\in X$ ma solo l'esito $A_i$; vogliamo
trovare un modo di misurare a priori la quantità media di \emph{informazione} su $x$ che si ottiene eseguendo l'esperimento. 
\Eacc naturale supporre che questa informazione media guadagnata, che chiamiamo \emph{entropia}, dipenda solo da $n$ e da $p_1,\dots,p_n$.

Se la situazione è un sistema fisico in evoluzione (ad esempio una scatola divisa in due contenente una particella di gas, di cui
in ogni istante possiamo sapere in quale metà si trova), spesso c'è anche una trasformazione $f:X\to X$
che preserva $\mu$: lavorando a tempo discreto, $f$ dice come si sposta un punto nello spazio delle fasi
all'istante successivo. Eseguendo l'esperimento negli istanti $0,\dots,T$ troveremo ancora più informazione su $x$:
in sostanza conosceremo $\uno_{A_i}\pa{f^k(x)}=\uno_{f^{-k}(A_i)}$, perciò l'entropia sarà quella associata
alla \emph{partizione congiunta} fatta da tutte le possibili intersezioni $A_{i_0}\cap f^{-1}(A_{i_1})\cap \dots\cap f^{-T}(A_{i_T})$.

\subsection{Definizione assiomatica dell'entropia}
Cerchiamo un'espressione esplicita per l'entropia, funzione di $p_1,\dots,p_n$, concentrandoci sulla situazione di una sola misurazione
(cioè senza trasformazioni $f$ e senza i diversi istanti di tempo).
Poiché $\sum_i p_i=1$, cerchiamo per ogni $n\ge 1$ una funzione
\[ H^{(n)}:\Delta^{(n)}\to [0,+\infty), \]
dove $\Delta^{(n)}:=\set{(x_1,\dots,x_n):0\le x_i\le 1,\ \sum_i x_i=1}$ è il simplesso $(n-1)$-dimensionale. Bastano alcuni assiomi di buon senso per determinare completamente $H^{(n)}$:
\begin{lista}
\item simmetria negli argomenti
\item $H^{(n)}(1,0,\dots,0)=0$
\item $H^{(n)}(0,p_2,\dots,p_n)=H^{(n-1)}(p_2,\dots,p_n)$
\item $H^{(n)}(p_1,\dots,p_n)\le H^{(n)}\pa{\frac{1}{n},\dots,\frac{1}{n}}$
\item $H^{(n\ell)}(\pi_{11},\dots,\pi_{1\ell},\dots,\pi_{n1},\dots,\pi_{n\ell})=H^{(n)}(p_1,\dots,p_n)
+\sum_{i=1}^n p_i H^{(\ell)}\pa{\frac{\pi_{i1}}{p_i},\dots,\frac{\pi_{i\ell}}{p_i}}$, dove $p_i:=\sum \pi_{ij}$.
\end{lista}

L'ultimo assioma equivale a chiedere che, data una partizione $(A_{ij})_{1\le i\le n,1\le j\le\ell}$
e posto $B_i:=\sqcup_j A_{ij}$, l'entropia di $(A_{ij})$ sia quella della partizione meno fine $(B_i)$
più la media pesata delle \emph{entropie relative} nei blocchi $B_i$, calcolate con la probabilità condizionata.

\begin{teo}Se le funzioni $H^{(n)}$ soddisfano i suddetti assiomi e sono continue, allora
\[ H^{(n)}(p_1,\dots,p_n)=-c\sum_{i=1}^n p_i\ln p_i \]
per qualche costante $c\ge 0$ indipendente da $n$.
\end{teo}

\begin{proof}Ovviamente $H^{(1)}(1)=0$. Calcoliamo intanto
$K(n):=H^{(n)}\pa{\frac{1}{n},\dots,\frac{1}{n}}$. \\
Grazie a $(3)$ e $(4)$, $K(n)$ è non decrescente. Inoltre $(5)$ dà $K(n\ell)=K(n)+K(\ell)$. \\
Supponendo ora $r,\ell\ge 2$, per ogni $n$ scegliamo $m$ tale che $\ell^m\le r^n\le\ell^{m+1}$.
$K(\cdot)$ e il logaritmo sono entrambi non decrescenti e trasformano prodotti in somme, quindi
\[ \ln(\ell^m)\le\ln(r^n)\le\ln(\ell^{m+1})\implica \frac{m}{n}\le\frac{\ln r}{\ln \ell}\le\frac{m}{n}+\frac{1}{n} \]
\[ K(\ell^m)\le K(r^n)\le K(\ell^{m+1})\implica \frac{m}{n}\le\frac{K(r)}{K(\ell)}\le\frac{m}{n}+\frac{1}{n} \]
e mandando $n\to\infty$ otteniamo $\frac{K(r)}{K(\ell)}=\frac{\ln r}{\ln n}$, da cui $K(n)=c\ln n$ per qualche $c\ge 0$. \\
Osserviamo ora che, grazie a $(3)$, l'assioma $(5)$ vale in questa forma più generale:
\[ H^{\pa{\sum \ell_i}}(\pi_{11},\dots,\pi_{1\ell_1},\dots,\pi_{n1},\dots,\pi_{n\ell_n})
=H^{(n)}(p_1,\dots,p_n)+\sum_{i=1}^n p_i H^{(\ell_i)}\pa{\frac{\pi_{i1}}{p_i},\dots,\frac{\pi_{i\ell_i}}{p_i}} \]
dove come al solito $p_i:=\sum \pi_{ij}$. Quindi se $r_1,\dots,r_n$ sono interi positivi e $s=\sum_i r_i$ otteniamo
\[ H^{(s)}\pa{\frac{1}{s},\dots,\frac{1}{s}}=H^{(n)}\pa{\frac{r_1}{s},\dots,\frac{r_n}{s}}
+\sum_{i=1}^n \frac{r_i}{s}H^{(r_i)}\pa{\frac{1}{r_i},\dots,\frac{1}{r_i}}, \]
ovvero definendo $p_i:=\frac{r_i}{s}$ abbiamo $H^{(n)}(p_1,\dots,p_n)=K(s)-\sum_i p_i K(r_i)=-c\sum_i p_i\ln p_i$.
Per la densità in $\Delta^{(n)}$ dei punti a coordinate razionali e la continuità di $H^{(n)}$ arriviamo alla tesi.
\end{proof}

\begin{oss}Le $H^{(n)}$ definite dalla formula trovata soddisfano effettivamente tutti gli assiomi.
L'unico non ovvio è il $(4)$: la convessità di $x\mapsto x\ln x$ su $[0,\infty)$ dà
$\frac{1}{n}\sum p_i\ln p_i\ge\frac{1}{n}\ln\pa{\frac{1}{n}}$ (perché $\sum p_i=1$), ovvero
$-\sum p_i\ln p_i\le\ln n=K(n)$.
\end{oss}

Scegliamo per convenzione $c=1$.
Data una partizione $\mathcal{P}=\set{A_1,\dots,A_n}$ di $X$, scriveremo anche $H^{(n)}(\mathcal{P})$ o $H(\mathcal{P})$
anziché $H^{(n)}\pa{\mu(A_1),\dots,\mu(A_n)}$.

\subsection{Entropia di Kolmogorov-Sinai}

\begin{defi}Date due partizioni $\mathcal{P}=\set{A_1,\dots,A_n}$ e $\mathcal{Q}=\set{B_1,\dots,B_\ell}$,
definiamo il \emph{raffinamento comune} $\mathcal{P}\vee\mathcal{Q}:=\set{A_i\cap B_j\mid 1\le i\le n,1\le j\le \ell}$.
\end{defi}

Se oltre a una partizione $\mathcal{P}$ abbiamo anche una trasformazione misurabile $f:X\to X$ che preserva $\mu$,
possiamo considerare l'informazione media che si acquisisce dopo $T$ istanti di tempo (rilevando
ad ogni istante $k$ l'insieme $A_i$ in cui finisce $f^k(x)$): questa è $\frac{1}{T}H^{(n^T)}(\mathcal{P}_T)$,
dove $\mathcal{P}_T:=\mathcal{P}\vee\cdots\vee f^{-(T-1)}(\mathcal{P})$ (vista come partizione composta
da $n^T$ insiemi). Osserviamo che abbiamo mediato anche nel tempo, cioè abbiamo diviso per $T$.

\begin{lemma}[Fekete, facoltativo]\label{fekete} Se $a:\N\nonzero\to\R$ è subadditiva, ovvero $a(m+n)\le a(m)+a(n)$, esiste sempre
$\lim_{n\to\infty}\frac{a(n)}{n}$ e coincide con $\inf_n\frac{a(n)}{n}$.
\end{lemma}

\begin{proof}Ovviamente $\liminf_{n\to\infty}\frac{a(n)}{n}\ge\inf_n\frac{a(n)}{n}$. Fissiamo ora $m>0$
e per ogni $n$ facciamo la divisione con resto: $n=km+r$. La subadditività dà
\[ \frac{a(n)}{n}\le\frac{a(km)+a(r)}{km+r}=\frac{k}{km+r}a(m)+\frac{a(r)}{km+r} \]
e mandando $n\to\infty$ (essendo $r$ limitato e così pure $a(r)$) otteniamo $\limsup_{n\to\infty}\le\frac{a(m)}{m}$.
Dato che questo vale per tutti gli $m$ segue la tesi.
\end{proof}

\begin{cor}Esiste sempre $h(f,\mathcal{P}):=\lim_{T\to\infty}\frac{1}{T}H(\mathcal{P}_T)$.
\end{cor}

\begin{proof}Grazie a Fekete, basta mostrare che $H(\mathcal{P}_{m+n})\le H(\mathcal{P}_m)+H(\mathcal{P}_n)$.
Essendo $\mathcal{P}_{m+n}=\mathcal{P}_m\vee f^{-m}(\mathcal{P}_n)$ e $H(\mathcal{P}_n)=H\pa{f^{-m}(\mathcal{P}_n)}$,
basta vedere che $H(\mathcal{P}\vee\mathcal{Q})\le H(\mathcal{P})+H(\mathcal{Q})$ per due generiche
partizioni $\mathcal{P}=\set{A_1,\dots,A_k}$ e $\mathcal{Q}=\set{B_1,\dots,B_\ell}$. \\
Dall'assioma $(5)$ troviamo
\[ H(\mathcal{P}\vee\mathcal{Q})=H(\mathcal{P})+\sum_{i,j}\lambda_i f(t_{ij}) \]
dove $\lambda_i=\mu(A_i)$, $t_{ij}=\frac{\mu(A_i\cap B_j)}{\mu(A_i)}$ e $f(x)=-x\ln x$. \\
Per la concavità di $f$ e il fatto che $\sum_i\lambda_i=1$ otteniamo (Jensen)
\[ \sum_i \lambda_i f(t_{ij})\le f\pa{\sum_i\lambda_i t_{ij}}=f(\mu(B_j))=-\mu(B_j)\ln\mu(B_j), \]
da cui $\sum_{i,j}\lambda_i f(t_{ij})\le -\sum_j \mu(B_j)\ln\mu(B_j)=H(\mathcal{Q})$.
\end{proof}

\begin{oss}La dimostrazione precedente mostra anche che, se $\mathcal{P}$ e $\mathcal{Q}$ sono partizioni indipendenti,
vale $H(\mathcal{P}\vee\mathcal{Q})=H(\mathcal{P})+H(\mathcal{Q})$: infatti in questo caso $t_{ij}=\mu(B_j)$
e quindi $\sum_{i,j}\lambda_i f(t_{ij})=\sum_j f\pa{\mu(B_j)}=H(\mathcal{Q})$.
\end{oss}

\begin{defi}L'\emph{entropia di Kolmogorov-Sinai} del sistema dinamico misurabile $(X,\mathcal{A},\mu,f)$ è
$h_{KS}(f):=\sup_{\mathcal{P}}h(f,\mathcal{P})$, al variare delle partizioni misurabili $\mathcal{P}$.
\end{defi}

\begin{oss}Si vede facilmente che $h_{KS}$ è invariante per isomorfismo di sistemi dinamici misurabili.
\end{oss}

%%%%%%%%%%%%%%%%%%%%%%%%%%%%%%%%%%%%%%%%%%%%%%%%%%%%%%%%%%%%%%%%
\subsection{Schemi di Bernoulli}


Consideriamo $X:=\set{1,\dots,N}^{\Z}$: questo diventa uno spazio metrico definendo, per $x,y\in X$, $a(x,y):=\min\set{\abs{i}:x_i\neq y_i}$ 
(che vale per convenzione $+\infty$ se $x=y$) e $d(x,y)=2^{-a(x,y)}$. $(X,d)$ è compatto, e lo shift sinistro $\sigma:X\to X$ dato da $\pa{\sigma(x)}_i:=x_{i+1}$ 
è un omeomorfismo di $X$.

Data una misura di probabilità arbitraria su $\set{1,\dots,N}$ che vale $p_i$ su $\set{i}$, possiamo mettere su $X$
la probabilità prodotto $\mu$, che è ovviamente $\sigma$-invariante. 
Chiamiamo \emph{schema di Bernoulli} il sistema dinamico misurabile che abbiamo costruito, e lo indicheremo con $BS(p_1,\dots,p_N)$.

Nel caso di due simboli, che chiamiamo $0$ e $1$ per comodità, abbiamo una mappa 
\[\tilde{\pi}:\set{0,1}^{\Z}\to [0,1]^2  \qquad (x_i)\mapsto (\xi,\eta) \mbox{ con }  \begin{cases} \xi=\sum_{i\ge 0}x_i 2^{-(i+1)} \\ \eta=\sum_{i<0}x_i 2^{-\abs{i}}\end{cases}.\] 
$\tilde{\pi}$ ha inversa definita q.o. e coniuga $\sigma$ nella ``mappa del panettiere'' $f(x,y):=\pa{\pa{2x},\frac{y+\floor{2x}}{2}}$,
che graficamente agisce come

\begin{center}\begin{tikzpicture}[scale=0.70]
\draw (0,0) -- (2,0);
\draw (2,2) -- (2,0);
\draw (2,2) -- (0,2);
\draw (0,2) -- (0,0);
\draw[dashed] (1,0) -- (1,2);
\draw[->] (3,1) -- (4,1);
\draw (5,0.5) -- (7,0.5);
\draw (5,0.5) -- (5,1.5);
\draw (5,1.5) -- (7,1.5);
\draw[dashed] (7,0.5) -- (7,1.5);
\draw (8,0.5) -- (10,0.5);
\draw (10,0.5) -- (10,1.5);
\draw (8,1.5) -- (10,1.5);
\draw[dashed] (8,0.5) -- (8,1.5);
\draw[->] (8.5,2) arc (60:120:2);
\draw[->] (11,1) -- (12,1);
\draw (13,0) -- (15,0);
\draw (13,1) -- (15,1);
\draw (13,2) -- (15,2);
\draw (13,0) -- (13,1);
\draw (15,1) -- (15,2);
\draw[dashed] (13,1) -- (13,2);
\draw[dashed] (15,0) -- (15,1);
\end{tikzpicture}\end{center}


L'entropia di uno schema di Bernoulli si calcola facilmente se diamo per buono questo fatto:

\begin{teo}[Kolmogorov-Sinai]Sia $(X,\mathcal{A},\mu,f)$ con $f$ invertibile e $f^{-1}$ anch'essa misurabile.
Se $\mathcal{P}$ è una partizione \emph{generante}, cioè tale che $\bigvee_{n=-\infty}^\infty T^n(\mathcal{P})=\mathcal{A}$ modulo $0$
(il primo membro è la $\sigma$-algebra generata da $T^n(\mathcal{P})$ per $n\in\Z$), allora $h_{KS}(f)=h(f,\mathcal{P})$.
\end{teo}

Sia $A_k:=\set{x\in X:x_0=k}$ per $k=1,\dots,N$; la partizione $\mathcal{P}:=\set{A_1,\dots,A_N}$ è generante.
Inoltre le partizioni $\sigma^n(\mathcal{P})$ sono tutte indipendenti congiuntamente, quindi per l'ultima osservazione
$H(\mathcal{P}_n)=nH(\mathcal{P})$, per cui $h(\sigma,\mathcal{P})=H(\mathcal{P})$. Dunque, per Kolmogorov-Sinai, l'entropia di 
$BS(p_1,\dots,p_N)$ è $h(\sigma,\mathcal{P})=H(\mathcal{P})=-\sum_{i=1}^N p_i\ln p_i$.

Sappiamo che due sistemi misurabili isomorfi hanno la stessa entropia. Sorprendentemente, per gli schemi di Bernoulli vale anche il viceversa
(che non dimostriamo):

\begin{teo}[Ornstein]Se due schemi di Bernoulli hanno la stessa entropia, sono isomorfi come sistemi dinamici misurabili.
\end{teo}

\subsection{Entropia topologica}

Vogliamo dare una definizione di entropia anche per sistemi dinamici topologici, senza parlare di misura.
Dato $(X,d,f)$ con $X$ compatto e $f:X\to X$ continua, poniamo $d_n(x,y):=\max_{0\le j\le n}d\pa{f^j(x),f^j(y)}$.
Indichiamo con $B(x,r,n)$ le palle rispetto alla distanza $d_n$.

\begin{defi}$S\subseteq X$ è $(n,\epsilon)$-\emph{spanning} se per ogni $x\in X$ c'è un $y\in S$
con $d_n(x,y)<\epsilon$.
\end{defi}

\begin{oss}Per la continuità di $f$, $d_n$ è una distanza equivalente a $d$, cioè
induce la stessa topologia. Dalla compattezza di $X$ segue che esiste sempre un insieme $(n,\epsilon)$-spanning finito.
\end{oss}

\begin{defi}Poniamo $r(n,\epsilon):=\min\set{\abs{S}:S\ (n,\epsilon)\text{-spanning}}$ ($<\infty$
per quanto appena osservato).
\end{defi}

\begin{oss}Se $X$ ha un ricoprimento con $m$ insiemi di diametro $<\epsilon$, $r(n,\epsilon)\le m^{n+1}$.
Infatti, chiamando questi insiemi $A_1,\dots,A_m$, basta prendere un punto da ogni insieme del tipo
$A_{i_0}\cap f^{-1}(A_{i_1})\cap\dots\cap f^{-n}(A_{i_n})$. Ogni $x\in X$ sta in uno di questi $m^{n+1}$ insiemi,
quindi la distanza $d_n$ tra $x$ e il punto corrispondente è $<\epsilon$ per costruzione.
\end{oss}

\begin{defi}L'\emph{entropia topologica} di $(X,d,f)$ è
\[ h_{top}(f):=\lim_{\epsilon\to 0}\limsup_{n\to\infty}\frac{1}{n}\ln r(n,\epsilon). \]
\end{defi}

Il limite esiste perché $\epsilon\mapsto\limsup_{n\to\infty}\frac{1}{n}\ln r(n,\epsilon)$ è decrescente.

\begin{prop}L'entropia topologica ha le seguenti proprietà:
\begin{lista}
\item se $f$ è un'isometria, $h_{top}(f)=0$
\item se $f\in\homeo(X)$, $h_{top}(f^{-1})=h_{top}(f)$
\item $h_{top}(f^m)=mh_{top}(f)$ per $m>0$.
\end{lista}
\end{prop}

\begin{proof}$(1)$: fissato $\epsilon$, scegliamo $S\subseteq X$ finito tale che per ogni $x\in X$
ci sia un $y\in S$ con $d(x,y)<\epsilon$. $S$ è $(n,\epsilon)$-spanning per ogni $n$ in quanto $d_n=d$.
Perciò $\limsup_{n\to\infty}\frac{1}{n}\ln r(n,\epsilon)\le\limsup_{n\to\infty}\frac{\ln\abs{S}}{n}=0$. \\
$(2)$: se $S$ è $(n,\epsilon)$-spanning per $f$, $f^n(S)$ è $(n,\epsilon)$-spanning per $f^{-1}$. \\
$(3)$: chiaramente $r(n,\epsilon,f^m)\le r(mn,\epsilon,f)$ perché un insieme $(mn,\epsilon)$-spanning per $f$
è $(n,\epsilon)$-spanning per $f^m$. Quindi $h_{top}(f^m)\le m h_{top}(f)$.
Per l'altra disuguaglianza: dato $\epsilon$ sia $\delta<\epsilon$ tale che per $d(x,y)<\delta$ si abbia $d\pa{f^j(x),f^j(y)}<\epsilon$,
per ogni $j=0,\dots,m-1$. Ora un insieme $(n,\delta)$-spanning per $f^m$ è $(mn,\epsilon)$-spanning per $f$:
basta scrivere ogni $0\le k\le mn$ come $k=qm+r$ e usare le definizioni.
Quindi $\limsup_{n\to\infty}\frac{1}{n}\ln r(n,\delta,f^m)\ge m\limsup_{n\to\infty}\frac{\ln r(mn,\epsilon,f)}{mn}
=m\limsup_{k\to\infty}\frac{\ln r(k,\epsilon,f)}{k}$.
\end{proof}

\begin{esempio}La rotazione irrazionale $R_\alpha$, essendo un'isometria, ha entropia nulla.
\end{esempio}

\begin{esempio}L'entropia può anche essere infinita. Consideriamo ad esempio
\[ X:=\set{x\in\ell^2(\N):\abs{x_i}\le 2^{-i}} \]
(compatto) con l'endomorfismo $f:X\to X$, $\pa{f(x)}_i:=2x_{i+1}$. \Eacc $r\pa{n,\frac{1}{k}}>k^n$:
infatti $d_n(x,y)\ge\max_{0\le k\le n}\abs{\pa{f^k(x)}_0-\pa{f^k(y)}_0}=\max_{0\le k\le n}2^k\abs{x_k-y_k}$,
quindi, se $S$ è $\pa{n,\frac{1}{k}}$-spanning, scegliendo
\[ x=\sum_{k=0}^n\alpha_k 2^{-k}e_k \]
(con $\abs{\alpha_k}\le 1$) deve esistere un $y\in S$ tale che $\abs{x_k-y_k}<\frac{1}{2^k\cdot k}$,
cioè $\abs{\alpha_k-(2^k y_k)}<\frac{1}{k}$ per $0\le k\le n$.
Ma gli $\alpha_k$ variano in modo arbitrario su $[-1,1]$, quindi
\[ [-1,1]^{n+1}\subseteq \bigcup_{y\in S}B\pa{\pi(y),\frac{1}{k}}, \]
dove $\pi(y)=(y_0,2y_1,\dots,2^n y_n)\in \R^{n+1}$ e le palle sono rispetto alla norma $\norm{\fantasma{a}}_\infty$.
Confrontando le misure, $2^{n+1}\le \abs{S}\cdot\pa{\frac{2}{k}}^{n+1}$ $\implica$ $\abs{S}\ge k^{n+1}$, che è la tesi. 
Dunque $h_{top}(f)=\lim_{k\to\infty}\limsup_{n\to\infty}\frac{1}{n}\ln r\pa{n,\frac{1}{k}}\ge\lim_{k\to\infty}\ln k=\infty$.
\end{esempio}

Vediamo altri modi equivalenti di definire l'entropia topologica.

\begin{defi}$A\subseteq X$ è $(n,\epsilon)$-\emph{separato} se per ogni $x,y\in A$ distinti
vale $d_n(x,y)\ge\epsilon$.
\end{defi}

\begin{prop}Sia $s(n,\epsilon)$ la massima cardinalità di un insieme $(n,\epsilon)$-separato.
Allora $h_{top}(f)=\lim_{\epsilon\to 0}\limsup_{n\to\infty}\frac{1}{n}\ln s(n,\epsilon)$.
\end{prop}

\begin{proof}Sia $A$ un insieme che realizza $s(n,\epsilon)$. $A$ è anche $(n,\epsilon)$-spanning:
ogni $x\in X\setminus A$ dista (nel senso di $d_n$) meno di $\epsilon$ da un elemento di $A$, altrimenti $A$
non sarebbe massimale. Quindi $r(n,\epsilon)\le s(n,\epsilon)$. \\
Inoltre $s(n,2\epsilon)\le r(n,\epsilon)$: siano $A$ è $(n,2\epsilon)$-separato
e $S=\set{x_1,\dots,x_k}$ è $(n,\epsilon)$-spanning; osserviamo che in realtà
$S$ è $(n,\epsilon')$-spanning per qualche $\epsilon'<\epsilon$ (per compattezza di $X$),
quindi $\set{B(x_i,\epsilon',n)\mid i=1,\dots,k}$ è un ricoprimento con diametro massimo (rispetto a $d_n$) $\le 2\epsilon'$,
perciò ogni palla $B(x_i,\epsilon',n)$ contiene al più un elemento di $A$.
Da ciò segue che $\abs{A}\le\abs{S}$. \\
Abbiamo ottenuto $s(n,2\epsilon)\le r(n,\epsilon)\le s(n,\epsilon)$, che dà la tesi.
\end{proof}

\begin{defi}Dati $\alpha,\beta$ ricoprimenti aperti di $X$, definiamo il \emph{ricoprimento congiunto}
$\alpha\vee\beta:=\set{A\cap B\mid A\in\alpha,B\in\beta}$. \\
Dato un ricoprimento aperto $\alpha$, chiamiamo $N(\alpha)$ la cardinalità minima di un sottoricoprimento di $\alpha$
(finita per compattezza).
\end{defi}

\begin{oss}$N(\alpha\vee\beta)\le N(\alpha)N(\beta)$ e $N\pa{f^{-1}(\alpha)}\le N(\alpha)$
(qui $f^{-1}(\alpha)=\set{f^{-1}(A)\mid A\in\alpha}$). \\
Dunque, ponendo temporaneamente $a(n):=\ln N\pa{\bigvee_{i=0}^{n-1}f^{-i}(\alpha)}$,
\[ a(m+n)\le\ln N\pa{\bigvee_{i=0}^{m-1}f^{-i}(\alpha)}+\ln N\pa{f^{-m}\pa{\bigvee_{i=0}^{n-1}f^{-i}(\alpha)}}
\le a(m)+a(n). \]
Perciò, per \hyperref[fekete]{Fekete}, esiste sempre $\lim_{n\to\infty}\frac{1}{n}\ln N\pa{\bigvee_{i=0}^{n-1}f^{-i}(\alpha)}$.
\end{oss}

Diamo ora una definizione alternativa di entropia topologica, che rende manifesta la somiglianza con l'entropia
di Kolmogorov-Sinai:

\begin{teo}$h_{top}(f)=\sup_\alpha\lim_{n\to\infty}\frac{1}{n}\ln N\pa{\bigvee_{i=0}^{n-1}f^{-i}(\alpha)}$.
\end{teo}

\begin{proof}[Dimostrazione (facoltativa)]Sia $H$ il secondo membro della tesi. \\
Dato $\alpha=\set{A_i}$ ricoprimento aperto, chiamiamo $\delta>0$ il numero di Lebesgue
associato. Sia ora $S\subseteq X$ $(n-1,\delta)$-spanning; per ogni $x\in S$ e ogni $0\le k\le n-1$ esiste $i_k$ per cui
$B(f^k(x),\delta)\subseteq A_{i_k}$, ovvero
\[ B(x,\delta,n-1)=\bigcap_{k=0}^{n-1} f^{-k}\pa{B\pa{f^k(x),\delta}}\subseteq\bigcap_{k=0}^{n-1} f^{-k}\pa{A_{i_k}}. \]
Dunque $\mathcal{U}:=\set{B(x,\delta,n-1)\mid x\in S}$ è un raffinamento di $\bigvee_{i=0}^{n-1}f^{-i}(\alpha)$,
da cui segue che $\abs{S}\ge N\pa{\bigvee_{i=0}^{n-1}f^{-i}(\alpha)}$ (per ogni elemento di $\mathcal{U}$
scelgo un elemento di $\bigvee_{i=0}^{n-1}f^{-i}(\alpha)$ che lo include e, siccome $\mathcal{U}$ è anche un ricoprimento,
i sovrainsiemi scelti ricoprono $X$). Infine
\[ h_{top}(f)\ge\limsup_{n\to\infty}\frac{1}{n}\ln r(n-1,\delta)
\ge\limsup_{n\to\infty}\frac{1}{n}\ln N\pa{\bigvee_{i=0}^{n-1}f^{-i}(\alpha)} \]
e passando al $\sup$ su $\alpha$ otteniamo $h_{top}(f)\ge H$. \\
Viceversa, fissiamo $\epsilon>0$ e sia $A\subseteq X$ $(n,\epsilon)$-separato di cardinalità massima.
Prendiamo un ricoprimento aperto $\alpha=\set{A_i}$ con $\max_i\diam(A_i)<\epsilon$ e osserviamo
che un elemento di $\bigvee_{k=0}^n f^{-k}(\alpha)$ contiene al più un punto di $A$,
in quanto $x,y\in\bigcap_{k=0}^n f^{-k}\pa{A_{i_k}}$ implica $f^k(x),f^k(y)\in A_{i_k}$ e quindi $d\pa{f^k(x),f^k(y)}<\epsilon$
per ogni $0\le k\le n$, da cui $d_n(x,y)<\epsilon$. \\
Ma allora $s(n,\epsilon)\le N\pa{\bigvee_{k=0}^n f^{-k}(\alpha)}$, da cui
\[ H\ge\lim_{n\to\infty}\frac{1}{n-1}\ln N\pa{\bigvee_{i=0}^{n-1}f^{-i}(\alpha)}
=\lim_{n\to\infty}\frac{1}{n}\ln N\pa{\bigvee_{i=0}^n f^{-i}(\alpha)}\ge\limsup_{n\to\infty}\frac{1}{n}\ln s(n,\epsilon) \]
e passando al limite per $\epsilon\to 0$ otteniamo $H\ge h_{top}(f)$.
\end{proof}

Concludiamo con il teorema più importante sull'entropia topologica:

\begin{teo}L'entropia topologica non dipende dalla metrica che induce la topologia di $X$.
Dunque è anche invariante per coniugazione.
\end{teo}

\begin{proof}Siano $d$ e $d'$ due metriche che inducono la stessa topologia di $X$ (compatto).
Fissiamo $\epsilon>0$ e poniamo
\[ D_\epsilon:=\set{(x,y)\in X\times X:d(x,y)\ge\epsilon}. \]
Per la continuità di $d:X\times X\to\R$, $D_\epsilon\subseteq X\times X$ è chiuso, quindi compatto (perché $X\times X$ è compatto).
Perciò esiste
\[ \delta'=\delta'(\epsilon):=\min_{(x,y)\in D_\epsilon}d'(x,y)>0 \]
e osserviamo che $d'(x,y)<\delta'\implica (x,y)\nin D_\epsilon\implica d(x,y)<\epsilon$. \\
Quindi, per ogni $x\in X$, $B_{d'}(x,\delta')\subseteq B_d(x,\epsilon)$ e deduciamo
\[ B_{d'_n}(x,\delta')=\bigcap_{k=0}^n f^{-k}\pa{B_{d'}\pa{f^k(x),\delta'}}\subseteq\bigcap_{k=0}^n f^{-k}\pa{B_d\pa{f^k(x),\epsilon}}
=B_{d_n}(x,\epsilon). \]
Ma allora un insieme $(n,\delta')$-spanning (per $d'$) è $(n,\epsilon)$-spanning (per $d$), da cui
$r_{d'}(n,\delta')\ge r_d(n,\epsilon)$ e infine
\[ h_{top}(f,d)=\sup_\epsilon \limsup_{n\to\infty}\frac{1}{n}\ln r_d(n,\epsilon)
\le \sup_\delta'\limsup_{n\to\infty}\frac{1}{n}\ln r_{d'}(n,\delta')=h_{top}(f,d'). \]
Scambiando i ruoli di $d$ e $d'$ otteniamo anche la disuguaglianza opposta.
\end{proof}

\begin{oss}Nonostante l'invarianza per coniugio, abbiamo già visto che in generale $h_{top}(f)\neq h_{top}(f^2)$.
Quindi l'entropia non dà un buon invariante nel caso dei flussi: se $S_t$ è un flusso, posto $f:=S_t$, vorremmo
una quantità $\tilde{h}$ che non cambia nemmeno riparametrizzando il tempo, in particolare
$\tilde{h}(f)=\tilde{h}(S_t)=\tilde{h}(S_{2t})=\tilde{h}(f^2)$.
L'entropia topologica non soddisfa questa proprietà (si dice che non è invariante per \emph{equivalenza orbitale}).
\end{oss}

\subsection{Mappe espansive e altri risultati}

\begin{defi}$f:X\to X$ è \emph{espansiva} se per qualche $\delta>0$ vale l'implicazione
$\forall n\ge 0\ d\pa{f^n(x),f^n(y)}\le\delta\implica x=y$.
\end{defi}

\begin{esempio}$f:\mathbb{T}^1\to\mathbb{T}^1$, $f(x):=px$ è espansiva per $p\ge 2$ intero.
\end{esempio}

Diamo un risultato utile senza dimostrazione:

\begin{teo}Se $f$ è espansiva, allora per ogni $\epsilon>0$ sufficientemente piccolo vale
\[ h_{top}(f)=\lim_{n\to\infty}\frac{1}{n}\ln r(n,\epsilon)=\lim_{n\to\infty}\frac{1}{n}\ln s(n,\epsilon) \]
(l'esistenza dei limiti fa parte dell'enunciato). Inoltre, se $\alpha$ è un ricoprimento aperto con $\max\diam \alpha<\delta$,
\[ h_{top}(f)=\lim_{n\to\infty}\frac{1}{n}\ln N\pa{\bigvee_{i=0}^{n-1}f^{-i}(\alpha)}. \]
\end{teo}

Altri fatti interessanti (sull'entropia di Kolmogorov-Sinai):

\begin{teo}[Shannon-Breiman-McMillan]Siano $(X,\mathcal{A},\mu,f)$ ergodico e $\alpha$ una partizione misurabile finita.
Fissato $x\in X$, indichiamo con $\alpha^n(x)$ l'elemento di $\bigvee_{i=0}^{n-1}f^{-i}(\alpha)$ che contiene $x$
(è ben definito per q.o. $x$). Sia $h_{KS}(f,\alpha):=\lim_{n\to\infty}\frac{1}{n}H\pa{\vee_{i=0}^{n-1}f^{-k}(\alpha)}$
l'entropia relativa alla partizione $\alpha$. Allora, per q.o. $x$,
\[ h_{KS}(f,\alpha)=\lim_{n\to\infty}-\frac{1}{n}\ln \mu\pa{\alpha^n(x)}. \]
\end{teo}

\begin{teo}[Brin-Katok]Siano $(X,d)$ compatto, $f:X\to X$ continua e $\mu$ misura di probabilità invariante ed ergodica.
Allora, per q.o. $x$,
\[ h_{KS}(f)=\sup_\epsilon\limsup_{n\to\infty}-\frac{1}{n}\ln\mu\pa{B(x,\epsilon,n)}. \]
\end{teo}

Quest'ultimo teorema è sorprendente se lo confrontiamo con il teorema di ricorrenza di Poincaré:
se l'entropia di $f$ è positiva, fissato qualsiasi $x_0$ per cui vale la tesi, esistono un $\epsilon>0$ e un $0<\gamma<1$ tali che
$\mu\pa{B(x_0,\epsilon,n)}\le \gamma^n$ definitivamente. \\
Quindi, scelto $\delta\ll\epsilon$, nonostante i punti di $B(x_0,\delta)$ siano ricorrenti, la probabilità che
due punti $x,y\in B(x_0,\delta)$ restino vicini nei primi $n$ istanti scende esponenzialmente con $n$.
Questo suggerisce che ritornino in $B(x_0,\delta)$ in momenti diversi e ``scorrelati''.

Infine entropia topologica e misurabile sono legate da questa relazione:
\begin{teo}Siano $(X,d)$ metrico compatto e $f:X\to X$ continua. Indichiamo con $\mathcal{P}_f(X)$ l'insieme
delle misure di probabilità $f$-invarianti. Vale
\[ h_{top}(f)=\sup_{\mu\in\mathcal{P}_f(X)}h_{KS}(f,\mu). \]
\end{teo}

\subsection{Catene di Markov}

Consideriamo un grafo \emph{diretto} $\Gamma$ su $N$ nodi, cioè in simboli $\Gamma\subseteq\set{1,\dots,N}^2$
(un arco può anche collegare un nodo a se stesso). La sua \emph{matrice di adiacenza} $A\in\R^{N\times N}$
è $a_{ij}:=\begin{cases}1 & \text{se }(i,j)\in\Gamma \\ 0 & \text{altrimenti}\end{cases}$. \\
Sia ora $\Sigma_A:=\set{x\in\set{1,\dots,N}^{\Z}:(x_i,x_{i+1})\in\Gamma}$, ovvero l'insieme dei cammini biinfiniti
su $\Gamma$, e poniamo $\sigma_A:=\restr{\sigma}{\Sigma_A}$ (restrizione dello shift sinistro). \\
Mettiamo su $\set{1,\dots,N}^{\Z}$ la distanza $d(x,y):=2^{-\min\set{\abs{i}:x_i\neq y_i}}$, che lo rende
uno spazio metrico compatto.

\begin{oss}Ovviamente $\sigma_A(\Sigma_A)\subseteq \Sigma_A$ e $\Sigma_A\subseteq\set{1,\dots,N}^{\Z}$ è chiuso,
dunque compatto. $(\Sigma_A,d,\sigma_A)$ è così un sistema dinamico topologico, che si chiama \emph{catena di Markov topologica}.
\end{oss}

Possiamo costruire esplicitamente misure invarianti. Per farlo ci serve il teorema di Perron-Frobenius.

\begin{defi}Data $A\in\R^{N\times N}$ con coefficienti $a_{ij}\ge 0$, diciamo che $A$ è \emph{irriducibile}
se per ogni $i,j$ esiste un $k\ge 0$ tale che $(A^k)_{ij}>0$. Diciamo che $A$ è \emph{primitiva} se
esiste $k\ge 0$ tale che $(A^k)_{ij}>0$ per ogni $i,j$.
\end{defi}

\begin{oss}Se $A$ è irriducibile, $I+A$ è primitiva (lo si vede sviluppando una potenza alta di $I+A$ con il binomio di Newton). \\
Se $A$ è primitiva e $A^k$ ha tutti i coefficienti $>0$, allora anche $A^h$ ha tutti i coefficienti $>0$ per ogni $k\ge h$.
\end{oss}

\begin{oss}$A$ è primitiva se e solo se $\Gamma$ è fortemente connesso, cioè se da ogni nodo si può raggiungere ogni altro nodo.
Infatti $(A^k)_{ij}>0$ se e solo se si può andare dal nodo $i$ al nodo $j$ con un cammino di esattamente $k$ passi.
\end{oss}

\begin{teo}[Perron-Frobenius]Se $A$ è primitiva, esiste un autovalore $\lambda_A>0$ tale che
\begin{itemize}
	\item $\abs{\lambda}<\lambda_A$ per ogni altro autovalore $\lambda\neq\lambda_A$
	\item gli autovettori destri e sinistri di $\lambda_A$ sono $>0$ (cioè hanno tutte le componenti positive) e sono unici
	\item $\lambda_A$ è una radice semplice del polinomio caratteristico $P_A(\lambda)$.
\end{itemize}
\end{teo}

Non dimostriamo Perron-Frobenius; vediamo piuttosto come si deduce il seguente

\begin{cor}Se $A\ge 0$ (coefficienti nonnegativi), esiste $\lambda_A\ge 0$ autovalore tale che
$\abs{\lambda}\le \lambda_A$ per ogni altro autovalore $\lambda$. Inoltre $\lambda_A$ ha un autovettore destro e un autovettore sinistro
con componenti $\ge 0$.
\end{cor}

\begin{proof}Sia $J\in\R^{N\times N}$ la matrice con tutti i coefficienti uguali a $1$. \\
$A_\epsilon:=A+\epsilon J$ è primitiva per ogni $\epsilon>0$. Sia $\rho(A_\epsilon)$ il raggio spettrale di $A_\epsilon$
(il massimo modulo degli autovalori). \\
$\epsilon\mapsto\rho(A_\epsilon)$ è continua, così come i coefficienti del polinomio caratteristico $P_{A_\epsilon}(t)$.
Quindi, da $P_{A_\epsilon}\pa{\rho(A_\epsilon)}=0$ (Perron-Frobenius), mandando $\epsilon\to 0$ otteniamo $P_A{\rho(A)}=0$.
Perciò $\rho(A)$ è autovalore di $A$ ed è il $\lambda_A$ cercato. \\
Infine sia $\epsilon_n\to 0$ una successione qualsiasi. Prendiamo per ogni $n$ un autovettore destro $v_n$
per $A_{\epsilon_n}$, con $\abs{v_n}=1$. A meno di sottosuccessioni, possiamo supporre $v_n\to v$ con $\abs{v}=1$
(compattezza di $S^{N-1}$). Scrivendo $A_{\epsilon_n}v_n=\rho\pa{A_{\epsilon_n}}v_n$ e mandando $n\to\infty$ deduciamo
$Av=\lambda_A v$; infine $v\ge 0$ essendo $v_n>0$. \\
L'esistenza di un autovettore sinistro con componenti $\ge 0$ è analoga.
\end{proof}

Ricordiamo che $v,w\in\R^N$ sono autovettori destro e sinistro di $\lambda_A$ se $Av=\lambda_A v$ e $w^t A=\lambda_A w^t$ rispettivamente.

Si può dimostrare che, se $A$ è primitiva ed è la matrice di adiacenza di un grafo, la catena di Markov
$(\Sigma_A,d,\sigma_A)$ ha entropia topologica $h_{top}(\sigma_A)=\ln \lambda_A$.

Torniamo alla costruzione di una misura invariante: fissato un nodo $i$, assegnamo ad ogni arco $(i,j)\in\Gamma$
una qualsiasi quantità $P_{ij}\ge 0$ in modo che $\sum_j P_{ij}=1$ (poniamo $P_{ij}=0$ se $(i,j)\nin\Gamma$).
$P_{ij}$ rappresenta la probabilità (assegnata da noi) di percorrere l'arco $(i,j)$ partendo da $i$. \\
$P$ è una matrice $N\times N$ e $P\ge 0$. Il suo raggio spettrale è $\rho(P)=1$:
se $Pv=\lambda v$ abbiamo $\abs{\lambda}\abs{v_i}=\abs{\sum_j P_{ij}v_j}\le\sum_j P_{ij}\abs{v_j}\le\max_j\abs{v_j}$
(essendo $\sum_j P_{ij}=1$), da cui $\abs{\lambda}\max_j\abs{v_j}\le\max_j\abs{v_j}$ e quindi $\abs{\lambda}\le 1$;
d'altro canto $(1,\dots,1)$ è autovettore destro di $1$. \\
Quindi per il corollario a Perron-Frobenius esiste un autovettore sinistro
$p=(p_1,\dots,p_N)$ per l'autovalore $1$. \\
Fissato $i_0\in\Z$, assegniamo all'insieme cilindrico
\[ C^{i_0,\cdots, i_0+k}_{j_0,\cdots, j_k}:=\set{x:x_{i_0}=j_0,x_{i_0+1}=j_1,\dots,x_{i_0+k}=j_k}\subseteq\set{1,\dots,N}^{\Z} \]
la misura $p_{j_0}P_{j_0j_1}\cdots P_{j_{k-1}j_k}$
(cioè assegniamo all'evento $\set{x_{i_0}=j_0}$ la probabilità
$p_{j_0}$ e consideriamo il movimento sul grafo come un processo di Markov).
Estendiamo a tutti gli insiemi cilindrici per additività: è facile vedere che la buona definizione della misura sui cilindrici
equivale al fatto che $p$ è autovettore sinistro di $P$
(infatti essendo $C^{i_0\cdots, i_0+k}_{j_0,\cdots, j_k}=\sqcup_{j=1}^N C^{i_0-1,i_0\cdots, i_0+k}_{j,j_0,\cdots, j_k}$
vogliamo che coincidano le quantità  $p_{j_0}P_{j_0 j_1}\cdots P_{j_{k-1}j_k}$ e $\sum_j p_jP_{jj_0}\cdots P_{j_{k-1}j_k}$,
ovvero vogliamo $\sum_j p_j P_{jj_0}=p_{j_0}$). \\
Il teorema di estensione di Kolmogorov permette di estendere il tutto a una misura di probabilità, che è ovviamente $\sigma_A$-invariante
e concentrata su $\Sigma_A$.

Si può dimostrare che l'entropia di Kolmogorov-Sinai di questo sistema misurabile è $-\sum_{i,j}p_iP_{ij}\ln P_{ij}$.



\end{document}
